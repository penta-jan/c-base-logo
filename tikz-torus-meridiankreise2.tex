\documentclass{standalone}
\usepackage{xcolor}
\usepackage[svgnames]{pstricks}
\usepackage{pst-solides3d}
\definecolor{eins}{HTML}{e7e7e8}   %% Ring 1 "core"     - weiß - Mittelpunkt, Ring um Mittelpunkt
\definecolor{zwei}{HTML}{ed1c24}   %% Ring 2 "com"      - rot -  "Fenster" innen
\definecolor{drei}{HTML}{fbad18}   %% Ring 3  "culture" - orange - fünf Module, quasi invertiert
\definecolor{vier}{HTML}{74c043}   %% Ring 4  "creactiv" - grün -  vier Module
\definecolor{fuenf}{HTML}{0089d0}  %% Ring 5 "cience"  - cyan (blau) - drei Module mit "Strich"
\definecolor{sechs}{HTML}{11357e}  %% Ring 6 "carbon" -  indigo - viele "Fenster" außen
\definecolor{sieben}{HTML}{000000} %% Ring 7 "clamp" -  schwarz, c-förmig
\definecolor{cbase}{HTML}{222222}  %% Körper der Raumstation    
\begin{document}

%% https://ftp.tu-chemnitz.de/pub/tex/graphics/pstricks/contrib/pst-solides3d/doc/pst-solides3d-doc.pdf

\begin{pspicture}(-4,-4)(4,4)
    \psset{Decran=20,viewpoint=5 11 25}
    \pstVerb{/iface 0 store}%
    \psSolid[
        fcol=48 {iface (green)
        iface 1 add (orange)
        iface 2 add (red) 
        iface 3 add (white) 
        iface 4 add (black) 
        iface 5 add (blue) 
        iface 6 add (blue) /iface
        iface 7 add store} repeat,
        r1=3,r0=2,
        object=tore,
        ngrid=7 14,
        RotY=30]
\end{pspicture}


\end{document}