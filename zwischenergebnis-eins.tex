\section{Zwischenergebnis}

Unter Berlin liegen die Überreste einer c-förmig aufgebauten Raumstation, die sich in sieben \ring{ringe} gliedert.

\ring{ringe} sind Aspekte oder Funktionen, die ineinander verschachtelt sind, miteinander kommunizieren, einander brauchen und sich gegenseitig stärken, und erst in zweiter Linie  Orte im Raum.

Sie sind \textbf{nicht} einander ausschließende Kategorien.
   
% Die kanonische Literatur über diese \ring{ringe} wird im Folgenden wiedergegeben und die Funktion der \ring{ringe} anschließend beschrieben. 

% Da unser Fokus auf der Funktion innerhalb des autopoietischen Systems \ceva{c-base} liegt, wird auf eine detaillierte Beschreibung einzelner Projekte, Module, Artefakte, Module und Bewohner wird verzichtet. 
    
% Eine sehr gute Übersicht über ausgewählte einzelne Projekte, Aggregate, Artefakte und Module bietet das kanonische \ceva{c-booc}~\cite{cbasebook}. Zur Funktion, Geschichte und Ausgrabungsgeschichte der Station und des Rolle des Vereins c-base e.V. sei auf die \ceva{Pressemappe}~\cite{cbasepressemap} verwiesen.
