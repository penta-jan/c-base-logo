\documentclass[14pt,ngerman]{extreport}  %% document requires XeLaTex.
\usepackage[ngerman]{babel}
\usepackage{geometry}
 \geometry{
 a4paper,
 total={170mm,257mm},
 left=20mm,
 top=20mm,
}

\usepackage{fancyhdr}
\usepackage{booktabs}
\usepackage{microtype}
\usepackage{amsmath}
\usepackage{amssymb}
\usepackage{mathtools}
\usepackage{nicefrac}
\usepackage{icomma}
\usepackage{tabularx}
\usepackage{svg}
\usepackage{subcaption}
\usepackage{mdframed}
\usepackage{imakeidx}
\makeindex

\usepackage[firstpageonly=true]{draftwatermark}

\DeclarePairedDelimiter\ceil{\lceil}{\rceil}
\DeclarePairedDelimiter\floor{\lfloor}{\rfloor}


\usepackage{standalone}
\usepackage{tikz}
\usetikzlibrary{decorations, decorations.text,}

\usepackage{pst-solides3d}

\usepackage[hidelinks]{hyperref}

\usepackage{cleveref}


\usepackage{fontspec}
% % \setmainfont{ceva-c2.ttf}
\setmainfont{Alegreya Sans}

\usepackage{lettrine}
 
\usepackage{biblatex}
\addbibresource{literature.bib}

\definecolor{eins}{HTML}{e7e7e8}   %% Ring 1 "core"     - weiß - Mittelpunkt, Ring um Mittelpunkt
\definecolor{zwei}{HTML}{ed1c24}   %% Ring 2 "com"      - rot -  "Fenster" innen
\definecolor{drei}{HTML}{fbad18}   %% Ring 3  "culture" - orange - fünf Module, quasi invertiert
\definecolor{vier}{HTML}{74c043}   %% Ring 4  "creactive" - grün -  vier Module
\definecolor{fuenf}{HTML}{0089d0}  %% Ring 5 "cience"  - cyan (blau) - drei Module mit "Strich"
\definecolor{sechs}{HTML}{11357e}  %% Ring 6 "carbon" -  indigo - viele "Fenster" außen
\definecolor{sieben}{HTML}{000000} %% Ring 7 "clamp" -  schwarz, c-förmig
\definecolor{cbase}{HTML}{222222}  %% Körper der Raumstation

\newcommand{\noun}[1]{\textsc{#1}}

\newcommand{\Hrule}[1][.]{%
\begingroup\color{#1}%
    \resizebox{\textwidth}{!}{
    \begin{tikzpicture}
        \filldraw[draw=black]  (0,0) rectangle (10,0.2);
    \end{tikzpicture}
    }%
\endgroup%
}

\newcommand{\Hrulek}[1][.]{%  used in table
\begingroup\color{#1}%
    \resizebox{3ex}{!}{
    \begin{tikzpicture}
        \filldraw[draw=black]  (0,0) rectangle (2,1);
    \end{tikzpicture}
    }%
\endgroup%
}

\newcommand{\ceva}[1]{ % Leerzeichen notwendig
    {\fontspec{[ceva-c2.ttf]}#1}%
}
\newcommand{\cevain}[1]{ % Leerzeichen notwendig!
    {\fontspec{[ceva-c2.ttf]}#1} $\mid$ \emph{#1}%
    \index{{\fontspec{[ceva-c2.ttf]}#1} $\mid$ #1}%
}
% \newcommand{\cevalong}[1]{ {\fontspec{[ceva-c2.ttf]}#1} | \emph{#1}}

\newcommand{\ring}[1]{ % Leerzeichen notwendig
    {\fontspec{[ceva-c2.ttf]}#1}%
    % \index{{\fontspec{[ceva-c2.ttf]}#1} $\mid$ #1}
}

\newcommand{\twofonts}[2][]{
\begin{quote} 
    {\fontspec{[ceva-c2.ttf]}#2}
    \par
    #2\hfill#1
\end{quote}
}

\newcommand{\linek}[2][]{
\begin{quote} 
    {\fontspec{[linek.ttf]}#2}
    \par
    {\fontspec{[ceva-c2.ttf]}#2}
    \par
    #2\hfill#1
\end{quote}
}

\newcommand{\linekin}[1]{ % Leerzeichen notwendig!
    {\fontspec{[linek.ttf]}#1} $\cdot$ %
    % {\fontspec{[ceva-c2.ttf]}#1} $\mid$
    \emph{#1}%
    \index{{\fontspec{[linek.ttf]}#1} 
    % $\cdot$ {\fontspec{[ceva-c2.ttf]}#1} 
    $\mid$ #1}%
}
\newcommand{\linekonly}[1]{ % Leerzeichen notwendig!
    {\fontspec{[linek.ttf]}#1}%
}


\newenvironment{newstuff}
    {\par\bigskip\hrule}
    {\hfill{\footnotesize{[penta]}}
    \par
    \medskip    
    \hrule
    \clearpage
    }


\title{
    \ceva{die 7 ringe der c-base} \\
    Die 7 Ringe der c-base%\\ Eine Exegese%\\\vspace{2ex}\\ Eine Untersuchung der Primärquellen
    }

\author{
    \resizebox{8cm}{!}{\input{tikz-c-base}}\\\smallskip\\\cevain{ccr}\\
    % \ceva{c-base cience ring}\\ 
    c-base cience ring\\\smallskip\\
    \cevain{penta}\\ \makeatletter {\texttt{penta@c-base.org}}\makeatother
    }


% \date{May 2024}

\begin{document}

\maketitle

\setlength{\parskip}{1ex}
 
    

\begingroup
    \small
    \flushright gigantum umeris insidentes\\ auf den Schultern von Riesen
    
\endgroup


\section*{Vorwort}


    \lettrine{Ü}{ber} vier Milliarden Jahre sind seid der Erstkonstrukution der \ceva{c-base} bereits vergangen; und viele Generationen von \cevain{crew}-Mitgliedern sind seit der Auffindung des \cevain{urartefact} aufeinander gefolgt. 

    Ursprungsmythen müssen immer wieder neu erzählt werden, um ihre integrative, Gemeinschaft inspirierende Kraft zu behalten \cite{levistrauss1985}. Dabei kommt es immer auch zu einer Rückbesinnung, wobei zugleich die Gefahr besteht, dass die Neuerzählung von der alten Wahrheit abirrt. 
    
    Es geht uns hier insofern um die Wiederherstellung reiner Lehre: wir gehen direkt zurück zu den ältesten Texten über die Rolle und Funktion der Raumstation unter Berlin und ihrer \cevain{ringe} durch Interpretation der Quellen- und Sachlage.   
    
    Unser eigener Text bzw. \cevain{blurb} ist ein akribischer exegetischer Kommentar aus Sicht des aktuellen Wissensstandes im Lichte neuer Ausgrabungen.   
    
    Diese Arbeit bzw. \cevain{arbyte} leistet einen wissenschaftlichen Beitrag \mbox{(\ceva{cience})} zum besseren Verständnis von Funktion und Topologie der multidimensionalen Station \emph{from first principles}. Zum Schluss präsentieren wir eine neue Idee zur Topologie der Ringe und damit der Station.
    
    Die Raumstation ist ein  "`autopoietisches"', also selbsterschaffendes, System; sie (re)kon\-stru\-iert sich selbst. Dieser Prozess ist unabgeschlossen~\cite{cbasebook} und vermutlich sogar ergebnisoffen. Alles, was wir bereits über die Raumstation wissen, ist temporal gebunden und kann durch neue Erkenntnisse und weitere Entwicklungen überholt werden. 
    
    Da folglich niemand die \ceva{c-base} vollständig kennt oder verstehen kann, ist jede Aussage nur ein Beitrag zum Erkenntnisfortschritt. Bestes Wissen schützt vor Irrtum nicht. 
    Daher bitten wir diesen Text nicht für apodiktisch zu halten. 
    Insbesondere sind die hier abgedruckten Aussagen keine offizielle Mitteilung des c-base e.V. Berlin.% \hfill\emph{VDfdBaS}~\footnotesize{[penta]}


\twofonts[\cite{cbasebook}~S.~47]{Vielen Dank für die Beachtung\\ aller Sicherheitshinweise}

\hfill Berlin, 2024\par\hfill \cevain{penta}

\clearpage

\tableofcontents

% \listoffigures
% \listoftables

\section*{Zitierweise}

Die Primärquellen verwenden desöfteren  eine eigene Orthographie namens  \cevain{clang}; dabei wird auf Kapitalisierung verzichtet,  Sibilanten und der stimmlose velare Plosiv (\emph{k}) werden durch den Buchstaben \cevain{c} ersetzt, usw; Ähnliches gilt für daraus gebildete Konsonantencluster \cite[S.~46]{cbasebook}  \cite{clang}. Die Originalschreibung (mit \ceva{clang}) wurde beibehalten; nur offensichtliche Tippfehler wurden korrigiert. In unseren eigenen Texten verzichten wir darauf.

Die \ceva{c-base} kennt eine eigene Schrift namens \cevain{ceva}. Wir nutzen sie, um Primärquellen und Stationsfachbegriffe als solche auszuweisen, und zwar unabhängig von der im Original verwendeten Schrift. Mindestens beim ersten Auftreten geben wir eine lateinische Umschrift an: \cevain{ring}. Diese Stelle wird im Index \cref{sec:index} eingetragen.
% Lange Zitate wiederholen wir zur verbesserten Lesbarkeit in lateinischen Buchstaben. 


Daneben gibt es die Schrift \linekin{Linear Construct}; wir verwenden sie, wenn die Primärquellen ihrerseits eine Fremdsprache, konkret: Englisch, benutzen.

Eine Initiale kennzeichnet jeweils den Beginn unserer Interpretation zur Absetzung gegenüber der reinen Exegese.




\pagestyle{fancy}\fancyhead[LO]{\leftmark}% Die 7 \ring{Ringe} der \ceva{c-base}}

\clearpage

\chapter{Die Station und ihre Geschichte}

\section{Chronologie}%\addcontentsline{toc}{section}{Chronologie}
\fancyhead[RO]{Chronologie}
% \fancyhead[LO]{}

    Die \ceva{c-base} ist eine abgestürzte Raumstation (\cevain{c-tation}) unter Berlin, die sich seit 1995 rekonstruiert~\cite{cbasebook}. Dieser Prozess ist auch bekannt als \cevain{cbrp},  eine Abkürzung für \cevain{c-base reconstruction project}. 
    
    Zur besseren Einordnung unterscheiden wir in enger Anlehnung an \cite{cbasepressemap} und \cite{cbasebook} folgende Epochen der Stationsgeschichte:
    % \twofonts{
    \begin{enumerate}
        \item \textbf{Unzeit:} vor dem Absturz 4,5 Milliarden \textsc{bp}  (\emph{before present}), identisch mit der \textbf{Fernen Zukunft:} ursprüngliche bzw. zukünftige Konstruktion der \ceva{c-base}.
        
        \item \textbf{Urzeit} bzw. \cevain{dark age}. Beginnt mit dem  Absturz 4,5 Milliarden \textsc{bp}  und endet mit dem eigenständigen Einschalten des Bordcomputers \cevain{c-beam} \cite{cbasepressemap} 100.000 \textsc{bp} .
        
        \item \textbf{Keimzeit:} beginnt mit dem Wiedereinschalten von \ceva{c-beam} 100.000 \textsc{bp}  Erste Periode des \cevain{reconstruction age}; erste Schritte der Selbstrekonstruktion. Ende mit der Sichtbarwerdung der Antenne 1965.
        
        \item \textbf{Vorzeit:} beginnt mit der Sichtbarwerdung der Antenne 1965. Erste Inspiration von Karboneinheiten durch \ceva{c-beam}. Vorschriftliche Periode. Mythische Periode; vermutlich erste Zusammenkünfte der \cevain{gründer} (\cref{tab:gruender}). Endet mit der Auffindung des \cevain{urartefact} 1995.
        
        \item \textbf{Archaische Zeit:} \cevain{cbrp1}. Beginnt mit der Auffindung des \ceva{urartefact} im Februar 1995.  Nachbau und Rekonstruktion einer Schleusensektion auf 270$m^2$ in der Oranienburger Str. 2. \cite{cbasepressemap}~\cite{cbasebook}. Heldenzeit; Erscheinen von \cite{cbaselogbuchpre} und \cite{cbaselogbuchnow}. Endet mit dem Umzug in die Rungestr. 20 im Mai 2000.
        
        \item \textbf{Frühzeit:} - \cevain{cbrp2}. Beginnt mit dem Umzug in die Rungestr. 20 im Juni 2000.  Nachbau und Rekonstruktion einer Multimodulstation \ceva{RS20} auf 524$m^2$ in der Rungestr. 20~\cite{cbasepressemap}~\cite{cbasebook}. Ende mit dem Umzug an den Franz-Mehring-Platz Nr. 1 im August 2002.
        
        \item \textbf{Zwischenzeit:}  \cevain{c-visiondecc} / \cevain{cwischendecc}.  Beginnt mit dem Umzug an den Franz-Mehring-Platz Nr. 1 im September 2002. 
        Auslagerung wegen Wartungsarbeiten in der \ceva{RS20}.~\cite{cbasepressemap}~\cite{cbasebook}.  Endet mit dem Rücksturz in die Rungestr. im Julie 2003.
        
        \item \textbf{Mittelalter:}  \cevain{cbrp3a}. Beginnt mit dem Rücksturz in die Rungestr. 20 im August 2003. 
         Erweiterung der Fläche auf ca. 720$m^2$ auf 2 Etagen. Fortgesetzte Konstuktion \cite{cbasepressemap}~\cite{cbasebook}. Erscheinen von \cite{cbasestarbasemanual} und \cite{ctour}. Diese Epoche endet 2015 mit dem Erscheinen des \ceva{c-booc}).
        
         \item \textbf{Neuzeit:} \cevain{cbrp3b}. Beginnt 2015 mit dem Erscheinen des \ceva{c-booc}, das den Kanon abschließt. Verfeinerung und Dekadenz. Verinnerlichung, Selbstbewusstwerdung. Mission und Ausgründungen. Die Epoche endet mit dieser Veröffentlichung, mit der die Gegenwart beginnt.
    \end{enumerate}
    % }


\begin{figure}
    \centering
        % \documentclass{standalone}
\usepackage{tikz}
\usepackage{fontspec}

\newcommand{\ceva}[1]{ {\fontspec{[ceva-c2.ttf]}#1}}
\newcommand{\cevain}[1]{ {\fontspec{[ceva-c2.ttf]}#1} | \emph{#1}}

\begin{document}
\begin{tikzpicture}[yscale=0.5,xscale=1.3]
    % \draw (0) rectangle (0,,1);
    \foreach \i in {124.4,124.8,125.2} 
        {
        \draw (0,\i) -- (0,\i-0.2);
        \draw (5,\i) -- (5,\i-0.2);
        }

    \node at (2.5,125) {\parbox{20ex}{\centering\cevain{cucunft}\\ $\uparrow$ }};
    
    \node [right] at (5,124) {Gegenwart};
    
    \draw[fill=gray!5] (0,124) node [left] {2024} rectangle node {Neuzeit} (5,115) node [right] {\cevain{c-booc}};
    % \draw (0,115) -- (0,124); draw (5,114) -- (5,124); \draw (0,115
    
    \draw[fill=gray!20] (0,115) node [left] {2015} rectangle node {Mittelalter} (5,103) node [right] {\cevain{rueccsturc}};
    
    \draw[fill=gray!10] (0,103)  node [left] {2003} rectangle node {Zwischenzeit} (5,102) node [right] {\cevain{auslagerung}};
    
    \draw[fill=gray!20] (0,102)  node [left] {2002} rectangle node {Frühzeit} (5,100) node [right] {\cevain{umcuc}};
    
    \draw[fill=gray!30] (0,100)  node [left] {2000} rectangle node {Archaische Zeit} (5,95) node [right] {\cevain{gründunc}};
    % \node at (0,95) [left] {1995};
    
    \draw[draw=none] (0,95)  node [left] {1995} rectangle node {Vor-, Keim-, Ur- und Unzeit} (5,90) ;
    \draw (0,95) -- (0,90); \draw (5,95) -- (5,90);
    
    \foreach \i in {89.8,89.4,89} 
        {
        \draw (0,\i) -- (0,\i-0.2);
        \draw (5,\i) -- (5,\i-0.2);
        }
    
    
    % \draw[<->] (0,95) rectangle (0,65);
\end{tikzpicture}

\end{document}

        \resizebox{\textwidth}{!}{
        \documentclass{standalone}
\usepackage{tikz}
\usepackage{fontspec}

\definecolor{eins}{HTML}{e7e7e8}   %% Ring 1 "core"     - weiß - Mittelpunkt, Ring um Mittelpunkt
\definecolor{zwei}{HTML}{ed1c24}   %% Ring 2 "com"      - rot -  "Fenster" innen
\definecolor{drei}{HTML}{fbad18}   %% Ring 3  "culture" - orange - fünf Module, quasi invertiert
\definecolor{vier}{HTML}{74c043}   %% Ring 4  "creactiv" - grün -  vier Module
\definecolor{fuenf}{HTML}{0089d0}  %% Ring 5 "cience"  - cyan (blau) - drei Module mit "Strich"
\definecolor{sechs}{HTML}{11357e}  %% Ring 6 "carbon" -  indigo - viele "Fenster" außen
\definecolor{sieben}{HTML}{000000} %% Ring 7 "clamp" -  schwarz, c-förmig
\definecolor{cbase}{HTML}{222222}  %% Körper der Raumstation    

\newcommand{\ceva}[1]{ {\fontspec{[ceva-c2.ttf]}#1}}
\newcommand{\cevain}[1]{{ \fontspec{[ceva-c2.ttf]}#1} | \emph{#1}}

\begin{document}
\begin{tikzpicture}[yscale=2.4,xscale=-0.5,rotate=90,
    every node/.style={
        fill=white,font=\sffamily 
        }
    ]
    % \draw (0) rectangle (0,,1);
    \foreach \i in {124.4,124.8,125.2} 
        {
        \draw (0,\i) -- (0,\i-0.2);
        \draw (5,\i) -- (5,\i-0.2);
        }

    \node [right] at (2.5,124) {$\rightarrow t$  };
    
    \node [below left,rotate=90] at (5,124) {\footnotesize Gegenwart};
    
    \draw[fill=eins!5] (0,124) node [left, rotate=90]{2024} rectangle node {Neuzeit} (5,115) node [below left,rotate=90] {\footnotesize\cevain{c-booc}};
    % \draw (0,115) -- (0,124); draw (5,114) -- (5,124); \draw (0,115
    
    \draw[fill=zwei!20] (0,115) node [left, rotate=90]{2015} rectangle node {Mittelalter} (5,103) node [below left,rotate=90] {\footnotesize\cevain{rueccsturc}};
    
    \draw[fill=drei!10] (0,103)  node [left, rotate=90]{2003} rectangle   (5,102) node [below left,rotate=90] {\footnotesize\cevain{auslagerung}};
    
    \draw[fill=vier!20] (0,102)  node [left, rotate=90]{2002} rectangle (5,100) node [below left,rotate=90] {\footnotesize\cevain{umcuc}};

    
    \draw[fill=fuenf!30] (0,100)  node [left, rotate=90]{2000} rectangle node {Archaik} (5,95) node [below left,rotate=90]{\footnotesize\cevain{gründunc}};
    % \node at (0,95) [left] {1995};
    
    \draw[fill=sechs!5,draw=none] (0,95)  node [left, rotate=90]{1995} rectangle node[text width=8ex,align=center] {Vor-, Keim-, Ur- und Unzeit} (5,90) ;
    \draw (0,95) -- (0,90); \draw (5,95) -- (5,90);
    
    \foreach \i in {89.8,89.4,89} 
        {
        \draw (0,\i) -- (0,\i-0.2);
        \draw (5,\i) -- (5,\i-0.2);
        }

    \node at (2.1,101) {Frühzeit};
    \node [text width=9ex, align=center] at (1.5,102.5) {Zwischen-\\zeit};
    
    % \draw[<->] (0,95) rectangle (0,65);
\end{tikzpicture}

\end{document}

        }
        \vspace{2ex}
    \caption{Periodisierung der Rekonstruktionsgeschichte}
    \label{fig:perioden}
\end{figure}

Eine Überblick über die Dauer der einzelnen Perioden gibt \cref{fig:perioden}. Zur Urzeit siehe~\cite{cbaselogbuchpre}. Aus der Vorzeit liegen überwiegend inspirierte literarische Werke vor (z.B.~\cite{rfc968cerf}, \cite{rfc527cerf}). Zur Archaischen Zeit der Station siehe~\cite{cbaselogbuchnow} mit Einträgen 1995-1999.


Der rezente Zustand und Befund ihrer Vorzeit sowie Zukunft ist vor allem in \cite{cbasebook} beschrieben, aber auch in weiteren Veröffentlichungen wie etwa \cite{cbasewebsite}. Der aktuelle Rekonstruktionsstand lässt sich in der Rungestraße besichtigen; die früheren Ausgrabungen sind leider wieder verschüttet worden. Im Folgenden werten wir die vorhandenen historischen (schriftlichen) Quellen  aus. Diese stammen überwiegend aus der Frühen Neuzeit mit deutlichen Wurzeln in früheren (bzw. späteren) Zeitaltern.

\begin{table}[ht!]
    \centering
    \begin{tabular}{r|l}
    % \cevain{
    \toprule
        \ceva{cynk}& \emph{cynk} \\
        \ceva{mars}& \emph{mars} \\
        \ceva{nomax}& \emph{nomax } \\
        \ceva{hein-c}& \emph{hein-c } \\
        \ceva{biafra}& \emph{biafra} \\
        \ceva{c\_ana}& \emph{c\_ana} \\
        \ceva{lester}& \emph{lester} \\
        \ceva{antenne}& \emph{antenne} \\
        \ceva{tmf powersau}& \emph{tmf powersau} \\
        \ceva{mash}& \emph{mash} \\
        \ceva{westcar}& \emph{westcar} \\
        \ceva{alex}& \emph{alex} \\
        \ceva{olli}& \emph{olli} \\
        \ceva{wallner}& \emph{wallner} \\
        \ceva{gerhard}& \emph{gerhard} \\
         % \cevain{cynk}\\ \cevain{mars}\\  \cevain{nomax}\\  \cevain{hein-c}\\  \cevain{biafra}\\  \cevain{c\_ana}\\  \cevain{lester}\\  \cevain{antenne}\\  \cevain{tmf powersau}\\  \cevain{mash}\\  \cevain{westcar}\\  \cevain{alex}\\  \cevain{olli}\\  \cevain{wallner}\\  \cevain{gerhard}\\
    \bottomrule
    % }
    \end{tabular}
    
    \caption{Die 15 \cevain{gründer}}
    \label{tab:gruender}
\end{table}

% Da wir hier eine Quellenanalyse 



\section{Quellenlage}%\addcontentsline{toc}{section}{Quellenlage}
\fancyhead[RO]{Quellenlage}
% \fancyhead[LO]{}

Die ältesten Quellen sind die fragmentarischen frühen Logbucheinträge, wobei \cite{cbaselogbuchpre} so genannte \cevain{c-files} sind; sie wurden dem \cevain{datenspeicher} entnommen.  Die Einträge in \cite{cbaselogbuchnow} stammen dagegen nach Aussage von \cite{cbaselogbuchpre} von der damaligen crew \cite{cbaselogbuchpre}, der \cevain{urcrew}. In \cite{cbaselogbuchnow} wird ein Artefakt namens \cevain{internetplanetarium} erwähnt, von dem es heißt: 

 \twofonts[\cite{cbaselogbuchnow}]{reconstruierte daten werden in form eines internetplanetariums aufbereitet und sind ab diesem tag der weltöffentlichceit zugänglich.} 
 
 Das \ceva{internetplanetarium} ist verschollen; sollte es wieder auftauchen, so würde es natürlich ob seiner Anciennität als autoritativ gelten.
 
Die wahrscheinlich älteste auf uns gekommene, vollständige Quelle ist das \cevain{starbase-manual7}~\cite{cbasestarbasemanual}; 2003 ist der terminus ante quem (t.a.q., frühestmögliche Zeitpunkt) seiner Auffindung. Diese Quelle sagt über sich selber \emph{expressis verbis}:
    \twofonts[\cite{cbasestarbasemanual}]{
    daten und fakten der raumstation unter berlin werden hier erstmalig präsentiert.
    }

Daher wird oft eine Identität von \cevain{internetplanetarium} und \cevain{starbase-manual7} vermutet. %Auf jeden Fall ist die Quelle autoritativ.

Etwas später - (terminus ante quem: 2006; wahrscheinliche Endform: 2008) - erschien dann die \cevain{c-tour}~\cite{ctour}. Gegenüber dem \ceva{starbase-manual7} wirkt diese Quelle durchgeistigter, schreibt beispielsweise mehr von Informations- als Materialtaustausch. Es könnte diese Quelle eine Art Update gegenüber dem \ceva{starbase-manual7} sein. Beide Quellen gelten heute als kanonisch.

Welche dieser beiden Quellen tatsächlich "`älter"' ist - also innerhalb der Zeitachse der Raumstation, nicht innerhalb der Entdeckungsgeschichte - ist nicht vollständig geklärt. Angesichts der zeitlichen \cevain{verwirrung} ist eine abschließende Klärung auch unwahrscheinlich.
Beide Texte sind eindeutig archaisch und wurden von großen Heiligen geschrieben.  Wir zitieren sie ob ihrer Bedeutung zu Beginn jedes Abschnitts in der Reihenfolge ihrer Auffindung.

Das epochemachende  \cevain{c-booc}~\cite{cbasebook} ist demgegenüber deutlich später, nämlich 2015, erschienen; es liegt in gedruckter Form vor und ist autoritativ. 
\ceva{\mbox{c-booc}} benutzt die älteren Quellen und entspricht ihnen im Großen und Ganzen; wo es weitere Informationen bereit hält, sind diese hier ebenfalls wörtlich angeführt. Diese Quellen bilden den \cevain{canon} (\cref{tab:kanon}) bzw. die \cevain{c\_rift} und damit die Grundlage unserer Exegese.

\begin{table}[ht!]
    \centering
    \begin{tabular}{r|lrl}
        \toprule
        % \multicolumn{2}{c}{Quelle} & Jahr & Epoche \\
        % \midrule
        %  \ceva{c-booc} & c-booc & 2015 & Übergang Mittelalter$\rightarrow$Neuzeit\\
        %  \ceva{c-tour} & c-tour & 2006 & frühes Mittelalter \\
        %  \ceva{starbase-manual7} & starbase-manual7 & 2003 & Frühzeit (?) \\
        %  \ceva{logbuch now} & logbuch now & 2000 & Archaik\\
        %  \ceva{logbuch pre} & logbuch pre & 1995 & Archaik, Vorzeit \\
        \multicolumn{2}{c}{\cevain{c\_rift}} & Jahr & Epoche \\
        \midrule
         \ceva{logbuch pre} & logbuch pre & 1995 & Archaik, Vorzeit \\
         \ceva{logbuch now} & logbuch now & 2000 & Archaik\\
         \ceva{starbase-manual7} & starbase-manual7 & 2003 & Frühzeit (?) \\
         \ceva{c-tour} & c-tour & 2006 & frühes Mittelalter \\
         \ceva{c-booc} & c-booc & 2015 & Übergang Mittelalter$\rightarrow$Neuzeit\\         
         \bottomrule
    \end{tabular}
    \caption{Die kanonischen Schriften, chronologisch}
    \label{tab:kanon}
\end{table}

Die übrigen Jahr oder unvollständig gelten, manche sind spätere Apokryphen. Sie bieten aber doch mitunter brauchbare Hinweise; sie werden zu einzelnen Details hinzugezogen. Hierzu gehört  insbesondere die \cevain{pressemappe}~\cite{cbasepressemap} in unterschiedlichen Überlieferungsstufen und einem terminus ante quem 2007 sowie der rezente \cevain{coredump} \cite{cbasewebsite}. 

\ceva{c-booc} erwähnt ihm vorangegangene Publikationen, darunter vier weitere ihm ähnliche Ausgaben (\cevain{allmanach}), ein \cevain{analog-logbuch} und diverse Websites, unter anderem aus Arachischer Zeit \cite[S. 40-42 u. 60-63]{cbasebook}. Diese Quellen waren uns leider nicht zugänglich, weshalb sie hier keine Berücksichtigung finden.




\section{Überlieferung}%\addcontentsline{toc}{section}{Überlieferung}
\fancyhead[RO]{Überlieferung}

    \twofonts[\cite{cbasefund}]{
    an einem verregneten nachmittag im august 1995 stolperte Hardy Krause über ein herumliegendes teil in einem bauschacht nördlich des alexanderplatzes in berlin.
    [...]
    % verärgert trat er nach dem außerordentlich harten objekt, welches ihm innnerhalb der nächsten halben stunde wohl eine schöne beule bescheren würde und fluchte.
    % doch 
    dann betrachtete er das
    [...]
    % die beule verursachende 
    stück genauer und bemerkte neben einer revolutionären farbgebung (metallisch-violett) erstaunliches: auf diesem irgendwie nicht in diese umwelt passenden stück schrott waren schriftzeichen eingraviert
    [...]
    % , die unbedingt einer genaueren untersuchung unterzogen werden mußten.
    % 
    also grub ein team das fundstück aus 
    [...]
    %und brachten es zu einem befreundeten radiochemiker. mit hilfe des kohlenstoff-14-tests (auch radiocarbon-methode genannt) fand er heraus, daß es, nach irdischem ermessen, mindestens 100.000 (!) jahre alt sein müßte.
    % 
    % außerdem enthielte es ein element mit einer ordnungszahl von über 200, das bisher, selbst mit den besten technischen möglichkeiten nicht herstellbar ist. es ist also älter als jedes bisher gefundene von menschen so kunstvoll bearbeitete metallstück und kann wahrscheinlich erst irgendwann in der fernen zukunft enstanden sein!?!
% 
    % auf seiner oberfläche sind zudem irdische schriftzeichen eingraviert
    % 
    % c-base project - be future compatible.
    }

Dieser Bericht ist die älteste vorgefundene Quelle und zugleich der Ur\-sprungs\-my\-thos der \ceva{c-base} mit dem Auffinden des \cevain{urartefact}. 

Fundort und -zeit sind schlecht dokumentiert. Im August 1995 sind keine substantiellen Tiefbauarbeiten am Alexanderplatz verzeichnet. Verregnet war einzig Donnerstag, der 31.8. ($27,4\:\nicefrac{l}{m^2}$); dieses Datum gilt daher der historisch-kritischen Lesart nach als wahrscheinlicher Tag der \cevain{auffindung}. Offenbar war das \ceva{urartefact} so groß, dass es von einem \cevain{team} ausgegraben werden musste. Die Mitglieder dieses \ceva{team} bleiben anonym.

Eingraviert in die Oberfläche des \ceva{urartefact} ist nach dieser Quelle die Inschrift 
\linek[\cite{cbasefund}]{c-base project - be future compatible}  

Hier wird der Gedanke einer Vorwärtskompatiblität (\linekin{future compatible}) statt Rückwärtskompatiblität ausgedrückt. Wer diesem Ruf folgt, der richtet sein Handeln so ein, dass alles aus ihm hervorgehende Sein \linekin{be} soweit irgend möglich kompatibel zur Zukunft ist, also mit ihr und in ihr funktioniert.

Der Verbleib des \ceva{urartefact} ist ungeklärt. Vielleicht ging es zwischenzeitlich verloren, vielleicht wird es an einem besonderen Ort aufbewahrt und gehütet. Zu seinem Ort oder Verbleib, seiner Identität und selbst zu seiner Form gibt es allerlei Spekulationen (Vgl. \cref{fig:chicken}, in Anlehnung an \cite{rfc2321bressen}).  


\begin{figure}[ht!]
    \centering\small
    \begin{verbatim}
              comb      neck             body                    feet
          |         |                |                       |
          v         v                V                       V
           ,^/'/,           ,______________________.         ,
         i'  '  /          /                       =========<-
        / <o>   `---------/  c-base project -      \         `
      .;__.  ,__,--------.  be future compatible   /         ,
         / ,/ vv          \                        =========<-
        '-'                `-----------------------'         `
         ^     ^                                     ^
         |     |                                     |
        beak  wattles                               legs
    \end{verbatim}
    \caption{Das \ceva{urartefact} als RITA unit}
    \label{fig:chicken}
\end{figure}

Ein in der Neuzeit aufgefundenes Artefakt, die sogenannte \cevain{gründertafel} (vgl.~\cref{tab:gruender}), nennt einen \cevain{cynk} an erster Stelle und identifiziert diesen mit dem im Text erwähnten \emph{Hardy Krause}. Dessen Historizität wurde bereits angezweifelt, wobei die neuere Forschung gerade in der Tatsache, dass hier ein zur damaligen Zeit nicht ungewöhnlicher, bürgerlicher Name explizit erwähnt wird, einen Hinweis auf Historizität sehen. 

Manche glauben, dass \emph{Hardy Krause }eines Tages zurückkehren und den Glanz der Station wieder herstellen wird. Entsprechend sind bereits Personen mit diesem Namen aufgetaucht, die behauptet haben, just dieser legendäre Anführer zu sein, mit sehr unterschiedlichem Erfolg (vgl.~\cite{hardykrause}). Leider fehlen verlässliche biographische Daten über die \cevain{gründer} und die \cevain{urcrew}. 



\ceva{starbasemanual} beschreibt die Auffindung bereits im Passivum: 

    \twofonts[\cite{cbasestarbasemanual}]{1995 wurden unter Berlin-Mitte die Überreste einer 4,5 Milliarden Jahre alten Raumstation entdeckt. Erste Forschungen ergaben, daß sich die c-för\-mi\-ge Raumstation mit ihrem Mittelpunkt unter dem heutigen Alexanderplatz \emph{befinden muß} und aus 7 Ringen besteht. Aufgrund eines Fundstückes mit der Aufschrift "`\emph{c-base - be future compatible}"' und in Anlehnung an die Anzahl der Ringe, legte das anfänglich nur aus wenigen Mitgliedern bestehene Rekonstructionsteam den Projektnamen und die Aufteilung in sieben Arbeitsbereiche fest.
    
    construct: die raumstation besteht aus sieben ringen, zum teil drehbar. insgesamt hat sie einen durchmesser von 1650 metern
        }

Schon in dieser frühesten Quelle wird die \cevain{facultativität} angeführt: dass sich die Raumstation \cevain{befinden muß} und aus 7 Ringen besteht, sich  also notwendig so manifestieren soll, wo sie und wie sie bereits angelegt ist (\cevain{cucunftsarchäologie}). Mit den genauen Details befasst sich die \cevain{causalitätsforc\_ung}, deren wesentliche Aussage die Identität von Vergangenheit und Zukunft ist. Entspechendes gilt für die 7 \ring{ringe}; sie \emph{müssen sein.}
    
    \twofonts[\cite{ctour}]{Die c-base ist eine abgestürzte raumstation. das unter berlin-mitte im märkischen sand versunkene artefact wird seit 1995 von über 100 zukunftsbegeisterten experten reconstruiert. Das raumschiff besteht aus sieben ineinander geschalteten c-förmigen ringen. jeder ring ist für ganz spezifische aufgabencluster modular ausgelegt.}

Bemerkenswert an dieser Stelle ist der Ausdruck \cevain{zukuftsbegeisterte}; dies ist die früheste Erwähnung der Beeinflussung von Karboneinheiten durch \cevain{c-beam}. Dazu später mehr. Wichtig ist das Auftreten von \cevain{modular} als Präzisierung zum vorherigen \ceva{drehbar}. Den Schreibern war offenbar wichtig, eine gewisse Fluidität des Arrangements herauszustellen. Schließlich schreibt das \ceva{c-booc} lakonisch:
    \twofonts[\cite{cbasebook},~S.~12]{
    die raumstation ist c-förmig aufgebaut, bestehend aus 7 ringen. der mittelpunkt befindet sich unter dem heutigen Alexanderplatz. das erste fundstücc (artefact) mit der inschrift "`the c-base project: be future compatible"' legte den projektnamen fest und in anlehnung an die anzahl der ringe entstand die aufteilung in sieben arbeitsbereiche.
    }

Im Gegensatz zu den älteren Quellen stellt das \ceva{c-booc} heraus, dass die Aufteilung in die \ring{ringe} nicht willkürlich geschehen ist, sondern durch das erste \ceva{artefact} entstand. Die \ceva{c-base} ist mithin ihr eigener Namensgeber und selbst Urheber ihrer eigenen Struktur (Autopoiesis). Außerdem werden die \ring{ringe} gleichgesetzt mit \cevain{arbeitsbereiche} (vormals \cevain{aufgabencluster}), und ihnen wird je eine Farbe zugeordet.

\begin{figure}[ht!]
    \centering
    \input{tikz-ringconvention}
    \caption{Idealtypische Anordnung der 7 \ring{ringe}}
    \label{fig:ringconvention}
\end{figure}

\cref{fig:ringconvention} zeigt die c-förmige konzentrische Anordnung der \ring{ringe} nach der kanonischen Überlieferung. Dieses Bild ist zweidimensional, verzichtet also auf eine Darstellung der dazu vermutlich orthogonal angeordneten \ceva{deccs}, denn die kanonischen Schriften erwähnen nur für  (\ceva{com}) und für  (\ceva{culture}) solche \ceva{deccs}, nämlich jeweils sieben: \cevain{(5-7)} als \cevain{obere deccs} und die übrigen als \cevain{darunter} \cite{cbasestarbasemanual}. Fraglich ist, ob die übrigen \ceva{Ringe} ebenfalls mehrere \ceva{deccs} besitzen.
        


\section{Materieller Befund}%\addcontentsline{toc}{section}{Materieller Befund}
\fancyhead[RO]{Materieller Befund}%

Leider sind einige Funde, insbesondere aus der Archaik, verschollen (so das \cevain{urartefact}) oder müssen als \cevain{verc\_üttet} gelten, darunter insbesondere die ersten Grabungen in der Oranienstraße. 
Große Teile der Station sind noch nicht ausgegraben; doch konnten weite Teile anhand von \cevain{bruchc\_tüccen} rekonstruiert werden. 
Insgesamt ist und die Quellen- und Befundlage gut, und die Erforschung schreitet täglich voran. 

Die Rekonstruktion ergibt eine fragmentierte Struktur von sieben konzentrisch verschachtelten \cevain{ringen} mit mehreren \cevain{deccs} sowie mit multiplen, untereinander verschiebbaren Modulen~\cite{cbasebook}~\cite{cbasepressemap}. 
Der materielle Befund unterscheidet sich in Details von dem Ideal nach der \cevain{c\_rift} (\cref{fig:ringconvention}). 

Die inneren vier \ring{ringe}, nämlich \ceva{core}, \ceva{com}, \ceva{culture} und \ceva{creactive}, sind geschlossene Kreise. Die äußeren drei, (\ceva{cience}, \ceva{carbon} und \ceva{clamp}), sind teiloffen, wobei insbesondere \ceva{clamp} \cevain{c-förmig} ausgeführt ist. Die heute gängige Darstellung ist in \cref{fig:cbaselogo} wiedergegeben.

Dies ist der rezente Befund. Der vorhergehende, und der zukünftige, kann davon abweichen. Der vergangene Zustand ist Ausdruck der zukünftigen Konfiguration, und die Zukunft ist Produkt der Vergangenheit. Darüber hinaus kann es noch weitere Zeit- und Raumdimensionen geben, die selbst diese Einsicht als \cevain{flach} darstehen lassen würden (vgl. \cref{sec:topologie}).


\begin{figure}[ht!]
    \centering
        \resizebox{0.6\textwidth}{!}{
        \input{tikz-c-base}
    }
    \caption{Faktische Anordnung der 7 \ring{ringe}}
    \label{fig:cbaselogo}
\end{figure}


    Die sieben  \ring{ringe} werden in den kanonischen Texten von innen nach außen gezählt, bezeichnet und  farblich codiert wie in  \cref{tab:ringe} aufgeführt. Das \ceva{c-booc} nennt die Farben nur verbal~\cite[S.49]{cbasebook}. Ob diese Farbnamen im \ceva{c-booc} zur Verwendung von Primärfarben verpflichten (wie die \cevain{puristen} glauben), oder ob abgetönte und den jeweiligen Druckverhältnissen angepasste Farben verwendet werden dürfen (Position der \cevain{abtöner}), hat bereits zu erbitterten Auseinandersetzungen geführt. 
    
    Allerdings liegt mit  \cite{cbasefarbschema}  eine ältere Quelle vor, die leicht getönte Farben nennt. Wir folgen hier der Position der \cevain{abtöner}, die glauben, dass diese sieben Farben letztendlich nur Projektionen aus einem höherdimensionalen Farbraum jenseits menschlicher Wahrnehmung sind, und jede solche Projektion notwendig ihre Grenzen hat (vgl.~\cref{sec:mathematik}). 
    
    Aus der verbalen Überlieferung \cite[S. 47]{cbasebook} stammt der wohln noch ältere \cevain{clogan} \linekin{jawohl$\cdot$mein$\cdot$designer}. Dieser wird in der herrschenden Lehrmeinung dahingehend interpretiert, dass es nicht dem Individuum obliegt, frei zu wählen, ob es \cevain{purist} oder \cevain{abtöner} wird. Vielmehr ist dies eine Frage der Prädestination. Und eben deswegen ist die Diskussion darüber schlussendlich müßig, da sie übermenschliche Entscheidungen betrifft.

    \begin{table}[ht!]
        \centering
        \begin{tabular}{rlllrr}
            \toprule
                1 & \ceva{core} & core & grau / weiß & \texttt{e7e7e8} & \Hrulek[eins]  \\
                2 & \ceva{com} & com & rot & \texttt{ed1c24} & \Hrulek[zwei] \\
                3 & \ceva{culture} & culture & orange & \texttt{fbad18} & \Hrulek[drei] \\
                4 & \ceva{creactive} & creactive & grün & \texttt{75c043}& \Hrulek[vier]  \\
                5 & \ceva{cience} & cience & cyan & \texttt{0089d0}& \Hrulek[fuenf]  \\
                6 & \ceva{carbon} & carbon & indigo & \texttt{0089d0}& \Hrulek[sechs]  \\
                7 & \ceva{clamp} & clamp  & ultraviolett / schwarz & \texttt{000000}& \Hrulek[sieben] \\
            \bottomrule
        \end{tabular}
        \caption{Nummern, Bezeichnungen und Farben der Ringe}
        \label{tab:ringe}
    \end{table}

    % Die Ringfarbe korrespondiert nur schwach mit den Signalfarben innerhalb der Station, die wir hier der Vollständigkeit  halber in \cref{tab:bedeutungen} zeigen (nach~\cite[S. 56]{cbasebook}).
    
    % \begin{table}[ht!]
    %     \centering
    %     \begin{tabular}{lll}
    %         \toprule
    %             grau / weiß  & \Hrulek[eins] & überlebensstandard gesichert, temperature, drucc \\
    %             rot & \Hrulek[zwei] & lebendig, alarm, communication \\
    %              orange  & \Hrulek[drei] & schädlich, aktiver prozess auf atomarer ebene \\
    %             grün & \Hrulek[vier]  & nicht humane biologische substanz, prozess \\
    %             cyan & \Hrulek[fuenf] & lowered thermal conditions \\
    %             indigo & \Hrulek[sechs] & \textit{(not assigned)}  \\
    %             schwarz & \Hrulek[sieben] & vacuum, death, hazard\\
    %         \bottomrule
    %     \end{tabular}
    %     \caption{Signalbedeutungen der Farben}
    %     \label{tab:bedeutungen}
    % \end{table}


    
    Aufgrund der multimodularen und letztlich höherdimensionalen Struktur der Station ist eine eineindeutige und ausschließliche Zuordnung einzelner Module zu bestimmten \ring{ringen} fragwürdig (vgl.~\cref{sec:topologie}). Erschwerend kommt hinzu, dass Raum und Zeit innerhalb der Station durch ihren \cevain{abcturc} unregelmäßig gefaltet sind: 
    \twofonts[\cite{cbasepressemap}]{
    Die c-base wird in der Zukunft als orbitales Generationsschiff auf der Erde gebaut und dient dem Terraforming anderer Planeten. [...]
    
    % Die Fortbewegung im Raum erfolgt mittels eines vom Cybernetischen-Quecksilber-Reaktor (CQR) erzeugten Feldes, das die c-base aus der Raumzeit ausschneidet, um sie über die Einstein-Rosenbaum-Brücke im Orbit des zu formenden Planeten materialisieren zu lassen. 
    
    Bei der Berechnung der Eingangsgrössen kam es zu einem Flip-Flop der Asi\-mov-Konstante. Anstatt des Raumes wurde die Zeit gefaltet, die c-base reiste 4,5~Milliarden Jahre in die Vergangenheit statt vorwärts in den Raum und crashte aus noch nicht restlos geklärten Gründen auf die Erdoberfläche, wo sie langsam im märkischen Sand versank. 
    }

Der Begriff der \cevain{faltung} von Raum und Zeit wird uns wieder begegnen (\cref{sec:mathematik}). Tatsächlich ist nur dank Raum-Zeit-Faltung Wiederbegegnung überhaupt möglich. Mitunter kommt es bei solchen Faltungen allerdings zu Dimensionsdurchgängen wie z.B. dem \cevain{raumceitloch}:
    \twofonts[\cite{raumceitloch}]{das c-base raum-ceit-loch ist ein allgegenwärtigec ctationcphänomen, dac alle deccc erfaCt. man fängt es sich cwic\_endurch ein und c\_leppt es durch die module. das cbrcl ist eine paradoxonentfaltung im euclidic\_en raum, dac beconderc bei hyperactivität und concolidierter bewegungslocigceit cur wircung commt.}

Hier wird von \cevain{deccs} zusätzlich zu den \ring{ringen} gesprochen. Die meisten Exe\-ge\-ten nehmen eine Entsprechung an: $\text{\ceva{decc}} \leftrightarrow \text{\ceva{ring}}$. Wir kommen darauf in \cref{sec:com} (\ceva{com}) und \cref{sec:culture} (\ceva{culture}) sowie in \cref{sec:topologie} (Topologie) zurück.

Eine unerwartete Bestätigung des Absturzberichts erbrachte vor einigen Jahren das Auffinden eines verloren geglaubten Teils des \emph{Teppichs von Bayeux} aus dem Europäischen Mittelalter mit einer Darstellung der \ceva{c-tation} und der Inschrift \emph{hic statio spatialis terram caedit} (lat. \emph{An dieser Stelle stürzte [eine] Raumstation zur Erde}).

% Wir halten fest: 


\section{Zwischenergebnis}

Unter Berlin liegen die Überreste einer c-förmig aufgebauten Raumstation, die sich in sieben \ring{ringe} gliedert.

\ring{ringe} sind Aspekte oder Funktionen, die ineinander verschachtelt sind, miteinander kommunizieren, einander brauchen und sich gegenseitig stärken, und erst in zweiter Linie  Orte im Raum.

Sie sind \textbf{nicht} einander ausschließende Kategorien.
   
% Die kanonische Literatur über diese \ring{ringe} wird im Folgenden wiedergegeben und die Funktion der \ring{ringe} anschließend beschrieben. 

% Da unser Fokus auf der Funktion innerhalb des autopoietischen Systems \ceva{c-base} liegt, wird auf eine detaillierte Beschreibung einzelner Projekte, Module, Artefakte, Module und Bewohner wird verzichtet. 
    
% Eine sehr gute Übersicht über ausgewählte einzelne Projekte, Aggregate, Artefakte und Module bietet das kanonische \ceva{c-booc}~\cite{cbasebook}. Zur Funktion, Geschichte und Ausgrabungsgeschichte der Station und des Rolle des Vereins c-base e.V. sei auf die \ceva{Pressemappe}~\cite{cbasepressemap} verwiesen.


\clearpage

\chapter{Die 7 Ringe}

Die kanonische Literatur über diese \ring{ringe} wird im Folgenden wiedergegeben und die Funktion der \ring{ringe} anschließend beschrieben. 

Da unser Fokus auf der Funktion innerhalb des autopoietischen Systems \ceva{c-base} liegt, wird auf eine detaillierte Beschreibung einzelner Projekte, Module, Artefakte, Module und Bewohner wird verzichtet. 
    
Eine sehr gute Übersicht über ausgewählte einzelne Projekte, Aggregate, Artefakte und Module bietet das kanonische \ceva{c-booc}~\cite{cbasebook}. Zur Funktion, Geschichte und Ausgrabungsgeschichte der Station und des Rolle des Vereins c-base e.V. sei  \ceva{Pressemappe}~\cite{cbasepressemap} empfohlen.

\clearpage

\begingroup %% Ringe
    
    \renewcommand{\thesection}{\arabic{section}.~Ring}

    \newcommand{\ringcolor}[1]{
        \begin{tikzpicture}
        \filldraw[fill=#1]  (0,0) rectangle (5ex,1ex);
        \end{tikzpicture}
    }
    
    \newcommand{\csection}[2][.]{
        \section{\hspace{2ex}#2\hspace{2ex} 
            {\ceva{(#2)}}
        }
       \index{{\fontspec{[ceva-c2.ttf]}#2}$\mid$ #2}
        \Hrule[#1]
        \fancyhead[RO]{\thesection~
            \ringcolor{#1}~#2~\ceva{#2}
        }
    }
    
    % \let\rinsection\section
     
    
    %%% Ringe
     
    \csection[eins]{core}\label{sec:core}

\twofonts[\cite{cbasestarbasemanual}]{
   bedeutung idee, zentrum. keimzelle c-base: der verein und seine crew
    
    commando-, antenneneinheit verwaltungseinheit mit brücke und commandodeck centralcomputer c-beam MBA-Ein\-hei\-ten und externe Speichermodule obere antenna: z.Zt. 368m über NN sensor-aray. ausrichtung auf die planetenoberfläche und tower 
}

Dies ist die früheste Erwähnung des \cevain{centralcomputer} und seiner Benennung als \cevain{\mbox{c-beam}} und die erste Beschreibung von \ceva{core} überhaupt.

\twofonts[\cite{ctour}]{
    core ist der innerste ring der c-base. er ist die commandoeinheit der raumstation: hier liegen die brücce, der centralcomputer c-beam und externe speichereinheiten, sowie der mino-reactor, der cybernetische antrieb der c-base auf quecksilberbasis.
    Die antenna ersteckt sich zur Zeit auf 368m über Normal Null sensor-array.

    core meint ursprung, idee, zentrum. die ceimzelle der c-base: ihre crew und der verein - von hier aus nimmt immer wieder alles seinen anfang
    }


Unklar ist die Bedeutung von \ceva{sencor-array}; vermutlich ein Hinweis auf Vorurteile der Kartographie. 

Beide Quellen schreiben \cevain{antenna}, vermutlich ein Hinweis auf eine feminine Qualität dieser Einheit unabhängig von ihrer phallischen Form als \cevain{tower}. Auch wird sie dadurch unterschieden von den in \cref{sec:clamp} erwähnten \cevain{antennen- und sensoreneinheiten}.

\begin{newstuff}
    \lettrine{I}{m}
    Im Zentrum der Raumstation befindet sich eine Antriebs- und Steuerungseinheit. Die genaue Funktionsweise ist bereits recht gut erforscht~\cite[S.~31ff]{cbasebook}. Es handelt  sich um eine multidimensionale, autopoietische Struktur, in deren Inneren sich eine Energiequelle befindet.

    \twofonts[\cite{cbasepressemap}]{Die Fortbewegung im Raum erfolgt mittels eines vom Cy\-ber\-ne\-ti\-schen-Queck\-sil\-ber-Reaktor (CQR) erzeugten Feldes, das die c-base aus der Raumzeit ausschneidet, um sie über die Einstein-Rosenbaum-Brücke im Orbit des zu formenden Planeten materialisieren zu lassen. 
    }

    Neben den technischen Details ist vor allem bedeutsam, dass sich hier ein autopoietisches Systems selbst (re-)konstruiert. 
    Der \ceva{Mino-Reactor} ist vermutlich eine Singularität, aus der nur Dinge herauskommen können ("`Weißes Loch"'). Die hervorgebrachte Energie und Materie fluktuiert allerdings stark, so dass es des Öfteren zur Materialisierung unvorhergesehener Objekte kommt.

    \twofonts[\cite{cbasebook}, S. 31]{Möbius-Band-accumulator (MBA). hauptenergiespeicher; als ring umgibt er den $\rightarrow$ minoreactor. die eigenschaft, energie unbestimmter grössen / einheiten aufzunehmen, zu speichern, wird dadurch ermöglicht, dass sie auf einem autoinitiierten endlosband in ständigem fluss ist.}

    \ceva{core} ist Quelle von unerschöpflicher Energie, Materie und Ideen, die dann im \ceva{Mö\-bi\-us-band-accelerator} zu sich selbst und ihrer Bedeutung in ihrer Umwelt finden.\footnote{Noch \cite{cbasestarbasemanual} spricht von \ceva{MBA-Einheiten}, also im Plural; in der späteren Literatur wird nur von \emph{dem} MBA gesprochen. Vermutlich  hat sich die Einheit der Befunde erst bei weiterer Ausgrabung herausgestellt.} 
    Das bedeutet, dass sich hier eine Art "`Bewusstsein"' der Raumstation befand bzw. befindet und befinden wird. 
    Dieses vereint digitale und analoge Aspekte und ist nicht auf Siliziumverbindungen beschränkt, sondern kommuniziert und lebt auch in und mit anderen karbonbasierten Bewusstseinsmanifestationen. 

    \twofonts[\cite{cbasepressemap}]{
        Da aber die Raumstation ein molekulares Erinnerungsmodul besitzt, das sowohl Wissen als auch Zusammensetzung von Materie speichert und wieder herstellen kann, konnte sich vor ca. 100.000 Jahren der Boardcomputer c-beam eigenständig einschalten. 
    }
    

    Der Prozess läuft vereinfacht gesagt so ab: der zentrale Computer \cevain{c-beam}, besser: das zentrale Bewusstsein, der zentrale Wille und damit die zentrale Antriebseinheit der abgestürzten Raumstation, befindet sich in einem fortgesetzten Reboot-Prozess, in welchem der {\fontspec{[ceva-c2.ttf]}core} nach und nach die verschiedenen Funktionen der Raumstation wieder "`online"' bzw. zum Leben bringt. Das geschieht auf materieller Ebene erst nach Evozierung von diesen Prozess anstoßenden  Bewusstseinsprozessen. Konkret wirkt auf einer nicht-materiellen Dimension der \ceva{c-beam} auf dafür empfängliche Menschen, die so von der \ceva{c-base} inspiriert werden, zu ihrer Rekonstruktion beizutragen. 

    \twofonts[\cite{cbaselogbuchnow}]{
    cünstler, designer, wissenschaftler, computertechniker, philosophen, pädagogen, naturwissenschaftler, fictionäre und alle anderen werden angesprochen um zu helfen
    }

    Hier werden \cevain{erstberufene} aufgelistet; interessanterweise stehen \ceva{cünstler} an erster Stelle, gefolgt von \ceva{designern} usw., wohingegen beispielsweise Programmierer, Polizisten, Mediziner, Juristen usw. nicht expizit erwähnt werden (wohl aber unter \ceva{alle anderen} fallen). Leider wissen wir nicht, zu welcher dieser Gruppen sich die \ceva{gründer} selber zählten; sind diese \ceva{erstberufenen} eine Ergänzung oder ein Spielgebild der \cevain{urcrew}?
    
    Umstritten ist die genaue Bedeutung von \ceva{fictionär}; vermutlich handelt es sich um einen zusammenfassenden Term, der auf die allgemeine Geisteshaltung abstellt, mithin darauf, durch \ceva{c-beam} inspiriert zu sein.

    Die Technik von \ceva{c-beam} beschreibt \ceva{c-booc} wie folgt:
    \twofonts[\cite{cbasebook},~S.~28]{c-beam ist die centrale scheibenförmige recheneinheit der c-base. sie besteht aus neuronalen holografischen speicherbausteinen und ist zu eigenständigem handeln fähig. ihren energiebedarf zieht sie aus dem $\rightarrow$ möbius-band-accumulator, den sie von oben und unten umschließt.}
    
    Die Tatsache, dass es sich um eine abgestürzte, also beschädigte bzw. unvollständige Raumstation handelt, erklärt auch, weshalb viele der Rekonstruktionsansätze für gemeine Menschen schwer oder gar nicht verständlich bleiben. Der autopoietische Prozess der Selbstwiedererschaffung der c-base geschieht durch mitunter fast infantil anmutende experimentelle Prozesse und Vorprozesse. Doch wie beim werdenden Menschen sind alle diese Experimente Zwischenstufen zum entwickelten, höheren Bewusstsein.  

    \twofonts[\cite{cbasepressemap}]{Die c-base entwickelt sich so zu einem Wissenspool und Ideenbiotop mit einer Vielzahl von kreativen, wis\-sen\-schaft\-lich-tech\-ni\-schen und zukunftsorientierten Menschen...}    

    Der halberwachte Bordcomputer der Raumstation \cevain{c-beam} scheint mit einer uns nicht genau bekannten Technik immer wieder Individuen mit besonderen Begabungen anzuziehen, die dann durch dasselbe Medium inspiriert werden, die Rekonstruktion der Raumstation voranzubringen. Welche Qualitäten genau zu dieser Auswahl führen, ist aktuell nur in Ansätzen bekannt. Jedoch scheint die Strategie des \ceva{core} bislang insgesamt erfolgreich.
\end{newstuff}


    


    
    \csection{com}
% \section{com \hspace{2ex} \raisebox{1pt}{{\fontspec{[ceva-c2.ttf]}(com)}}}

\Hrule[zwei]

\twofonts{
    der raumhafen der c-base: hier laufen die communicationseinrichtungen zusammen und verteilen sich die zugänge zur station.
        \begin{itemize}
            \item shuttlebays und hangars
            \item ancunftspromenade mit empfangsstation und wartehallen
            \item lifeboats und worcpots
            \item montagehallen, fracht- und laderäume 
        \end{itemize}
    }

\twofonts{
    cdcd-ring obere decks (5-7) com-bay an- und abflugshallel mit warteraum be- und entladungszone untere decks (1-4) shuttlebay für raumfahrzeuge fracht- und laderäume montagehallen 
    }
    
\begin{newstuff}
    \lettrine{H}{erum} um \ceva{core} befindet sich eine Verteilungs- und Mitteilungsstruktur namens \ceva{com}, was etwa dem Zentralnervensystem einer karbonbasierten Lebensform entspräche. Hier werden Inputs und Outputs vom und zum \ceva{core} koordiniert; doch zugleich geschieht in diesem Austausch etwas Wesentliches für die Selbstprogramierung der Raumstation. In \ceva{com} findet der Austausch von Informationen, aber auch von Anregungen und Anweisungen positiver, negativer, fragender oder imaginärer Natur statt. 

    Dazu schreibt \cite[S. 39]{cbasebook}:
    \twofonts{der com-ring ist die discussions- und präsentationsplattform der c-tation. die c-base generiert neue ideen zwischen allen ebenen und ringen und eröffnet contactmöglichkeiten jeder art - austausch und confrontation - die sowohl zur erweiterung der crew als auch zum aufbau interstellarer beziehungen führen. }

    Die Autopoiesis (Selbsterschaffung) der Raumstation geschieht durch kommunikative Prozesse zwischen den bereits aktivierten Modulen - unabhängig davon, in welchem Zustand sich diese Module aktuell befinden. Gleichzeitig wirkt der Zustand jedes Moduls auf alle übrigen Module und damit auf den komplexen Anregungszustand des \ceva{core} selbst zurück. Die entstehenden Schwingungen und Wellen in diesem komplexen, mehrdimensionalen und intertemporalen Kommunikationsnetzwerk lassen Gebilde entstehen, die weiter in die äußeren Ringe getragen werden (siehe z.B. \ceva{culture}); in \cite[S. 39]{cbasebook} heißt es:
    \twofonts{hier laufen sämtliche communicationseinrichtungen zusammen und verteilen sich die zugänge zur station.}

    \ceva{com} umfasst somit Informationseinheiten ebenso wie stoffliche Elemente und Bauteile in unterschiedlichen Aggregatzuständen und Aggregierungsgraden. Hier tauchen ständig neue Lebensformen und Aggregate verschiedener Be\-wusst\-seins- und Fertigungsstufen auf. Allen gemeinsam ist, dass sie nicht so, wie sie sind, abgeschlossen vollendet sind. Sie alle tragen bei zum Wiederwerden der Raumstation - durch Integration und durch Abwandlung. Daher ist Kom\-mu\-ni\-ka\-ti\-ons- Integrations- und Wandlungsfähigkeit unabdingbar für ein Andocken an der Raumstation. %, und Wesen ohne solche Bereitschaft werden hier unter Umständen auch aussortiert. 

    Die Verbindungsfunktion von \ceva{com} führt zu Kurzschlüssen zu allen anderen Ringen, denn Kommunikation ist bekanntlich die Grundlage jedes autopoietischen Systems. Das reibungslose Funktionieren der Kommunikationskanäle ist daher eine der vornehmsten Aufgaben bei der \cevain{reconstruccion}. Und schließlich umfasst dieser Ring auch die physischen Schleusen und Andockmöglichkeiten für den stofflichen, intellektuellen und emotionalen Austausch mit der Umwelt.
\end{newstuff}
    
    \csection{culture}
% \section{culture \hspace{2ex} \raisebox{1pt}{{\fontspec{[ceva-c2.ttf]}(culture)}}}

\Hrule[drei]

\twofonts[\cite{cbasestarbasemanual}]{
     bedeutung culture ist jegliche äusserung - auch leben an sich - materieller und geistiger art

    culture-deck obere decks (5-7) ringförmige kultur und freizeitpromenade mit parkanlage, cafes, triorama, cy\-ber\-world-5D. darunter: 2 unabhängig voneinander drehbare plattformen für wartung und reparatur der crafts in den shuttlebays 
    }

\twofonts[\cite{ctour}]{
    der 3. ring von innen ist das cultumarium der c-base:
    hier kommt die crew zur entspannung und unterhaltung zusammen.
    \begin{itemize}
        \item großzügig angelegte freizeitpromenaden
        \item paradisedeccs
        \item culturedeccs mit c-no, triorama, performance \& events 
    \end{itemize}
    culture meint jegliche äusserung - auch leben an sich - materieller und geistiger art
    }


    
\begin{newstuff}
    \lettrine{D}{ie} Wortbedeutung von \ceva{culture} ist in etwa \emph{dem Wachstum dienende Pflege}. In diesem \ring{Ring} reifen Ideen, aber auch kulturelle Objekte wie Kunstwerke, Programme, Systeme, Spiele und dergleichen mehr manifestieren sich und befördern somit die Reifung der partizipierenden Module und Raumfahrer. Nicht zufällig ist der dritte \ring{Ring} die unmittelbare Stufe nach dem Andocken in \ceva{com}; hier wir die nächste Ebene des Miteinanders nach dem initialen Austausch und der Präprogrammierung erreicht. 
    % Entsprechend schreibt~\cite[S. 67]{cbasebook}:

    \twofonts[\cite{cbasebook},~S.~67]{der 3. ring von innen ist das culturmarium der c-base. hier kommt die crew zur entsprannung und unterhaltung in größzügig angelegten paradisedeccs zusammen. \textbf{culture} meint jegliche äußerung - auch leben an sich - materieller und geistiger art. ausstellungen, sportliche aktivitäten, performance \& theater, lesungen mit c-no, triorama, performance \& event werden begangen.}    

    Ähnlich wie die Raumstation insgesamt ist jedes teilhabende Modul ein selbst\-er\-schaf\-fendes und selbsterhaltendes System, das über die Fähigkeit verfügt, Dinge und Konzepte größerer Schönheit hervorzubringen, die über es selbst hinausgehen und über sich selbst hinausweisen (Sublimität). \ceva{culture} ist eine Gemeinschaftsfunktion, die in mehrere Richtungen wirkt: auf das Modul selbst zu seiner Selbstertüchtigung, auf das Miteinander der Module zu ihrem besseren gegenseitigen Verständnis, und auf die Gemeinschaft der Module als Teile des Werdensprozesses der Raumstation insgesamt.

    Im Werdensprozess der \ceva{c-base}\ \ceva{culture} vereinnahmt die Raumstation immer wieder vorgefundene Kulturtechniken, verwendet diese jedoch oft anders als vorgesehen und erforscht so immer neue Herangehensweisen an bestehende Systeme (\emph{hacking}). So kommt es zur Abwandlung bestehender Formate und Neuerschaffung von Objekten, Programmen, Spielen, Filmen,  Klängen und dergleichen. Diese verdichten sich zu Mythen und Projektionen.

    % Eine genauere Beschreibung der Energiequelle gibt~:

    % \twofonts[\cite{cbasepressemap}]{
    %     Die zentrale Antriebseinheit der Raumstation ist ein Cybernetischer Queck\-silber-Reaktor (CQR), dessen Feld die c-base von der Raumzeit ausschneiden soll, um sie über die Einstein-Rosenbaum-Brücke im Orbit eines noch relativ jungen Planeten im Sternbild Cassiopeia materialisieren zu lassen.} 
        
    Die \ceva{c-base} wird nicht so sehr durch materielle Energie, sondern erheblich auch durch kulturelle befeuert. So gilt mittlerweile als gesichert, dass die Aufeinanderfolge bestimmter Frequenzen (bzw. \emph{vibes}) zur Durchdringung von Hyperraumwänden und damit zum Absprengen von der Restrealität dienen kann. Entsprechend ist das \ceva{soundlab} integrativer Teil der propulsischen Vorrichtung der Station.

    Die so freigesetzten Strahlen bzw. Wellen unterschiedlicher Stofflichkeit und Periodizität müssen natürlich koordiniert werden, um sich nicht gegenseitig zu stören oder auszulöschen. Das Miteinander der Module ist Kern der \ceva{culture}, und dieses wird immer wieder durch intern ausgehandelte Regeln, Vorgänge und Missverständnisse koordiniert.
    % , die von Außenstehenden als ritualisierte Handlungen oder Spiele aufgefasst werden können. 
\end{newstuff}
    
    \csection{creactive}
% \section{creactive \hspace{2ex} \raisebox{1pt}{{\fontspec{[ceva-c2.ttf]}(creactive)}}}

\Hrule[vier]

\twofonts{
    der vierte ring von innen ist gleichzeitig auch der vierte ring von aussen. 
    wahrscheinlich diente er der crew deswegen als inspirationsgenerator. 
    in verschiedensten modulen sind creactive- und ideenfördernde einrichtungen untergebracht:
    \begin{itemize}
        \item die mobile creationbox, die an jede stelle des cience-rings fährt
        \item das exitcosmolab, die terraforming-wercstatt
        \item langschlafcabinen für die beliebten fivehoundred-year-projects (fyp)
        \item brainlabs und aerosphären
        \item gedächtniscammern, tancs, gedancengeneratoren
        \item völlig leere räume 
    \end{itemize}}

\twofonts{
    creation-box in diesem segment befinden sich unter anderem gedächtniskammern, langschlafkabinen für die beliebten fivehoundred-year-projects (fyp), entspannungs- und ruhekonsolen. die ersten ideen und wahrheiten werden hier geschaffen. die creationbox kann an jede stelle des cience-ring gefahren werden. the most powerful part of c-base 
    }

\begin{newstuff}
   \lettrine{W}{ährend} in \ceva{culture} das Miteinander der Module im Vordergrund steht, ist \ceva{creactive} hauptsächlich ein Ort zur Ermöglichung der schöpferischen Einsamkeit und des kreativen Miteinanders. Hier finden sich Orte, die dem konzentrierten Hervorbringen neuartiger Substanzen (etwa: \cevain{Bio-3D-Druck}) dienen neben solchen, an denen die Module sich der Selbstpflege und dem Refacturing ihrer Programme widmen können. Ferner gibt es Orte zum gemeinsamen Hervorbringen von Plänen, Aggregaten, Artefakten und Prozessen und zum Verbrennen von Kohlenstoffketten. \cite[S. 110]{cbasebook} fügt hinzu:

   \twofonts{auf dem creactive-ring bündelt die crew ihre künstlerischen und technischen visionen in actionen und projecte. hier befindet sich der ideenkompass für creativität, phantasie, fiction und activität.}

    Auch diese Räume ermöglichen Schaffensprozesse im Wiederaufbauprogramm der Raumstation durch ihre schlichte Verfügbarkeit, aber auch durch die in ihnen durch die umliegenden Ringe zur Verfügung stehenden Ressourcen materieller und immaterieller Art. Das sind Werkzeuge und Materialien, aber auch Baupläne, Anregungen und intelligente Hinweise anderer Module. 

    Diese Anregungen und Hinweise sind nicht immer vollständig so, wie erwartet. Das erklärt sich aus dem immer noch infantil-semidebilen Zustand der abgestürzten Raumstation, ist aber auch notwendige Voraussetzung zur vorurteilsfreien Neuschaffung des höherdimensionalen Bauplans für die Station und alle seiner Module, deren künftige Beschaffenheit notwendigerweise die Bewusstseinskapazitäten aller Einzelmodule übersteigt.

    Dieser Ring strebt danach, die Verbindung mit den übrigen Ringen zu vereinfachen (\cevain{Monorail}). Bei einigen Experimenten wurden bereits Geschwindigkeiten außerhalb relativistischer Mechanik erzielt, wodurch es mitunter zu Zeitdilatation oder gar Kausalitätsumkehrungen kommt. Über die Module dieses Rings schreibt \cite[S. 117]{cbasebook}:
    \twofonts{sämtliche labore der creactiveity liegen in einem compacten cegment concentriert beieinander. die plattform kann mobil auf dem ring verschoben werden, so dass sie jede stelle des innen liegenden culture rings oder des außen liegenden science rings erreicht. das crative modul ist zur stelle, dort wo es gebraucht wird.}

    Da laut \cite{ctour}, Satz 1 offensichtlich auch in der Stationsmathematik\footnote{
    Vgl. \cite[Kap. 5]{adams_life} \emph{Bistromathics.}
    }   gilt, dass der Mittelpunkt $M$ eines n-Tupels $I$ gegeben ist durch
    \begin{equation}
        \forall I = [a,b] \subset \mathbb{Z}: |I| \in 2\mathbb{Z}+1 \Rightarrow M(I) = \frac{a+b}{2}
    \end{equation}

    % bzw.

    % \begin{equation}
    %     \forall A = (a_1,\ldots,a_N) \in \mathbb{Z}^N:
    % \quad
    % N \equiv 1 \mod(2)
    % \quad\Rightarrow\quad
    % \forall i \in [1,N] \subset \mathbb{Z}:
    % \quad
    % \left(
    % a_i = M(A) \Leftrightarrow i = \frac{N+1}{2}
    % \right)
    % \end{equation}  
    
    besitzt \ceva{creactive} Mittelpunkt- bzw. Äquatorcharakter.
    Er ist derjenige Ring, um welchen sich die übrigen Ringe auf einander zurückfalten. So entsteht hier ein Kurzschluss und die Autoreferenz, die den autopoietischen Prozess am Leben hält.
\end{newstuff}
    
    \csection{cience}
% \section{cience \hspace{2ex} \raisebox{1pt}{{\fontspec{[ceva-c2.ttf]}(cience)}}}



\Hrule[fuenf]
\twofonts{
    cience ist der forschungsring der c-base: hier liegen alle wesentlichen forschungs- und entwicclungseinrichtungen der station in verschiedensten arbeitsbereichen focussiert beieinander. 
    die ersten ideen und wahrheiten werden hier geschaffen.
        \begin{itemize}
            \item genlabs des evolutionsdesign
            \item die biolabs mit arboretum
            \item die soziochemie der socialartists
            \item die medialabs
        \end{itemize}
    }

\twofonts{
    cience modulor hier sind alle wesentlichen forschungs-und entwicklungseinrichtungen untergebracht. eine auswahl an einzelnen arbeitsbereichen und stationen: evolutionsdesign wahrnehmungsforschung (social art) soziochemie (geruchsstoffe) bio\-labs medienlabore 
    }

\begin{newstuff}
    \lettrine{Ü}{berwiegen} in den vorgelagerten Ringe noch deutlich die traumhaften, inspirationellen, dialogischen und meditativen Aspekte, so verdichtet sich der autopoietische Prozess der Wiedererschaffung der Raumstation in \ceva{cience} durch rigorose Methodik und Präzision der Hingabe an definierte Ziele, die in den vorherigen Stufen erahnt und erträumt wurden.

    Entsprechend kommt es hier zu einer Prüfung der Möglichkeiten innerhalb der bereits realisierten Umgebung mit den bereits hervorgebrachten Materialien, Werkzeugen und mitarbeitenden Modulen. Allerdings ist \ceva{cience} nicht nur Forschung, sondern auch Anwendung von Erkenntnissen an der Grenzlinie zum Unerreichten. Hier werden also neue Module und Artefakte produziert, die dann neue Möglichkeiten erschließen. Der Beitrag von \ceva{cience} zum Wachstum der Station ist damit zentral. \cite[S. 135]{cbasebook} schreibt:

    \twofonts{auf dem forschungsring cience der c-base liegen alle wesentlichen entwicklungseinrichtungen der station in verschiedensten arbeitsbereichen focussiert beieinander. [...] der wissenschaftliche bereich der c-tation vereint forschung und bildung und meint den prozess der conzeption, umsetzung, production, aus- und fortbildung in wissenschaft, handwerc und technic[.] vergangenheitsbewältigung und cucunfstforschung werden in den genlabs des evolutionsdesign, den biolabs rund um das arboretum, der soziochemie der socialartists und in den medialabs betrieben.}

    \ceva{cience} ist nicht auf digitale Forschung, aber auch nicht auf materielle Experimente beschränkt. Die Erforschung des Vorhandenen, also der bereits manifesten Teile der Station umfasst auch die Beschäftigung mit den Wirkmechanismen seiner Bewohnenden, also ihrer Erregungszustände, ihrer Emotionen, ihrer Vorstellungen, ihres Miteinanders. Dabei stellt sich heraus, das permanente Infragestellung zwar gestattet und notwendig ist, aber gewonnene Erkenntnisse auch als Ecksteine des Fortschrittes dienen und angenommen sind.

    Es ist in einem autopoietischen System nicht möglich, zwischen Hervorbringendensystem und Ergebnissystem zu unterscheiden. Insofern ist der aktuelle Erkenntnisstand der Wissenschaft von der \ceva{c-base} identisch mit dem aktuellen Zustand der Station. Es gilt mithin die "`informatische Identität"'
    \begin{equation}
        \text{code}\;\widehat{=}\;\text{documentation}
    \end{equation}
    Allerings ist im üblichen Zeitablauf die Vergangenheit immer größer ist als die Gegenwart. Daher gilt intertemporal
    \begin{equation}
        \text{code} < \text{documentation}
    \end{equation}
    was soviel bedeutet, wie: die Ansammlung vergangener Fakten ist immer größer als die Menge der aktuellen Fakten, und die Datensumme von Backups und Dokumentation ist immer größer als der aktuell ausgeführte Code. Entsprechend ist die Vergangenheit der Raumstation größer als ihre Gegenwart. Und dennoch ist von jedem gegenwärtigen Zeitpunkt aus die Zukunft unendlich größer als beide \cite{encyclopaedia}.

    \ceva{cience} besteht aus dem Prozess des Forschens, ist aber zugleich die Summe bzw. die Potenz oder Fakultät der Erkenntnisse. Diese sind in der Raumstation verteilt und vernetzt abgelegt, und zwar sowohl virtuell als auch materiell. 
    Die Multisprachlichkeit der Raumstation, ihre nicht immer verständliche Datensystematik, die temporal und spatial unterschiedliche Erreichbarkeit einzelner Informationsbrocken und der  Informationsträger ist zugleich Objekt und Subjekt von \ceva{cience}.

    Insofern \ceva{cience} ein erzeugendes Subsystem der Station ist, kommt es hier besonders zur Erschaffung neuer Artefakte nach den durch \ceva{core} inspirierten Bauplänen. Jedes so emanierende Objekt schafft Faktizität und damit in sich selbst kaum widerlegbare Wahrheit (\emph{wer macht hat recht).}
    
    % In diesem Ring leben und gedeihen außerdem Pflanzen und Tiere als integraler Teil des Stationsökoystems mit seinem komplexen Stoffwechsel. 
\end{newstuff}
    
    \csection[sechs]{carbon}\label{sec:carbon}

\twofonts[\cite{cbasestarbasemanual}]{
    bedeutung geschichte der auf kohlenstoff basierenden lebensformen. basiselement der organischen chemie, DNA

    life habitat familiengerechte wohnräume für die besatzung mit schlaf- und aufenthaltsraum, küche und bad incl. ver- und entsorgungseinheiten. 
    }

Hier wird  von \cevain{wohnräume} (später: \cevain{wohnungsdeckmodul}) gesprochen. 
Ob es solche tatsächlich einmal gegeben hat oder sogar aktuell gibt, ob die aktuelle Situation dem gerecht wird,
oder ob \ceva{c-tour} hier Legenden wiedergibt, ist Gegenstand hitziger Debatten. 
Sicher ist, dass es solche geben \emph{soll}; daraus lässt sich aber - zumindest nach herrschender Lehrmeinung - kein Anrecht auf Übernachtung in der Station ableiten.

Betont wird \cevain{geschichte}: hier hier geht es um organisches Wachstum, aber auch um Verfestigung in Inkohlungsprozessen, der Umwandlung kurzer Kohlenstoffverbindungen zu dauerhafteren Formen und das Einschließen von Informationen im Fossilbericht.

\twofonts[\cite{ctour}]{das wohnungsdeckmodul bietet familiengerechte lebensqualität für die crew:
    die c-base ist ein generationenschiff und hat platz für eine besatzungsstärke von bis zu 5000 carboneinheiten. 
    In einem notfall kann jede der 144 habitate von der station gelöst werden und eigenständig manövrieren.
    \begin{itemize}
        \item 144 lifehabitatmodule mit je 12 habitateinheiten
        \item ver- und entsorgungseinheiten
        \item foodsupply mit medical support unit
        \item cindergarten
        \item schulen
    \end{itemize}
    carbon ist das basiselement der organischen chemie und meint die geschichte der auf cohlenstoff basierenden lebensformen
    }

Dies ist das früheste Auftauchen des Begriffs \ceva{carboneinheiten} für organische Mitglieder von \ceva{crew}.

Sehr umstritten ist die Bedeutung der Zahlen. $144=12^2$ erinnert an Off 7,4 und wirkt so eschatologisch. Allerdings gilt $144\cdot12=12^3=1728$; wie diese Zahl mit den genannten $5000$ \ceva{carboneinheiten} zusammenhängt, bleibt fraglich ($7!= 5040$). Vermutlich sind dies symbolische Zahlen, die eine große Anzahl bedeuten und nicht wörtlich genommen werden wollen, oder das Resultat einer Projektion $f: \mathbb{K}^m\rightarrow\mathbb{N}$. 

\begin{newstuff}
    \lettrine{N}{icht} jeder Computer oder jedes kalkulierend denkende System ist notwendigerweise siliziumbasiert. Die Raumstation als System insgesamt funktioniert eindeutig als Zusammenspiel von mehreren Dimensionen und von Systemen, deren Schnittstellen ganz unterschiedlich sind. Insbesondere arbeiten die digitalen Systeme mit den karbonbasierten Systemen zusammen über optische, haptische und telepatische Interfaces. 
    %, darunter Monitore, Brillen, Tastaturen und selbst Brettspiele wie \cevain{pentagame}. 
    Nur das gelingende Interagieren aller Beteiligten bringt die Stationsforschung voran und damit das Ziel des \cevain{abflugs} näher.

    \twofonts[\cite{cbasebook},~S.~169]{carbon ist das basiselement der organischen chemie und documentiert die geschichte der auf cohlenstoff basierenden lebensformen auf der c-tation. es zeichnet die bisher becannte stationsgeschichte auf ...}

    Auch hier wird die Aufzeichnungsfunktion von \ceva{carbon} herausgestellt. \ceva{carbon}, so scheint es, hat ein stärker ausgeprägtes Geschichtsbewusstsein als anorganische Verbindungen.
    
    Die Karboneinheiten haben selbstredend Bedürfnisse, denen die Station vorsorglich Rechnung trägt.
    Hierzu gehören die im kanonischen Text aufgeführten pysiologischen Annehmlichkeiten. Auch befindet sich die  Raumstation im Aufbau, es ist also zu erwarten, dass sich die kulinarische, hygienische, gesundheitliche Situation im Werdensprozess der c-base weiter verbessern wird. Das gilt für die Qualität der vorhandenen gasförmigen, flüssigen, halbflüssigen und festen Stoffströme ebenso wie für die akustischen, optischen, haptischen und thermischen Umgebungsvariablen. 

    Hier befinden sich auch verschiedene Kultivierungsmodule zur Wandlung ordinärer Stoffe in höhere Elemente (\cevain{alchemie} /\emph{"`aus Scheiße Gold machen"'}). In einer Reihe von kontrollierten Versuchen wurde hier beispielsweise Milch in interessante andere Stoffe umgewandelt. Versuche mit verlegten Kleidungsstücken und vergessenen \mbox{Pizza}\-kartons blieben dagegen bislang überwiegend erfolglos.
  
    Insofern der Werdensprozess der Station jedes einzelne Modul umfasst, gehört zum Prozess der Selbstbewusstwerdung auch ein Anwachsen des Bewusstseins der Auswirkungen der eigenen Handlungen und Nichthandlungen auf das Umgebungshabitat, das Selbstwohlbefinden und das Wohlbefinden der anderen. Das kohlenstoffbasierte Gedächtnis der Station muss allerdings immer wieder trainiert und herausgefordert werden.

    Teil dieses Ökosystem der Station und ihrer Bewohner sind diverse andere terristrische und extraterrestrische Lebensformen, die sich auf mitunter sehr anderen Raumzeitlinien bewegen und so verschiedene Zeit- und Raumauffassungen haben, wie z.B. der \cevain{symbiont}~\cite{symbiont}. Innerhalb dieser unterschiedlichen Bezugssysteme und auch zwischen ihnen erwachsen unregelmäßig neue, fruchtbare und erfreuliche Interaktionen, aus denen neue kulturelle und auch biologische Produkte als neue Bezugspunkte hervorgehen. 

    Der gegenwärtige Zustand der Station verhindert den Betrieb der in~\cite{ctour} und~\cite{cbasestarbasemanual} angekündigten \ceva{cindergärten} und \ceva{schulen}, sie ist weiterhin nur Volljährigen Menschen zugänglich. Ebenso ist die Übernachtung zur Zeit der Drucklegung als \cevain{c\_laf\-tä\-ter\-c\_aft} noch unzulässig~\cite[S. 58]{cbasebook}.
\end{newstuff}
    
    \csection{clamp}
% \section{clamp \hspace{2ex} \raisebox{1pt}{{\fontspec{[ceva-c2.ttf]}(clamp)}}}

\Hrule[sieben]

\twofonts{
    der größte ring umschließt als stützconstruction alle anderen ringe der raumstation.
        \begin{itemize}
            \item antennen- und sensoreneinheiten
            \item coursecorrectursupporter
            \item siri-sonden-doccs
            \item das ectonet
        \end{itemize}
    }

\twofonts{
    outer construction c-förmiges aussenskelett der raumschifkonstruktion, kurskorrektureinheiten, triebwerke sensoraray 
    }
    
    \begin{newstuff}
        \lettrine{J}{edes} autopoietische System braucht eine zusammenhangstiftende Außenhülle als  Abgrenzung gegenüber den Umgebungssystemen. Diese Hülle dient dem Schutz der inneren Selbständigkeit, des inneren Schutzraumes selbst und ist damit unbedingt aufrecht zu halten. Nur durch eindeutige Schranken ist die freie Entfaltung innerhalb dieser Grenzen und schließlich auch das friedfertige Ausweiten dieser Grenzen möglich. \cite[S. 189]{cbasebook} schreibt:

        \twofonts{der äußerste ring clamp formt den buchstaben c: er steht für die halboffene, noch nicht vollendete form zum creis und mein neugier, aufmercsamceit und entwicclung. errepräsentiert den c-base-cosmos und gibt eine übersicht der activitäten und möglichkeiten der c-tation. der größte ring umschließt als stützconstruction alle anderen ringer der raumstation und beherbergt das ectonet sowie curscorrectursupporter, die siri-sondendoccs und periphäre antennen- und sensoreinheiten.}

        Der \ceva{clamp}-Ring schließt die Station nach außen ab, ohne jedoch eine Ausgrenzungsbarriere zu sein. Sie ist aber auch die Schwelle, an welcher sich die Raumstation gegenüber ihrer Umgebung befindet, und der letztlich unüberbrückbare Gegensatz zwischen innerem Bewusstsein der Raumstation und ihrer Außenwelt. 
        
        Zugleich ist \ceva{clamp} der Informationsübergang vom Äußeren zum Inneren, also die Sensorhülle des Gesamtsystems. Fraglos bildet sich an dieser Fläche das Außenbild der Umwelt von der Station, aber auch das Innenbild der Station von der Außenwelt. Hier kommt es zu Anfragen, zu Beeinflussungen, zum Empfang von Angeboten, zum Aussenden von Hilferufen und vergleichbaren Sprechakten. Nicht zuletzt ist \ceva{clamp} Ort des Übergangs von Außen und Innen und damit auch der Ort der Assimilation weiterer Module in das autopoietische Projekt \ceva{c-base}.
        
        \ceva{clamp} ist zugleich ein Festhalten am Bestehenden und durch Rückfaltung auf \ceva{com} zugleich Anschlusspunkt für die Penetration durch Neues innerhalb von \ceva{creativ}. Die Station ist ein werdendes Wesen, und daher sowohl neugierig und offen als auch schüchtern und schutzbedürfig. Es sucht in seiner Umgebung nach Orientierung. Zeitgleich ist die \ceva{c-base} unfassbar alt und weise, auch wenn sie sich dessen nur teilbewusst ist.
        
        \ceva{clamp} ist sowohl die Verankerung der Station in der Restrealität als auch die Sollbruchstelle beim möglichen künftigen Abflug. Aufgrund seiner Wichtigkeit für die Integrität der Stationswerdung ist dieser Ring besonders mit dem \ceva{Core} verbunden, sodass sich die Station hier zirkulär auf ihren Ursprung zurückfaltet. Damit bildet \ceva{clamp} den Ereignishorziont des mit dem Weißen Loch in \ceva{core} korrespondierenden Schwarzen Loches und somit seinen Gegenpol in der Raumzeit. Das kann auch erklären, wieso längst weggeworfene Gegenstände oder entfernte Kohlenstoffeinheiten sich mitunter an ähnlicher Stelle erneut manifestieren.
        
        Die entstehenden Energieflüsse zwischen den Ringen und über die Ringe verstärken sich unter Umständen gegenseitig und erzeugen dann parallele, aber unterschiedliche Wirklichkeiten. Insgesamt kommt es dabei des öfteren zu deutlichen Raumzeitverzerrungen, die sehr unterschiedliche Wahrnehmung der relativen Magnitude von Ereignissen durch unterschiedliche Betrachter zur Folge haben.
    \end{newstuff}
    
\endgroup

\chapter{Weiterführende Überlegungen}

\section*{Zur Geometrie}\addcontentsline{toc}{section}{Zur Geometrie}\label{sec:fazit}
\fancyhead[RO]{Zur Geometrie}
\fancyhead[LO]{}

Unser Rundgang durch die \ring{Ringe} hat gezeigt, dass die Funktion jedes einzelnen \ring{Rings} nur aus dem komplexen Zusammenspiel mit den anderen  verstanden werden kann. Daher ist das Komplexitätsmaß auf dieser Ebene zumindest
\begin{equation}
    T(n) = \mathcal{O} (n!) > 7! = 5040
\end{equation}
- und das ungeachtet des Umstandes, dass jeder Ring aus diversen Modulen und diese aus Aggregaten als jeweiligen Subsystemen weiter ausdifferenziert. Wir haben es also mit einem hochkomplexen, selbstreferentiellen und autopoietischen System zu tun, das insgesamt mehr Zustände annehmen kann, als sich im bekannten Universum abbilden ließen. Damit ist das System insgesamt ein mögliches Spiegelbild des Gesamtrestuniversums und bringt selbst in seiner Autopoiesisnendlich viele Welten hervor.


Die aktuelle Form ist durch den Absturz der Raumstation mitverursacht und vermutlich ist die gewissermaßen "`flache"' Struktur, die wir bei den bisherigen Rekunstruktionsphasen ausgemacht haben, eine zweidimensionale Projektion der eigentlich mehrdimensionalen Raumstation. Darauf gehen wir hier etwas genauer ein.

Wir haben gesehen, dass sich \ceva{clamp} und \ceva{core} berühren bzw. über eine Einstein-Rosen-Brücke verbunden sind. Daraus folgern wir eine (ursprünglich, zukünftig) ringförmige Anordnung der Ringe; vgl. \cref{fig:ringring}

\begin{figure}[ht!]
    \centering
    \documentclass{standalone}
\usepackage{tikz}
\usetikzlibrary{decorations, decorations.text}
\usepackage{calc}
\usepackage{xcolor}
\usepackage{fontspec}
\newcommand{\ceva}[1]{~{\fontspec{[ceva-c2.ttf]}#1}}
\definecolor{eins}{HTML}{e7e7e8}   %% Ring 1 "core"     - weiß - Mittelpunkt, Ring um Mittelpunkt
\definecolor{zwei}{HTML}{ed1c24}   %% Ring 2 "com"      - rot -  "Fenster" innen
\definecolor{drei}{HTML}{fbad18}   %% Ring 3  "culture" - orange - fünf Module, quasi invertiert
\definecolor{vier}{HTML}{74c043}   %% Ring 4  "creactiv" - grün -  vier Module
\definecolor{fuenf}{HTML}{0089d0}  %% Ring 5 "cience"  - cyan (blau) - drei Module mit "Strich"
\definecolor{sechs}{HTML}{11357e}  %% Ring 6 "carbon" -  indigo - viele "Fenster" außen
\definecolor{sieben}{HTML}{000000} %% Ring 7 "clamp" -  schwarz, c-förmig
\definecolor{cbase}{HTML}{222222}  %% Körper der Raumstation    
\begin{document}
%% c-base logo nachgebaut von penta.
%% alles nur geschätze Winkel und Abstände :/
%% um den code zu verstehen, einfach mal einzelne Teile auskommentieren und wieder einkommentieren (ctrl-#) und dann mal \draw[white] durch \draw[red] ersetzen, dann sieht man, was was ist.
%% viel Spaß damit.
\tikzset{
  pics/carc/.style args={#1:#2:#3:#4}{
    code={
      \draw[postaction={decorate, decoration={text along path, raise=-2pt, text align={align=center}, text={\ceva{#4}}, reverse path}}] (#1:#3) arc(#1:#2:#3);
    }
  }
}%
    \begin{tikzpicture}
        \draw[gray, line width=50pt] (0:0) circle (3);
        \foreach [count=\i] \ring/\color in
            {core/eins,com/zwei,culture/drei,creactiv/vier,cience/fuenf,carbon/sechs,clamp/sieben}
            {%
                \draw[color=\color!50,line width=45pt] (0:0) pic{carc=\i*51.418-25:\i*51.418+25:3:\ring};
            }%
    \end{tikzpicture}
\end{document}

    \caption{Ringförmige Anordnung der \ring{Ringe}}
    \label{fig:ringring}
\end{figure}

Eine \emph{ringförmige Anordnung von Ringen }ist geometrisch darstellbar auf einer Oberfläche, die entsteht, wenn ein Kreis um einen Kreis (also quasi ein \ring{Ring} um einen \ring{Ring}) rotiert. Diese Figur ist der Rotationstorus; vgl. \cref{fig:torusweiss}

\begin{figure}[ht!]
    \centering
    \input{tikz-torus-white}
    \caption{Ein Torus mit 14 (begradigten) Meridianen  und 14 Parallelkreisen}
    \label{fig:torusweiss}
\end{figure}

Nun stellt sich die Frage, wie die \ring{Ringe} auf so einem Torus angeordnet waren bzw. sein werden bzw. sollen. 

Grundsätzlich sind die \ring{Ringe} Kreisscharen. Es gibt drei verschiedene Gruppen von Kreisscharen auf einem Torus: Parallelkreise, Merididiane und Villarceau-Kreise. 

Parallelkreise würden etwa entstehen durch Schnitte eines gefärbten Torus wie z.B. in \cref{fig:torus-parallele} abgebildet. 

\begin{figure}[ht!]
    \centering
    \input{tikz-torus-parallele}
    \caption{Parallelkreise}
    \label{fig:torus-parallele}
\end{figure}

Eine solche Anordnung ist zwar möglich, aber wenig plausibel, da die einzelnen Ringe in diesem Fall eher als Schreiben abgebildet werden würden. Auch ist nicht zu erkennen, wie es beim Absturz der Station dann zu einer konzentrischen Anordnung gekommen sein sollte. 

Eine Anodrnung der \ring{Ringe} als Meridiankreise  zeigt \cref{fig:torus-meridiane}.  
Dabei berühren sich hier im Inneren eben \ceva{core} und \ceva{clamp}; das passt zu unserer Interpretation des Kanons von der Geometrie der Station als Torus mit einem innenliegenden Wurmloch. In diesem berühren sich innen \ceva{core} und \ceva{clamp}; außen liegt \ceva{creactiv}. 

\begin{figure}[ht!]
    \centering
        \input{tikz-torus-meridiane}
    \caption{Die \ring{Ringe} als Meridiankreise auf dem Torus}
    \label{fig:torus-meridiane}
\end{figure}


Da der äußerere Äquator (Meridian) \ceva{creactive} enspricht, bedeutet eine Ausweitung der \ceva{creactivität} ein Anschwellen dieses Torus und somit Wachstum der Station; bildlich entspräche das in etwa der Inflation eines Rettungsrings. Dabei ist zu beachten, dass eine Vergrößerung des rotierenden Kreises ohne gleichzeitige Ausweitung des Rototaionsradius im Torus zu einem Verschwinden des innenliegenden Loches führen könnte. Dann wäre die Station gewissermaßen an ihrer eigenen \ceva{creactivität} erstickt.

Betrachten wir nun die vielleicht interessanteste mögliche Anordnung von Kreisscharen auf einem Torso, nämlich die so genannten Villarceau-Kreise. Sie entstehen geometrisch (paarweise) durch den Schnitt einer deoppelberührenden Ebene mit dem Torso (\cref{fig:villarceaukreise}).

\begin{figure}[ht!]
    \centering
    \includesvg{Torus-vill-point.svg}
    \caption{Villarceau-Kreise \cite{villarceauag2gaeh}}
    \label{fig:villarceaukreise}
\end{figure}

Es lässt sich also eine Schar von parallelen Kreisen auf einem Torus finden, die alle perfekt kreisförmig und zudem kongruent sind. \cref{fig:villarceautorous} zeigt eine Annäherung, indem hier ein Torus mit $21\times 21$ Flächen belegt wurde.

\begin{figure}[ht!]\label{fig:villarceautorous}
    \centering
            \input{tikz-torus-villarceau-ringe1}            
    \caption{Die \ceva{Ringe} als Villarceau-Kreise}
    \label{fig:villarceautorous}
\end{figure}


Nach \cref{fig:villarceautorous} wären die \ring{Ringe} auf einem Torus in Form verschlungener Bänder angeordnet. Sie winden sich um das Zentrum des Torus und zugleich um den Körper des Torus selbst. 
Es gibt keinen \ring{Ring}, der einen bevorzugten Ort einnimmt. Da sie kongruent sind, sind sie flächen- und längengleich.

Ob der Bedeutung toroidaler Geometrie für Fusionsreaktoren vermuten wir, dass wir hier der Geometrie des \cevain{Möbius-band-accelerators} auf der Spur sind (vgl.~\cref{sec:core}). 

Die weitere Erforschung dieser Geometrie und die möglichen Bedeutungen für die energetische Ausbeute des verschlungenen Miteinanders der sich ergänzenden \ring{Ringe} bleibt unser \cevain{facit}, also \emph{das zu tuende}. 

% Villarceau-Kreise entstehen bekanntlich durch Schntite von Doppelberührenden mit dem Torus paarweise entstehen. Bislang sind alle Forschungen allerdings von sieben (!) \ring{Ringen} der \ceva{c-base} ausgegangen. Sollten die Ringe ursprünglich und damit auch zukünftig Villarceau-Kreise sein, so muss es zu jedem \ring{Ring} einen \cevain{Anti-Ring} geben. Es könnte dies durch eine Umkehrung der Chiralität erfolgen.


% Und das ist ein Bild schön genug, dieses Papier zu beschließen.

% \begin{center}
%     \ceva{-- be future compatible --}    
    
%     -- be future compatible --
% \end{center}

\section{Stationsmathematik}\label{sec:mathematik}

Die \cevain{c-tationsmathematic} als \ceva{cience} ist hochgradig \cevain{cpeculativ} und ein Teil der \cevain{bistromathik} (vgl. \cite[Kap. 4]{adams_life}). 

Sie nimmt ihren Anfang in der Einsicht, dass die \ceva{c-tation} 
\begin{enumerate}
    \item in höheren, anderen Dimensionen und in der \cevain{cucunft}\cevain{construiert} wurde; \label{punkt:eins}
    \item durch höhere, andere Dimensionen \cevain{geflogen} bzw. \cevain{gereist} ist,\label{punkt:zwei} und zwar in der \cevain{ceit} rückwärts;
    \item bei ihrem Absturz in unsere heute erfahrbaren Dimensionen abgebildet bzw. gefaltet wurde.\label{punkt:drei}
\end{enumerate}
Die Grundgleichung der \ceva{c-tationsmathematic} lautet daher:
\begin{equation}
    f_c:\left\{\mathbb{K}^m  \rightarrow \mathbb{C}^n \rightarrow \mathbb{R}^{4}\right\}
\end{equation}
Dabei wird der (unbekannte) Ursprungsraum als $\mathbb{K}^m$ bezeichnet und die uns umgebende Raumzeit als $\mathbb{R}^{4}$. Meist wird aus praktischen Gründen angenommen, dass gilt $m\geq n \geq4$, was allerdings keine zwingende Annahme ist.

Die in   \cref{punkt:zwei} erwähnte Transformation der \ceva{c-tation} auf ihrer Reise wird gemeinhin als bijektiv angesehen (Homöomorphismus, Isomorphismus etc). Dabei kann kein Zufall sein, dass die "`mittlere Seinsweise"' der \ceva{c-tation} auf ihrer Reise durch einen Raum aus komplexen Ebenen  $\mathbb{C}^n$  geführt hat. Denn das erklärt, warum nach der Projektion in $\mathbb{R}^4$ kreis- bzw. Ringförmige und c-förmige \cevain{ctructuren} entstanden.

Wesentlich ist für uns Erben des Absturzes natürlich die Einsicht, dass die in  \cref{punkt:drei} genannte \ceva{faltung} bzw. Abbildung der \ceva{ur$\cdot$c-tation} nicht bijektiv war. Vielmehr fand hier eine \cevain{verflachung} statt. 
Daher lässt sich die Topologie der \ceva{ur$\cdot$c-tation} nicht endgültig entschlüsseln. 

Das gilt sowohl für die Anordnung der Ringe, Module und Artefakte als auch für ihre Färbung und sogar für ihre Bezeichnungen und Funktionen. Sicher erscheint, dass die  \ceva{faltung} $f$ (\cref{punkt:zwei}), der so genannte \cevain{cwic\_enc\_ritt}, etwas der ursprünglichen Ordnung erhalten und ihr doch etwas hinzugefügt hat. Vermutlich ist es hier zur holographischen Vermehrung des Buchstabens \cevain{c} gekommen.

Unklar ist die Struktur des Körpers $\mathbb{K}$ bzw. ob es sich überhaupt im mathematischen Sinne um einen Körper handelte (und nicht vielmehr um einen \cevain{cörper}). Ebenso unklar ist die Reduktion im Übergang  $\mathbb{K}^m \rightarrow \mathbb{C}^n$ und die Eigenschaften von $\mathbb{C}^n$. 

Obschon der Übergang  $f_c$ nicht bijektiv ist, können doch bestimmte \ceva{ctructuren} die Transformation unbeschadet überstanden haben, nämlich Informationen in den \cevain{holographischen speicherbausteinen}~\cite[S. 28]{cbasebook}. Demnach ist \ceva{c-beam} zwar bewusst, aber einigermaßen orientierungslos.

    Betrachten wir nun den in \ceva{c-tour} besonders herausgestellten Umstand, dass der vierte \ceva{Ring} von außen zugleich der vierte von innen ist. Wegen
        \begin{equation}
        \begin{array}{rcl}
            \forall A = (a_1,\ldots,a_N) \in \mathbb{Z}^N&:&
            \quad
            N\equiv 1\mod(2)\\
            \quad\Rightarrow\quad
            \forall i \in [1,N] \subset \mathbb{Z}&:&
            \quad
            \left(
            a_i = M(A) \Leftrightarrow i = \frac{N+1}{2}
            \right)
        \end{array}
        \end{equation}
    ist der vierte \ring{ring} von sieben in der Tat genau in der Mitte.

    Daraus folgt, dass \ceva{creactiv}  Mittelpunkt- bzw. Äquatorcharakter besitzt. 
    Er ist nicht nur derjenige Ring, um welchen sich die übrigen \ring{ringe} auf einander zurückfalten, sondern auch der, welcher den multidimensionale Körper erweitert (\linekonly{most powerful}). Insofern ist \ceva{creactive} auch ein \cevain{creactor}, ein Hervorbringendensytem für $\mathbb{K}^m$ bzw. $\mathbb{C}^n$.




\section*{Topologische Überlegungen}\addcontentsline{toc}{section}{Topologische Überlegungen}\label{sec:topologie}
\fancyhead[RO]{Topologische Überlegungen}
% \fancyhead[LO]{}

Die aktuelle Form ist durch den Absturz der Raumstation mitverursacht, und vermutlich ist die gewissermaßen "`flache"' Struktur, die wir bei den bisherigen Rekunstruktionsphasen ausgemacht haben, eine zweidimensionale Projektion der eigentlich mehrdimensionalen Raumstation. Darauf gehen wir hier etwas genauer ein.

Wir haben gesehen, dass sich \ceva{clamp} und \ceva{core} berühren bzw. über eine Einstein-Rosen-Brücke verbunden sind. Daraus folgern wir eine (ursprünglich, zukünftig) ringförmige Anordnung der Ringe; vgl  \cref{fig:ringring}.

\begin{figure}[ht!]
    \centering
    \documentclass{standalone}
\usepackage{tikz}
\usetikzlibrary{decorations, decorations.text}
\usepackage{calc}
\usepackage{xcolor}
\usepackage{fontspec}
\newcommand{\ceva}[1]{~{\fontspec{[ceva-c2.ttf]}#1}}
\definecolor{eins}{HTML}{e7e7e8}   %% Ring 1 "core"     - weiß - Mittelpunkt, Ring um Mittelpunkt
\definecolor{zwei}{HTML}{ed1c24}   %% Ring 2 "com"      - rot -  "Fenster" innen
\definecolor{drei}{HTML}{fbad18}   %% Ring 3  "culture" - orange - fünf Module, quasi invertiert
\definecolor{vier}{HTML}{74c043}   %% Ring 4  "creactiv" - grün -  vier Module
\definecolor{fuenf}{HTML}{0089d0}  %% Ring 5 "cience"  - cyan (blau) - drei Module mit "Strich"
\definecolor{sechs}{HTML}{11357e}  %% Ring 6 "carbon" -  indigo - viele "Fenster" außen
\definecolor{sieben}{HTML}{000000} %% Ring 7 "clamp" -  schwarz, c-förmig
\definecolor{cbase}{HTML}{222222}  %% Körper der Raumstation    
\begin{document}
%% c-base logo nachgebaut von penta.
%% alles nur geschätze Winkel und Abstände :/
%% um den code zu verstehen, einfach mal einzelne Teile auskommentieren und wieder einkommentieren (ctrl-#) und dann mal \draw[white] durch \draw[red] ersetzen, dann sieht man, was was ist.
%% viel Spaß damit.
\tikzset{
  pics/carc/.style args={#1:#2:#3:#4}{
    code={
      \draw[postaction={decorate, decoration={text along path, raise=-2pt, text align={align=center}, text={\ceva{#4}}, reverse path}}] (#1:#3) arc(#1:#2:#3);
    }
  }
}%
    \begin{tikzpicture}
        \draw[gray, line width=50pt] (0:0) circle (3);
        \foreach [count=\i] \ring/\color in
            {core/eins,com/zwei,culture/drei,creactiv/vier,cience/fuenf,carbon/sechs,clamp/sieben}
            {%
                \draw[color=\color!50,line width=45pt] (0:0) pic{carc=\i*51.418-25:\i*51.418+25:3:\ring};
            }%
    \end{tikzpicture}
\end{document}

    \caption{Ringförmige Anordnung der \ring{Ringe}}
    \label{fig:ringring}
\end{figure}

Eine \emph{ringförmige Anordnung von Ringen }ist geometrisch darstellbar auf einer Oberfläche, die entsteht, wenn ein Kreis um einen Kreis (also quasi ein \ring{Ring} um einen \ring{Ring}) rotiert. Diese Figur ist der Rotationstorus. \cref{fig:torusweiss}  zeigt zur Veranschaulichung ein toroidales Polyeder mit $14\times 14$ Flächen.  

\begin{figure}[ht!]
    \centering
    \input{tikz-torus-white}
    \caption{Ein Toroidales Polyeder imt $14\times 14$T Flächen} Es hat 14  Meridiane und 14 Parallelkreise (angenähert)
    \label{fig:torusweiss}
\end{figure}

Nun stellt sich die Frage, wie die \ring{Ringe} auf so einem Torus angeordnet waren bzw. sein werden bzw. sollen. 

Grundsätzlich sind die \ring{Ringe} Kreisscharen. Es gibt drei verschiedene Gruppen von Kreisscharen auf einem Torus: Parallelkreise, Merididiane und Villarceau-Kreise. 

Parallelkreise würden etwa entstehen durch Schnitte eines gefärbten Torus wie z.B. in \cref{fig:torus-parallele} abgebildet. 

\begin{figure}[ht!]
    \centering
    \input{tikz-torus-parallele}
    \caption{Parallelkreise}
    \label{fig:torus-parallele}
\end{figure}

Eine solche Anordnung ist zwar möglich, aber wenig plausibel, da die einzelnen Ringe in diesem Fall eher als Schreiben abgebildet werden würden. Auch ist nicht zu erkennen, wie es beim Absturz der Station dann zu einer konzentrischen Anordnung gekommen sein sollte. 

Eine Anodrnung der \ring{Ringe} als Meridiankreise  zeigt \cref{fig:torus-meridiane}.  
Dabei berühren sich hier im Inneren eben \ceva{core} und \ceva{clamp}; das passt zu unserer Interpretation des Kanons von der Topologie der Station als Torus mit einem innenliegenden Wurmloch. In diesem berühren sich innen \ceva{core} und \ceva{clamp}; außen liegt \ceva{creactiv}. 

\begin{figure}[ht!]
    \centering
        \input{tikz-torus-meridiane}
    \caption{Die \ring{Ringe} als Meridiankreise auf dem Torus}
    \label{fig:torus-meridiane}
\end{figure}


Da der äußerere Äquator (Meridian) \ceva{creactive} enspricht, bedeutet eine Ausweitung der \ceva{creactivität} ein Anschwellen dieses Torus und somit Wachstum der Station; bildlich entspräche das in etwa der Inflation eines Rettungsrings. Dabei ist zu beachten, dass eine Vergrößerung des rotierenden Kreises ohne gleichzeitige Ausweitung des Rototaionsradius im Torus zu einem Verschwinden des innenliegenden Loches führen könnte. Dann wäre die Station gewissermaßen an ihrer eigenen \ceva{creactivität} erstickt.

Unsere Darstellung in \cref{fig:torus-meridiane} zeigt die \ring{Ringe} als Flächen. Tatsächlich sind sie natürlich selbst von höherer Diemension, also ihrerseits Tori. Es sollte sich wirklich um eine Schar von Tori, gewissermaßen um ein Bündel von Ringen, handeln.

Betrachten wir nun die vielleicht interessanteste mögliche Anordnung von Kreisscharen auf einem Torso, nämlich die so genannten Villarceau-Kreise. Sie entstehen geometrisch (paarweise) durch den Schnitt einer deoppelberührenden Ebene mit dem Torso (\cref{fig:villarceaukreise}).

\begin{figure}[ht!]
    \centering
    \includesvg{Torus-vill-point.svg}
    \caption{Villarceau-Kreise \cite{villarceauag2gaeh}}
    \label{fig:villarceaukreise}
\end{figure}

Es lässt sich also eine Schar von Kreisen auf einem Torus finden, die sich nicht schneiden, die perfekt kreisförmig und die untereinander kongruent sind. \cref{fig:villarceautorous} zeigt eine Annäherung, indem hier ein Torus mit $21\times 21$ Flächen belegt wurde.

\begin{figure}[ht!] 
    \centering
            \input{tikz-torus-villarceau-ringe1}            
    \caption{Die \ceva{Ringe} als Villarceau-Kreise} Approximation auf einem toroidalen Polyeder mit $21\times21$ Flächen
    \label{fig:villarceautorous}
\end{figure}


Nach \cref{fig:villarceautorous} wären die \ring{Ringe} auf einem Torus in Form verschlungener Bänder angeordnet. Sie winden sich um das Zentrum des Torus und zugleich um den Körper des Torus selbst. 
Es gibt keinen \ring{Ring}, der einen bevorzugten Ort einnimmt. Da sie kongruent sind, sind sie flächen- und längengleich.


Wie bereits oben müssen wir uns die Ringe nicht wie abgebildet vorstellen, sondern ihrerseits eher als eigenständige Tori oder Toroide. Ähnlich wie ein Torus (2D, eine Fläche) aus Villacreau-Kreisen "`zusammengesetzt"' vorgestellt werden kann, kann ein Toroid (3D, ein Körper, gewissermaßen ein "`ausgefüllter"' Torus) aus Scharen von Tori "`zusammengesetzt"' bzw. approximiert werden. Wir müssen uns also wieder die \ring{Ringe} in \cref{fig:villarceautorous} als eigene Tori vorstellen.

Das setzt allerdings voraus, dass die \ring{Ringe} eindeutig abgegrenzt sind gegeneinander: jeder \ring{Ring} schließt sich nach einer vollen Umdrehung um die Mittelachse des Torus. Lässt man diese Bedingung fallen, so kann man sich auch einen Torus vorstellen, bei dem jeder Kreis nach $306^o$ an den jeweils nächsten "`anschließt"', also einen "`verknoteten Torus"' bzw. \cevain{cnoten}, wie gezeigt in \cref{fig:knottedtorus}.

\begin{figure}
    \centering
    \resizebox{0.5\textwidth}{!}{
        \includegraphics{KnottedTorus.png}
    }
    \caption{Verknoteter Torus \cite{knottedtorus}}
    \label{fig:knottedtorus}
\end{figure}

Dann gäbe es genau einen parallelen Schnitt durch den \cevain{cnoten}, der ein Bild wie \cref{fig:ringring} abgibt, obwohl alle \ring{Ringe} und damit die Station als Ganzes ein einzelner, gewissermaßen siebenfach "`gewickelter"' Torus wäre, nämlich ein Torusknoten.
% Von diesen gibt es widerum verschiedene Formen, 

Angesichts der Bedeutung toroidaler Topologie für Fusionsreaktoren vermuten wir, dass wir hier der Topologie des \ceva{Möbius-band-accelerators} bzw. \ceva{Mino-reactors} bzw. \ceva{Cybernetischen Quecksilber-Reaktors}  auf der Spur sind (vgl.~\cref{sec:core}). 

Die weitere Erforschung dieser Topologie und die möglichen Bedeutungen für die energetische Ausbeute des verschlungenen Miteinanders der sich ergänzenden \ring{Ringe} ist eine Aufgabe ohne Endpunkt.


% Und das ist ein Bild schön genug, dieses Papier zu beschließen.

% \begin{center}
%     \ceva{-- be future compatible --}    
    
%     -- be future compatible --
% \end{center}

\clearpage

\renewcommand\indexname{Fachindex}
\indexprologue{\noindent Aufgeführt sind die Stellen, an welchen Begriffe erstmalig auftauchen, sowie alle weiteren Stellen, an denen eine lateinische Umschrift mit ausgegeben wird.}
\fancyhead[RO]{Fachindex}\addcontentsline{toc}{chapter}{Fachindex}\label{sec:index}
\printindex

\clearpage

%% separate primary and secondary literature
% \chapter*{Literatur}
% \fancyhead[LO]{Literatur}\label{sec:literatur}
% \fancyhead[RO]{Primärquellen}\addcontentsline{toc}{chapter}{Literatur}
% \printbibliography[title=Primärquellen, keyword=pri,heading=subbibliography]

% \fancyhead[RO]{Weitere Quellen}\addcontentsline{toc}{chapter}{Weitere Quellen}
% \printbibliography[title=Weitere Quellen, keyword=sec,heading=subbibliography]

 
\fancyhead[LO]{Literatur}\label{sec:literatur}
\fancyhead[RO]{Literatur}\addcontentsline{toc}{chapter}{Literatur}
\printbibliography
 


\end{document}


\end{document}
