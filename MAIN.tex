\documentclass[14pt,ngerman]{extarticle}  %% document requires XeLaTex.
\usepackage[ngerman]{babel}
\usepackage{geometry}
 \geometry{
 a4paper,
 total={170mm,257mm},
 left=20mm,
 top=20mm,
 }
\usepackage{fancyhdr}
\usepackage{booktabs}
\usepackage{microtype}
\usepackage{amsmath}
\usepackage{amssymb}
\usepackage{mathtools}
\usepackage{nicefrac}
\usepackage{tabularx}

\DeclarePairedDelimiter\ceil{\lceil}{\rceil}
\DeclarePairedDelimiter\floor{\lfloor}{\rfloor}


\usepackage{standalone}
\usepackage{tikz}

\usepackage{pst-solides3d}

\usepackage[hidelinks]{hyperref}

\usepackage{cleveref}
% \AddToHook{cmd/section/before}{\clearpage}

\usepackage{fontspec}
% % \setmainfont{ceva-c2.ttf}
\setmainfont{Alegreya Sans}

\usepackage{lettrine}
 
\usepackage{biblatex}
\addbibresource{literature.bib}

\definecolor{eins}{HTML}{e7e7e8}   %% Ring 1 "core"     - weiß - Mittelpunkt, Ring um Mittelpunkt
\definecolor{zwei}{HTML}{ed1c24}   %% Ring 2 "com"      - rot -  "Fenster" innen
\definecolor{drei}{HTML}{fbad18}   %% Ring 3  "culture" - orange - fünf Module, quasi invertiert
\definecolor{vier}{HTML}{74c043}   %% Ring 4  "creactive" - grün -  vier Module
\definecolor{fuenf}{HTML}{0089d0}  %% Ring 5 "cience"  - cyan (blau) - drei Module mit "Strich"
\definecolor{sechs}{HTML}{11357e}  %% Ring 6 "carbon" -  indigo - viele "Fenster" außen
\definecolor{sieben}{HTML}{000000} %% Ring 7 "clamp" -  schwarz, c-förmig
\definecolor{cbase}{HTML}{222222}  %% Körper der Raumstation

\newcommand{\noun}[1]{\textsc{#1}}

\newcommand{\Hrule}[1][.]{%
\begingroup\color{#1}%
    \resizebox{\textwidth}{!}{
    \begin{tikzpicture}
        \filldraw[draw=black]  (0,0) rectangle (10,0.2);
    \end{tikzpicture}
    }%
\endgroup%
}

\newcommand{\Hrulek}[1][.]{%
\begingroup\color{#1}%
    \resizebox{3ex}{!}{
    \begin{tikzpicture}
        \filldraw[draw=black]  (0,0) rectangle (2,1);
    \end{tikzpicture}
    }%
\endgroup%
}

\newcommand{\ceva}[1]{~{\fontspec{[ceva-c2.ttf]}#1}}
\newcommand{\cevain}[1]{~{\fontspec{[ceva-c2.ttf]}#1} | \emph{#1}}

\newcommand{\twofonts}[1]{
\begin{quote}%\small
    {\fontspec{[ceva-c2.ttf]}#1}
    \par
    {#1}
\end{quote}
}

\newcommand{\csection}[1]{
\section{#1 \hspace{2ex} \raisebox{1pt}{{\fontspec{[ceva-c2.ttf]}(#1)}}}
\fancyhead[RO]{#1}
\fancyhead[LO]{\thesection}
}

\newenvironment{newstuff}
{}
{\hfill{\footnotesize{[penta]}}}


\title{
    {\fontspec{[ceva-c2.ttf]}die 7 ringe der c-base} \\
    Die 7 Ringe der c-base
    }
\author{
    \resizebox{4cm}{!}{\documentclass{standalone}
\usepackage{tikz}
\usepackage{calc}
\usepackage{xcolor}
\begin{document}
%% c-base logo nachgebaut von penta.
%% alles nur geschätze Winkel und Abstände :/
%% um den code zu verstehen, einfach mal einzelne Teile auskommentieren und wieder einkommentieren (ctrl-#) und dann mal \draw[white] durch \draw[red] ersetzen, dann sieht man, was was ist.
%% viel Spaß damit.
\tikzset{
  pics/carc/.style args={#1:#2:#3}{
    code={
      \draw[pic actions] (#1:#3) arc(#1:#2:#3);
    }
  }
}
\definecolor{eins}{HTML}{FFFFFF}   %% Ring 1 "core"     - weiß - Mittelpunkt 
\definecolor{zwei}{HTML}{FF0000}   %% Ring 2 "com"      - rot -  Ring um Mittelpunkt
\definecolor{drei}{HTML}{FF7F00}   %% Ring 3  "culture" - orange - "Fenster" innen
\definecolor{vier}{HTML}{AAFF00}   %% Ring 4  "creativ" - grün -  fünf Module, quasi invertiert
\definecolor{fuenf}{HTML}{00FFFF}  %% Ring 5 "cience"  - blau - vier Module
\definecolor{sechs}{HTML}{800080}  %% Ring 6 "carbon" -  violett - drei Module mit "Strich"
\definecolor{sieben}{HTML}{4B0082} %% Ring 7 "clamp" -  indigo - viele "Fenster" außen

    \begin{tikzpicture}%
        \def\radi{10}%     

        %% äußerer Radius
        \filldraw[black] (0:0) circle (\radi-0.3);
        % was wegnehmen / weiß übermalen
        \draw[white, line width=1.6cm] (0:0) pic{carc=-55:90:\radi-0.72}; 

        %% clamp - viele kleine Fenster
        \foreach [count=\i] \ii in {%
        70,74,78,100,
        114,118,...,170,
        182,186,
        202,206,...,250,
        266,270,...,304}
            \draw[sieben, line width=0.5cm] (0:0) pic{carc=\ii:\ii-2:\radi-2}; 
        
        \draw[white, line width=3.05cm] (0:0) pic{carc=50:-55:\radi-3};
        
        \draw[white, line width=2cm] (0:0) pic{carc=-90:0:\radi-3.5};

  
        \draw[sechs, line width=1.8cm] (0:0) 
            pic{carc=82:135:\radi-3.5};
        \draw[sechs, line width=1.8cm] (0:0) 
            pic{carc=65:74:\radi-3.5};
        \draw[sechs, line width=1.8cm] (0:0) 
            pic{carc=53:62:\radi-3.5};
            
        \draw[black, line width=0.4cm] (0:0) 
            pic{carc=110:135:\radi-3.3};

        \foreach [count=\i] \ii in {1,2,3,4}
            \draw[fuenf, line width=1.5cm] (0:0) 
            pic{carc=90*\i:90*\i+45:\radi-5.5};


        \filldraw[vier] (0:0) circle (\radi-6.5);
        
        \foreach [count=\i] \ii in {1,2,3,4,5}
            \draw[black, line width=0.7cm] (0:0) 
            pic{carc=72*\i+27:72*\i+36+27:\radi-7};

        \filldraw[black] (0:0) circle (\radi-7.5);

        \foreach [count=\i] \ii in {160,172,...,490}
            \draw[drei, line width=0.4cm] (0:0) pic{carc=\ii:\ii-8:\radi-8};   
        
        \filldraw[zwei] (0:0) circle (\radi-8.5);
        \filldraw[black] (0:0) circle (\radi-8.8);
        \filldraw[eins] (0:0) circle (\radi-9.8);
     
    \end{tikzpicture}
\end{document}
}\\
    penta}
% \date{May 2024}

\begin{document}


\maketitle

% \setlength{\parindent}{0pt}
\setlength{\parskip}{1ex}
% \setlength\itemsep{0em}

% \begin{quote}\small
%     \twofonts{es gibt drei möglichkeiten in der c-base, um an informationen zu gelangen: würfeln, bogglen, googlen.} \hfill Anonymous, nach \cite[S. 47]{cbasebook}
% \end{quote}

\begin{abstract}
    Dieses Papier dokumentiert die aktuell zuhandenen Informationen über die Rolle und Funktion der Ringe der Raumstation unter Berlin \cevain{c-base}. Jeder Abschnitt beginnt mit einer Wiedergabe der Primärquellen  \cevain{c-tour}~\cite{ctour} und des \cevain{starbase-manual7}~\cite{cbasestarbasemanual}.
    Das offensichtlich später, aber gedruckt erschienene \cevain{c-book}~\cite{cbasebook} benutzt diese Quellen und entspricht ihnen im Großen und Ganzen; wo es weitere Informationen bereit hält, sind diese ebenfalls wörtlich, aber in den Text eingebettet, angeführt. 
    
    Unser Haupttext ist ein Kommentar aus Sicht des aktuellen Wissenstands im Lichte neuer Ausgrabungen und der vorangegangenen Exegese. 
    
    Die Raumstation ist ein  "`autopoietisches"', also selbsterschaffendes, System; sie (re)kon\-stru\-iert sich selbst. Dieser Prozess ist unabgeschlossen~\cite{cbasebook} und vermutlich sogar ergebnisoffen; alles, was wir bereits über die Raumstation wissen, ist temporal gebunden und kann durch neue Erkenntnisse und weitere Entwicklungen überholt werden. 
    
    Da folglich niemand die \ceva{c-base} vollständig kennt oder verstehen kann, ist jede Aussage nur ein Beitrag zum Erkenntnisfortschritt. 
    Insofern bitten wir diesen Text nicht für apodiktisch zu halten. Insbesondere sind die hier abgedruckten Aussagen keine offizielle Mitteilung des c-base e.V. Berlin. 

    Diese Arbeit hofft einen Beitrag zum besseren Verständnis von Funktion und Architektur der multidimensionalen Station zu leisten.
    
    Vielen Dank für die Beachtung aller Sicherheitshinweise.
\end{abstract}

\clearpage

\tableofcontents

\clearpage

\section*{Methodik der Qullenwidergabe}\addcontentsline{toc}{section}{Methodik der Qullenwidergabe}

Die \ceva{c-base} verwendet eine eigene Schreibweise namens  \cevain{clang}; dabei wird auf Kapitalisierung verzichtet,  Sibilanten und der stimmlose velare Plosiv (\emph{k}) werden mehr oder weniger ausschließlich durch den Buchstaben \cevain{c} ersetzt; Ähnliches gilt für daraus gebildete  Konsonantencluster.\footnote{Eine genaue Übersicht über die Ersetzungsregeln bietet \cite[S.~46]{cbasebook}.} In unseren eigenen Texten verzichten wir darauf.

Die \ceva{c-base} kennt eine eigene Schrift namens \cevain{ceva}. Wir nutzen sie, um Primärquellen und Stationsfachbegriffe als solche auszuweisen, und zwar unabhängig von der im Original verwendeten Schrift. 

Lange Zitate wiederholen wir zur verbesserten Lesbarkeit in lateinischen Buchstaben. Die Originalschreibung (mit \ceva{clang}) wurde beibehalten; nur offensichtliche Tippfehler wurden korrigiert. 

So nicht weiter angegeben, sind die Primärtexte in der Reihenfolge ihres vermutlichen Entstehens zitiert, nämlich \cite{cbasewebsite} $\rightarrow$~\cite{cbasestarbasemanual} und dann, abgesetzt im Fließtext, $\rightarrow$~\cite{cbasebook}. Die übrigen Quellen können als abgeleitet oder unvollständig gelten, bieten aber doch mitunter brauchbare Hinweise; sie werden zu einzelnen Details hinzugezogen.
 
Fachbegriffe aus der Stationssprache stehen auch innerhalb des Haupttextes in \ceva{\mbox{ceva}}. Beim ersten Auftreten geben wir eine lateinische Umschrift an: \cevain{cwelle}.






\section*{Forschungsgeschichte}\addcontentsline{toc}{section}{Forschungsgeschichte}
\fancyhead[RO]{Forschungsgeschichte}
\fancyhead[LO]{}

    Die \ceva{c-base} ist eine abgestürzte Raumstation unter Berlin, die sich seit 1995 rekonstruiert~\cite{cbasebook}, ein als \cevain{cbrp} - \cevain{c-base reconstruction project, cbrp} bekannter Prozess. Folgende Rekonstruktionsphasen \ceva{cbrp} lassen sich unterscheiden~\cite{cbasepressemap}~\cite{cbasebook}:
    % \twofonts{
    \begin{enumerate}
        \item Phase I.  Februar 1995 -- Mai 2000 Oranienburger Str. 2.  Nachbau und Rekonstruktion einer Schleusensektion der \ceva{c-base} Raumstation auf 270$m^2$.~\cite{cbasepressemap}~\cite{cbasebook} 
        \item Phase II.   Juni 2000 -- August 2002 Rungestr. 20.  Nachbau und Rekonstruktion der Multimodulstation \cevain{RS20} der \ceva{c-base} Raumstation auf 524$m^2$.~\cite{cbasepressemap}~\cite{cbasebook} 
        \item Zwischendeck.  September 2002 -- Juli 2003 Franz-Mehring-Platz Nr. 1. 
        Auslagerung wegen Wartungsarbeiten in der \ceva{RS20}~\cite{cbasepressemap}~\cite{cbasebook} 
        \item Phase III.  August 2003 -- heute. Rungestr. 20. 
         Erweiterung der Fläche auf ca. 720$m^2$ auf 2 Etagen. Fortgesetzte Konstuktion.
         % in der Multimodulstation RS20. Mainhall- und Brückenrekonstruktion, neue Schleusensektion, Ausbau \cevain{c-level} ... work in progress.~\cite{cbasepressemap}~\cite{cbasebook} 
    \end{enumerate}
    % }

Die Ergebnisse dieser Rekonstruktionen fanden ihren Niederschlag in diversen Publikationen (siehe Literaturliste auf S.~\pageref{sec:literatur}), deren Wichtigste wir im Folgenden auswerten. 


\section*{Befund}\addcontentsline{toc}{section}{Befund}
    
    \cite{ctour}, \cite{cbasepressemap} und \cite{cbasebook} stimmen im Befund und teils wörtlich überein. In \cite{ctour}, steht: 
    
    \twofonts{1995 wurden unter Berlin-Mitte die Überreste einer 4,5 Milliarden Jahre alten Raumstation entdeckt. Erste Forschungen ergaben, daß sich die c-för\-mi\-ge Raumstation mit ihrem Mittelpunkt unter dem heutigen Alexanderplatz befinden muß und aus 7 Ringen besteht. Aufgrund eines Fundstückes mit der Aufschrift "`\emph{c-base - be future compatible}"' und in Anlehnung an die Anzahl der Ringe, legte das anfänglich nur aus wenigen Mitgliedern bestehene Rekonstructionsteam den Projektnamen und die Aufteilung in sieben Arbeitsbereiche fest.}

    Außerdem:
    
    \twofonts{Die c-base ist eine abgestürzte raumstation. das unter berlin-mitte im märkischen sand versunkene artefact wird seit 1995 von über 100 zukunftsbegeisterten experten reconstruiert. Das raumschiff besteht aus sieben ineinander geschalteten c-förmigen ringen. jeder ring ist für ganz spezifische aufgabencluster modular ausgelegt.}
    
    
    Die Rekonstruktion zeigt eine fragmentierte Struktur von konzentrisch verschachtelten Ringen mit multiplen, untereinander verschiebbaren Modulen. 
    Die konzentrische Anordnung in der Architektur der Station zeigt \cref{fig:siebenringe}.

\begin{figure}[ht!]
    \centering
        \resizebox{0.6\textwidth}{!}{
        \documentclass{standalone}
\usepackage{tikz}
\usepackage{calc}
\usepackage{xcolor}
\begin{document}
%% c-base logo nachgebaut von penta.
%% alles nur geschätze Winkel und Abstände :/
%% um den code zu verstehen, einfach mal einzelne Teile auskommentieren und wieder einkommentieren (ctrl-#) und dann mal \draw[white] durch \draw[red] ersetzen, dann sieht man, was was ist.
%% viel Spaß damit.
\tikzset{
  pics/carc/.style args={#1:#2:#3}{
    code={
      \draw[pic actions] (#1:#3) arc(#1:#2:#3);
    }
  }
}
\definecolor{eins}{HTML}{FFFFFF}   %% Ring 1 "core"     - weiß - Mittelpunkt 
\definecolor{zwei}{HTML}{FF0000}   %% Ring 2 "com"      - rot -  Ring um Mittelpunkt
\definecolor{drei}{HTML}{FF7F00}   %% Ring 3  "culture" - orange - "Fenster" innen
\definecolor{vier}{HTML}{AAFF00}   %% Ring 4  "creativ" - grün -  fünf Module, quasi invertiert
\definecolor{fuenf}{HTML}{00FFFF}  %% Ring 5 "cience"  - blau - vier Module
\definecolor{sechs}{HTML}{800080}  %% Ring 6 "carbon" -  violett - drei Module mit "Strich"
\definecolor{sieben}{HTML}{4B0082} %% Ring 7 "clamp" -  indigo - viele "Fenster" außen

    \begin{tikzpicture}%
        \def\radi{10}%     

        %% äußerer Radius
        \filldraw[black] (0:0) circle (\radi-0.3);
        % was wegnehmen / weiß übermalen
        \draw[white, line width=1.6cm] (0:0) pic{carc=-55:90:\radi-0.72}; 

        %% clamp - viele kleine Fenster
        \foreach [count=\i] \ii in {%
        70,74,78,100,
        114,118,...,170,
        182,186,
        202,206,...,250,
        266,270,...,304}
            \draw[sieben, line width=0.5cm] (0:0) pic{carc=\ii:\ii-2:\radi-2}; 
        
        \draw[white, line width=3.05cm] (0:0) pic{carc=50:-55:\radi-3};
        
        \draw[white, line width=2cm] (0:0) pic{carc=-90:0:\radi-3.5};

  
        \draw[sechs, line width=1.8cm] (0:0) 
            pic{carc=82:135:\radi-3.5};
        \draw[sechs, line width=1.8cm] (0:0) 
            pic{carc=65:74:\radi-3.5};
        \draw[sechs, line width=1.8cm] (0:0) 
            pic{carc=53:62:\radi-3.5};
            
        \draw[black, line width=0.4cm] (0:0) 
            pic{carc=110:135:\radi-3.3};

        \foreach [count=\i] \ii in {1,2,3,4}
            \draw[fuenf, line width=1.5cm] (0:0) 
            pic{carc=90*\i:90*\i+45:\radi-5.5};


        \filldraw[vier] (0:0) circle (\radi-6.5);
        
        \foreach [count=\i] \ii in {1,2,3,4,5}
            \draw[black, line width=0.7cm] (0:0) 
            pic{carc=72*\i+27:72*\i+36+27:\radi-7};

        \filldraw[black] (0:0) circle (\radi-7.5);

        \foreach [count=\i] \ii in {160,172,...,490}
            \draw[drei, line width=0.4cm] (0:0) pic{carc=\ii:\ii-8:\radi-8};   
        
        \filldraw[zwei] (0:0) circle (\radi-8.5);
        \filldraw[black] (0:0) circle (\radi-8.8);
        \filldraw[eins] (0:0) circle (\radi-9.8);
     
    \end{tikzpicture}
\end{document}

    }
    \caption{Die 7 Ringe (Approximation)}
    \label{fig:siebenringe}
\end{figure}


    Die  Ringe werden in allen vorliegenden Texten von innen nach außen gezählt, bezeichnet und  farblich codiert  wie in  \cref{tab:ringe} aufgeführt. Wir verwenden die Werte aus \cite{cbasefarbschema}, das die älteste und zugleich genaueste Quelle zu sein scheint.\footnote{\cevain{jawohl mein designer} \cite[S. 47]{cbasebook}} 

    \begin{table}[ht!]
        \centering
        \begin{tabular}{rlllrr}
            \toprule
                1 & \ceva{core} & core & grau / weiß & \texttt{e7e7e8} & \Hrulek[eins]  \\
                2 & \ceva{com} & com & rot & \texttt{ed1c24} & \Hrulek[zwei] \\
                3 & \ceva{culture} & culture & orange & \texttt{fbad18} & \Hrulek[drei] \\
                4 & \ceva{creactive} & creactive & grün & \texttt{75c043}& \Hrulek[vier]  \\
                5 & \ceva{cience} & cience & cyan & \texttt{0089d0}& \Hrulek[fuenf]  \\
                6 & \ceva{carbon} & carbon & indigo & \texttt{0089d0}& \Hrulek[sechs]  \\
                7 & \ceva{clamp} & clamp  & ultraviolett / schwarz & \texttt{000000}& \Hrulek[sieben] \\
            \bottomrule
        \end{tabular}
        \caption{Nummern, Bezeichnung und Farbe der Ringe}
        \label{tab:ringe}
    \end{table}

    Die Ringfarbe korrespondiert nur schwach mit den Signalfarben innerhalb der Station, die wir hier der Vollständigkeit  halber in \cref{tab:bedeutungen} zeigen (nach \cite[S. 56]{cbasebook}).
    
    \begin{table}[ht!]
        \centering
        \begin{tabular}{lll}
            \toprule
                grau / weiß  & \Hrulek[eins] & überlebensstandard gesichert, temperature, drucc \\
                rot & \Hrulek[zwei] & lebendig, alarm, communication \\
                 orange  & \Hrulek[drei] & schädlich, aktiver prozess auf atomarer ebene \\
                grün & \Hrulek[vier]  & nicht humane biologische substanz, prozess \\
                cyan & \Hrulek[fuenf] & lowered thermal conditions \\
                indigo & \Hrulek[sechs] & \textit{(not assigned)}  \\
                schwarz & \Hrulek[sieben] & vacuum, death, hazard\\
            \bottomrule
        \end{tabular}
        \caption{Signalbedeutungen der Farben}
        \label{tab:bedeutungen}
    \end{table}

    Diese sieben konzentrischen Ringe werden im Folgenden einzeln genauer beschrieben. Dabei liegt der Fokus dieser Arbeit auf der allgemeinen Funktion innerhalb des autopoietischen Systems \ceva{c-base}; auf eine detaillierte Beschreibung einzelner Module, Artefakte und Bewohner wird verzichtet. Dem interessierten Leser sei dazu insbesondere \cite{cbasebook} nahegelegt. Aufgrund der multimodularen und letztlich höherdimensionalen Struktur der Station ist eine eineindeutige und ausschließliche Zuordnung einzelner Module zu bestimmten Ringen ohnehin fragwürdig.
    
    % Ähnlichkeiten mit lebenden Personen sind daher rein zufällig.
% \vfill

\clearpage


\pagestyle{fancy}
 
\csection{core}
% \section{core \hspace{2ex} \raisebox{1pt}{{\fontspec{[ceva-c2.ttf]}(core)}}}

\Hrule[eins]

\twofonts{
    core ist der innerste ring der c-base. er ist die commandoeinheit der raumstation: hier liegen die brücce, der centralcomputer exitc-beam und externe speichereinheiten, sowie der mino-reactor, der cybernetische antrieb der c-base auf quecksilberbasis.
    Die antenna ersteckt sich zur Zeit auf 368m über Normal Null sensor-array.
    }

\twofonts{
    commando-, antenneneinheit verwaltungseinheit mit brücke und commandodeck centralcomputer c-beam MBA-Ein\-hei\-ten und externe Speichermodule obere antenna: z.Zt. 368m über NN sensor-aray. ausrichtung auf die planetenoberfläche und tower 
}

\begin{newstuff}
    \lettrine{I}{m}
    Im Zentrum der Raumstation befindet sich eine Antriebs- und Steuerungseinheit. Die genaue Funktionsweise ist bereits recht gut erforscht \cite[S.~31ff]{cbasebook}. Es handelt  sich um eine multidimensionale, autopoietische Struktur, in deren Inneren sich eine Energiequelle befindet \cite[S. 31]{cbasebook}:

    \twofonts{Möbius-Band-accumulator (MBA). hauptenergiespeicher; als ring umgibt er den $\rightarrow$ minoreactor. die eigenschaft, energie unbestimmter grössen / einheiten aufzunehmen, zu speichern, wird dadurch ermöglicht, dass sie auf einem autoinitiierten endlosband in ständigem fluss ist.}

    Neben den technischen Details ist vor allem Bedeutsam, dass sich hier ein autopoietisches Systems selbst (re)-konstruiert. 
    Der Mino-Reactor ist vermutlich eine Singularität, aus der nur Dinge herauskommen können ("`Weißes Loch"'). Die hervorgebrachte Energie und Materie fluktuiert allerdings stark, so dass es desöfteren zur Materialisierung unvorhergesehener Objekte kommt.

    Der Mino-Reaktor \ceva{core} ist Quelle von unerschöpflicher Energie, Materie und Ideen, die dann im Möbius-Band-Generator zu sich selbst und ihrer Bedeutung in ihrer Umwelt finden.
    Das bedeutet, dass sich hier eine Art "`Bewusstsein"' der Raumstation befand bzw. befindet und befinden wird. 
    Dieses vereint digitale und analoge Aspekte und ist nicht auf Siliziumverbidnungen beschränkt, sondern kommuniziert und lebt auch in und mit anderen karbonbasierten Bewusstseinsmanifestationen. 

    Das lässt sich vereinfachend etwas so beschreiben: der zentrale Computer, besser: das zentrale Bewusstsein, der zentrale Wille und damit die zentrale Antriebseinheit der abgestürzte Raumstation befindet sich in einem fortgesetzten "`Reboot"'-Prozess, in welchem der {\fontspec{[ceva-c2.ttf]}core} nach und nach die verschiedenen Funktionen der Raumstation wieder "`online"' bzw. zum Leben bringt. Das geschieht auf materieller Ebene erst nach Evozierung von diesen Prozess anstoßenden  Bewusstseinsprozessen. Konkret wirkt auf einer nicht-materiellen Dimension der \ceva{core} auf dafür empfängliche Menschen, die so von der \ceva{c-base} inspiriert werden, zu ihrer Rekonstruktion beizutragen. 

    Die Tatsache, dass es sich um eine abgestürzte, also beschädigte bzw. unvollständige Raumstation handelt, erklärt auch, weshalb viele der Rekonstruktionsansätze für gemeine Menschen schwer oder gar nicht verständlich bleiben. Der autopoietische Prozess der Selbstwiedererschaffung der c-base geschieht durch mitunter fast infantil anmutende experimentelle Prozesse und Vorprozesse. Doch wie beim werdenden Menschen sind alle diese Experimente Zwischenstufen zum entwickelten, höheren Bewusstsein.  

    Der halberwachte Bordcomputer der Raumstation scheint mit einer uns nicht genau bekannten Technik immer wieder Individuen mit besonderen Begabungen anzuziehen, die dann durch dasselbe Medium inspiriert werden, die Rekonstruktion der Raumstation voranzubringen. Welche Qualitäten genau zu dieser Auswahl führen, ist aktuell nur in Ansätzen bekannt. Jedoch scheint die Strategie des \ceva{core} bislang insgesamt erfolgreich.
\end{newstuff}


    



\csection{com}
% \section{com \hspace{2ex} \raisebox{1pt}{{\fontspec{[ceva-c2.ttf]}(com)}}}

\Hrule[zwei]

\twofonts{
    der raumhafen der c-base: hier laufen die communicationseinrichtungen zusammen und verteilen sich die zugänge zur station.
        \begin{itemize}
            \item shuttlebays und hangars
            \item ancunftspromenade mit empfangsstation und wartehallen
            \item lifeboats und worcpots
            \item montagehallen, fracht- und laderäume 
        \end{itemize}
    }

\twofonts{
    cdcd-ring obere decks (5-7) com-bay an- und abflugshallel mit warteraum be- und entladungszone untere decks (1-4) shuttlebay für raumfahrzeuge fracht- und laderäume montagehallen 
    }
    
\begin{newstuff}
    \lettrine{H}{erum} um \ceva{core} befindet sich eine Verteilungs- und Mitteilungsstruktur namens \ceva{com}, was etwa dem Zentralnervensystem einer karbonbasierten Lebensform entspräche. Hier werden Inputs und Outputs vom und zum \ceva{core} koordiniert; doch zugleich geschieht in diesem Austausch etwas Wesentliches für die Selbstprogramierung der Raumstation. In \ceva{com} findet der Austausch von Informationen, aber auch von Anregungen und Anweisungen positiver, negativer, fragender oder imaginärer Natur statt. 

    Dazu schreibt \cite[S. 39]{cbasebook}:
    \twofonts{der com-ring ist die discussions- und präsentationsplattform der c-tation. die c-base generiert neue ideen zwischen allen ebenen und ringen und eröffnet contactmöglichkeiten jeder art - austausch und confrontation - die sowohl zur erweiterung der crew als auch zum aufbau interstellarer beziehungen führen. }

    Die Autopoiesis (Selbsterschaffung) der Raumstation geschieht durch kommunikative Prozesse zwischen den bereits aktivierten Modulen - unabhängig davon, in welchem Zustand sich diese Module aktuell befinden. Gleichzeitig wirkt der Zustand jedes Moduls auf alle übrigen Module und damit auf den komplexen Anregungszustand des \ceva{core} selbst zurück. Die entstehenden Schwingungen und Wellen in diesem komplexen, mehrdimensionalen und intertemporalen Kommunikationsnetzwerk lassen Gebilde entstehen, die weiter in die äußeren Ringe getragen werden (siehe z.B. \ceva{culture}); in \cite[S. 39]{cbasebook} heißt es:
    \twofonts{hier laufen sämtliche communicationseinrichtungen zusammen und verteilen sich die zugänge zur station.}

    \ceva{com} umfasst somit Informationseinheiten ebenso wie stoffliche Elemente und Bauteile in unterschiedlichen Aggregatzuständen und Aggregierungsgraden. Hier tauchen ständig neue Lebensformen und Aggregate verschiedener Be\-wusst\-seins- und Fertigungsstufen auf. Allen gemeinsam ist, dass sie nicht so, wie sie sind, abgeschlossen vollendet sind. Sie alle tragen bei zum Wiederwerden der Raumstation - durch Integration und durch Abwandlung. Daher ist Kom\-mu\-ni\-ka\-ti\-ons- Integrations- und Wandlungsfähigkeit unabdingbar für ein Andocken an der Raumstation. %, und Wesen ohne solche Bereitschaft werden hier unter Umständen auch aussortiert. 

    Die Verbindungsfunktion von \ceva{com} führt zu Kurzschlüssen zu allen anderen Ringen, denn Kommunikation ist bekanntlich die Grundlage jedes autopoietischen Systems. Das reibungslose Funktionieren der Kommunikationskanäle ist daher eine der vornehmsten Aufgaben bei der \cevain{reconstruccion}. Und schließlich umfasst dieser Ring auch die physischen Schleusen und Andockmöglichkeiten für den stofflichen, intellektuellen und emotionalen Austausch mit der Umwelt.
\end{newstuff}

\csection{culture}
% \section{culture \hspace{2ex} \raisebox{1pt}{{\fontspec{[ceva-c2.ttf]}(culture)}}}

\Hrule[drei]

\twofonts{
    der 3. ring von innen ist das cultumarium der c-base:
    hier kommt die crew zur entspannung und unterhaltung zusammen.
    \begin{itemize}
        \item großzügig angelegte freizeitpromenaden
        \item paradisedeccs
        \item culturedeccs mit c-no, triorama, performance \& events 
    \end{itemize}
    }

\twofonts{
    culture-deck obere decks (5-7) ringförmige kultur und freizeitpromenade mit parkanlage, cafes, triorama, cy\-ber\-world-5D. darunter: 2 unabhängig voneinander drehbare plattformen für wartung und reparatur der crafts in den shuttlebays 
    }
    
\begin{newstuff}
    \lettrine{D}{ie} Wortbedeutung von \ceva{culture} ist in etwa \emph{dem Wachstum dienende Pflege}. In diesem Ring reifen Ideen, aber auch kulturelle Objekte wie Kunstwerke, Programme, Systeme, Spiele und dergleichen mehr manifestieren sich und befördern somit die Reifung der partizipierenden Module und Raumfahrer. Nicht zufällig ist der dritte Ring die unmittelbare Stufe nach dem Andocken in \ceva{com}; hier wir die nächste Ebene des Miteinanders nach dem initialen Austausch und der Präprogrammierung erreicht. Entsprechend schreibt \cite[S. 67]{cbasebook}:

    \twofonts{der 3. ring von innen ist das culturmarium der c-base. hier kommt die crew zur entsprannung und unterhaltung in größzügig angelegten paradisedeccs zusammen. \textbf{culture} meint jegliche äußerung - auch leben an sich - materieller und geistiger art. ausstellungen, sportliche aktivitäten, performance \& theater, lesungen mit c-no, triorama, performance \& event werden begangen.}    

    Ähnlich wie die Raumstation insgesamt ist jedes teilhabende Modul ein selbsterschaffendes und selbsterhaltendes System, das über die Fähigkeit verfügt, Dinge und Konzepte größerer Schönheit hervorzubringen, die über es selbst hinausgehen. \ceva{culture} ist eine Gemeinschaftsfunktion, die in mehrere Richtungen wirkt: auf das Modul selbst zu seiner Selbstertüchtigung, auf das Miteinander der Module zu ihrem besseren gegenseitigen Verständnis, und auf die Gemeinschaft der Module als Teile des Werdensprozesses der Raumstation insgesamt.

    Im Werdensprozess der \ceva{c-base}\ \ceva{culture} vereinnahmt die Raumstation immer wieder bestehende Kulturtechniken, verwendet diese jedoch oft anders als vorgesehen und erforscht so immer neue Herangehensweisen an bestehende Systeme (\emph{hacking}). So kommt es zur Abwandlung bestehender Formate und Neuerschaffung von Spielen, Filmen und Klängen. Diese verdichten sich zu Mythen und Projektionen.

    Eine genauere Beschreibung der Energiequelle gibt \cite{cbasepressemap}:

    \twofonts{
        Die zentrale Antriebseinheit der Raumstation ist ein Cybernetischer Queck\-silber-Reaktor (CQR), dessen Feld die c-base von der Raumzeit ausschneiden soll, um sie über die Einstein-Rosenbaum-Brücke im Orbit eines noch relativ jungen Planeten im Sternbild Cassiopeia materialisieren zu lassen.} 
        
    Die \ceva{c-base} wird nicht so sehr durch materielle Energie, sondern erheblich auch durch kulturelle befeuert. So gilt mittlerweile als gesichert, dass die Aufeinanderfolge bestimmter Frequenzen (bzw. \emph{vibes}) zur Durchdringung von Hyperraumwänden und damit zum Absprengen von der Restrealität dienen kann. Entsprechend ist das \ceva{soundlab} integrativer Teil der propulsischen Vorrichtung der Station.

    Die so freigesetzten Strahlen bzw. Wellen unterschiedlicher Stofflichkeit und Periodizität müssen natürlich koordiniert werden, um sich nicht gegenseitig zu stören oder auszulöschen. Das Miteinander der Module ist Kern der \ceva{culture}, und dieses wird immer wieder durch intern ausgehandelte Regeln und Vorgänge koordiniert.
    % , die von Außenstehenden als ritualisierte Handlungen oder Spiele aufgefasst werden können. 
\end{newstuff}

\csection[vier]{creactive}\label{sec:creactive}

\twofonts[\cite{cbasestarbasemanual}]{
    datenblatt: bedeutung kunstwort für ideenkompass: kreativität, phantasie, fiction, aktivität

    creation-box in diesem segment befinden sich unter anderem gedächtniskammern, langschlafkabinen für die beliebten fivehoundred-year-projects (fyp), entspannungs- und ruhekonsolen. die ersten ideen und wahrheiten werden hier geschaffen. die creationbox kann an jede stelle des cience-ring gefahren werden.
    }
     \linek[\cite{cbasestarbasemanual}]{the most powerful part of c-base}

Der Ausdruck \linekin{most$\cdot$powerful}  hat Meinungsverschiedenheiten erzeugt, glaubten doch einige \cevain{creactive} daraus ein Primat gegenüber den übrigen Ringen ableiten zu können (Dogma von der Unfehlbarkeit des Designers).

Inzwischen gibt es drei breit akzeptierte Interpretationen:
\begin{enumerate}
    \item Der Satz betont die besondere Verantwortung der \cevain{creactivität}, z.B. angesichts leerer Flaschen (vgl. \cevain{matelight}).
    \item Er drückt eine besonders enge Verbindung von \ceva{creactive} und \ceva{core} aus. Vermutlich liegt die Wahrheit einmal mehr in der Mitte, bzw. die Aussagen schließen einander nicht aus.
    \item Er muss mit \linekonly{future$\cdot$compatible} zusammen gelesen werden; \ceva{creactive} Entscheidungen sind nicht endgültig, sondern müssen \cevain{beginngültig} sein.  
\end{enumerate}
Unter \ceva{beginngültigkeit} wird allgemein die Fruchtbarkeit einer Idee verstanden. Das Gegenteil ist \emph{non sequitur,} beispielsweise das abwürgende Dogma der \ceva{puristen} vom eingeschränkten Farbraum.

%\footnote{Zur Anwendbarkeit des \emph{tertium non datur} in der \ceva{c-tation} siehe}
% ist die herrschende (wenn auch nicht unangefochtene) Lehrmeinung, dass dieser Satz die besondere Verantwortung der \cevain{creactivität} angesichts leerer Flaschen betont (vgl. \cevain{matelight} \cite{matelight}).

\twofonts[\cite{ctour}]{
    der vierte ring von innen ist gleichzeitig auch der vierte ring von aussen. 
    wahrscheinlich diente er der crew deswegen als inspirationsgenerator. 
    in verschiedensten modulen sind creactive- und ideenfördernde einrichtungen untergebracht:
    \begin{itemize}
        \item die mobile creationbox, die an jede stelle des cience-rings fährt
        \item das cosmolab, die terraforming-wercstatt
        \item langschlafcabinen für die beliebten fivehoundred-year-projects (fyp)
        \item brainlabs und aerosphären
        \item gedächtniscammern, tancs, gedancengeneratoren
        \item völlig leere räume 
    \end{itemize}
    creactivity meint den ideencompass für creativität, phantasie, fiction, activität
    }

Beide Quellen erwähnen ein Artefakt namens \cevain{creationbox}. Es konnte offenbar vor allem in \ceva{cience} fahren - war also ein Teil von \ceva{creactive} innerhalb von \ceva{cience}. Was genau dieses war, ist uns heutigen nicht mehr bekannt. Vielleicht handelt es sich auch um ein mythisch-metaphorisches Vehikel; die kanonischen Quellen bezeichneten dann damit die Rückeinführung der \ceva{creactivität} in die Wissenschaft. 

\ceva{c-tour} spricht an dieser Stelle erstmals vom \cevain{ideencompass}, der in den archaischen Schriften offenbar sogar mit \ceva{creactive} überhaupt gleichgesetzt wurde. Erst im \ceva{c-booc} ist dieser \ceva{ideencompass} ein Objekt innerhalb von \ceva{creactive}. Vermutlich handelt es sich um eine mit dem Navigationssystem verbundenen Einheit. Ein mitten in der Neuzeit aufgefundenes Artefakt - \cevain{pentagame} - wurde mit diesem Interface identifiziert, wobei auch dies nicht unumstritten ist. Das \ceva{cosmolab} wird seit 2020 ausgegraben (zunächst fehlerhaft als \cevain{robolab} identifiziert).

Einige Aufmerksamkeit bekam in jüngster Zeit der Ausdruck \ceva{völlig leere räume}. Aufgrund der archaischen Ausdrucksweise geht die herrschende Lehre heute davon aus, dass hier damals tatsächlich existierende, physische und physisch leere Räume gemeint waren - so schwer so etwas aus heutiger Perspektive auch vorstellbar ist. - Demgegenüber halten einige Interpreten dies für eine an Zen-Buddhismus gemahnende Zeile; wieder andere verweisen auf die Leere hinter der Stirn so mancher Adepten. Vermutlich wird sich dieser Streit niemals auflösen lassen.

\begin{newstuff}
   \lettrine{W}{ährend} in \ceva{culture} das Miteinander der Module im Vordergrund steht, ist \ceva{creactive} hauptsächlich ein Ort zur Ermöglichung der schöpferischen Einsamkeit und des kreativen Miteinanders (\cevain{creactivität}). Hier finden sich Orte, die dem konzentrierten Kacken (\cevain{bio-\-4d--druck}, siehe auch \ceva{carbon}) dienen neben solchen, an denen die Module sich der Selbstpflege und dem Refacturing ihrer Programme widmen können. Ferner gibt es Orte zum gemeinsamen Hervorbringen von Plänen, Aggregaten, Artefakten und Prozessen - und zum Verbrennen von Kohlenstoffketten. 

   \twofonts[\cite{cbasebook},~S.~110]{auf dem creactive-ring bündelt die crew ihre künstlerischen und technischen visionen in actionen und projecte. hier befindet sich der ideenkompass für creativität, phantasie, fiction und activität.}

   An allen Quellen fällt die Reichhaltigkeit des Vokabulars und die Listenhaftigkeit des Wortlauts ins Auge; erwähnt werden unter anderem \cevain{creationbox}, \cevain{cosmolab}~\cite{ctour} und \cevain{ideencompass}.

    Diese Räume und Aggregate ermöglichen Schaffensprozesse im Wiederaufbauprogramm der Raumstation durch ihre schlichte Verfügbarkeit, aber auch durch die in ihnen durch die umliegenden \ring{ringe} zur Verfügung stehenden Ressourcen materieller und immaterieller Art. Das sind Werkzeuge und Materialien, aber auch Baupläne, Anregungen und intelligente Hinweise anderer Module. 

    Diese Anregungen und Hinweise sind nicht immer vollständig so, wie erwartet. Das erklärt sich aus dem immer noch infantil-semidebilen Zustand der abgestürzten Raumstation, ist aber auch notwendige Voraussetzung zur vorurteilsfreien Neuschaffung des höherdimensionalen Bauplans für die Station und alle seiner Module, deren künftige Beschaffenheit notwendigerweise die Bewusstseinskapazitäten aller Einzelmodule übersteigt.

    \ceva{creactive} strebt danach, die Verbindung mit den übrigen \ring{ringen} zu vereinfachen (Stichwort: \cevain{monorail}). Bei einigen Experimenten wurden bereits Geschwindigkeiten außerhalb relativistischer Mechanik erzielt, wodurch es mitunter zu Zeitdilatation oder gar Kausalitätsumkehrungen kommt. 
    
    \twofonts[\cite{cbasebook},~S.~117]{sämtliche labore der creactivity liegen in einem compacten cegment concentriert beieinander. die plattform kann mobil auf dem ring verschoben werden, so dass sie jede stelle des innen liegenden culture rings oder des außen liegenden cience rings erreicht. das creactive modul ist zur stelle, dort wo es gebraucht wird.}

    Zwar sind grundsätzlich alle \ring{ringe} modular und überall zur Stelle, aber besonders herausgestellt wird dies vor allem für \ceva{creactiv}. Zugleich wird der Zusammenhalt betont: \ceva{compact ... concentriert beieinander}.
    Durch die \ceva{creactivität} faltent sich die \ceva{c-base} über diesen \ring{ring} zurück auf sich selbst. So entstehen hier Kurzschlüsse und die Autoreferenz, die den autopoietischen Prozess am Leben hält. Die schöpferische Kraft \ceva{creactivität} spielt eine\cevain{c\_lüsselrolle} bei der Emergenz der \ceva{c-tation}.

    \ceva{creactive} ist als Energieverstärkungs- und Umwandlungsmodul dem \ceva{MBA} (vgl. \cref{sec:core}) verwandt. Diese Umwandlungs- oder Energiequelle \ceva{creactivität} ist nicht zu verwechselnd mit \cevain{cunst}, die hier gar nicht erst erwähnt wird. Für \ceva{creactivität} ist \ceva{cunst} ein notwendiges Beiprodukt.
    
    Die \ceva{cwellen} rechtfertigen eine Sonderstellung von \ceva{creactive} in der Stationsdynamik und -topologie. Daraus lassen sich einige Schlüsse ziehen, was wir weiter unten, in \cref{sec:mathematik}, getan haben werden. 
\end{newstuff}

\csection{cience}
% \section{cience \hspace{2ex} \raisebox{1pt}{{\fontspec{[ceva-c2.ttf]}(cience)}}}



\Hrule[fuenf]

\twofonts[\cite{cbasestarbasemanual}]{
    bedeutung kunstwort. der prozeß konzeption, umsetzung, produkt. auch forschung, aus-/fortbildung in wissenschaft, handwerk und technik

    cience modulor hier sind alle wesentlichen forschungs-und entwicklungseinrichtungen untergebracht. eine auswahl an einzelnen arbeitsbereichen und stationen: evolutionsdesign wahrnehmungsforschung (social art) soziochemie (geruchsstoffe) bio\-labs medienlabore 
    }
    
\twofonts[\cite{ctour}]{
    cience ist der forschungsring der c-base: hier liegen alle wesentlichen forschungs- und entwicclungseinrichtungen der station in verschiedensten arbeitsbereichen focussiert beieinander. 
    die ersten ideen und wahrheiten werden hier geschaffen.
        \begin{itemize}
            \item genlabs des evolutionsdesign
            \item die biolabs mit arboretum
            \item die soziochemie der socialartists
            \item die medialabs
        \end{itemize}
    cience meint den prozess der conzeption, umsetzung, production, forschung, aus- und fortbildung in wissenschaft, handwerk und technik
    }



\begin{newstuff}
    \lettrine{Ü}{berwiegen} in den vorgelagerten \ring{Ringen} noch deutlich die traumhaften, inspirationellen, dialogischen und meditativen Aspekte, so verdichtet sich der autopoietische Prozess der Wiedererschaffung der Raumstation in \ceva{cience} durch rigorose Methodik und Präzision der Hingabe an definierte Ziele, die in den vorherigen Stufen erahnt und erträumt wurden.

    Entsprechend kommt es hier zu einer Prüfung der Möglichkeiten innerhalb der bereits realisierten Umgebung mit den bereits hervorgebrachten Materialien, Werkzeugen und mitarbeitenden Modulen. Allerdings ist \ceva{cience} nicht nur Forschung, sondern auch Anwendung von Erkenntnissen an der Grenzlinie zum Unerreichten. Hier werden also neue Module und Artefakte produziert, die dann neue Möglichkeiten erschließen. Der Beitrag von \ceva{cience} zum Wachstum der Station ist damit zentral.

    \twofonts[\cite{cbasebook},~S.~135]{auf dem forschungsring cience der c-base liegen alle wesentlichen entwicklungseinrichtungen der station in verschiedensten arbeitsbereichen focussiert beieinander. [...] der wissenschaftliche bereich der c-tation vereint forschung und bildung und meint den prozess der conzeption, umsetzung, production, aus- und fortbildung in wissenschaft, handwerc und technic[.] vergangenheitsbewältigung und cucunfstforschung werden in den genlabs des evolutionsdesign, den biolabs rund um das arboretum, der soziochemie der socialartists und in den medialabs betrieben.}

    \ceva{cience} ist nicht auf digitale Forschung, aber auch nicht auf materielle Experimente beschränkt. Die Erforschung des Vorhandenen, also der bereits manifesten Teile der Station umfasst auch die Beschäftigung mit den Wirkmechanismen seiner Bewohnenden, also ihrer Erregungszustände, ihrer Emotionen, ihrer Vorstellungen, ihres Miteinanders. Dabei stellt sich heraus, das permanente Infragestellung zwar gestattet und notwendig ist, aber gewonnene Erkenntnisse auch als Ecksteine des Fortschrittes dienen und angenommen sind.

    Es ist in einem autopoietischen System nicht möglich, zwischen Hervorbringendensystem und Ergebnissystem zu unterscheiden. Insofern ist der aktuelle Erkenntnisstand der Wissenschaft von der \ceva{c-base} identisch mit dem aktuellen Zustand der Station. Es gilt mithin die "`informatische Identität"'
    \begin{equation}
        \text{code}\;\widehat{=}\;\text{documentation}
    \end{equation}
    Allerings ist im üblichen Zeitablauf die Vergangenheit immer größer ist als die Gegenwart. Daher gilt intertemporal
    \begin{equation}
        \text{code} < \text{documentation}
    \end{equation}
    was soviel bedeutet, wie: die Ansammlung vergangener Fakten ist immer größer als die Menge der aktuellen Fakten, und die Datensumme von Backups und Dokumentation ist immer größer als der aktuell ausgeführte Code. Entsprechend ist die Vergangenheit der Raumstation größer als ihre Gegenwart. Und dennoch ist von jedem gegenwärtigen Zeitpunkt aus die Zukunft unendlich größer als beide~\cite{encyclopaedia}.

    \twofonts[\par\hfill"`rabbinische weisheit"', \cite{cbasebook},~S.~47]{die zucunft hat eine lange vergangenheit}

    \ceva{cience} besteht aus dem Prozess des Forschens, ist aber zugleich die Summe bzw. die Potenz oder Fakultät der Erkenntnisse. Diese sind in der Raumstation verteilt und vernetzt abgelegt, und zwar sowohl virtuell als auch materiell. 
    Die Multisprachlichkeit der Raumstation, ihre nicht immer verständliche Datensystematik, die temporal und spatial unterschiedliche Erreichbarkeit einzelner Informationsbrocken und der  Informationsträger ist zugleich Objekt und Subjekt von \ceva{cience}.

    Insofern \ceva{cience} ein erzeugendes Subsystem der Station ist, kommt es hier besonders zur Erschaffung neuer Artefakte nach den durch \ceva{core} inspirierten Bauplänen. Jedes so emanierende Objekt schafft Faktizität und damit in sich selbst kaum widerlegbare Wahrheit. Ergo gilt: \cevain{wer baut hat recht}~\cite[S.~47]{cbasebook}.
\end{newstuff}

\csection{carbon}
% \section{carbon \hspace{2ex} \raisebox{1pt}{{\fontspec{[ceva-c2.ttf]}(carbon)}}}

\Hrule[sechs]

\twofonts{das wohnungsdeckmodul bietet familiengerechte lebensqualität für die crew:
    die c-base ist ein generationenschiff und hat platz für eine besatzungsstärke von bis zu 5000 carboneinheiten. 
    In einem notfall kann jede der 144 habitate von der station gelöst werden und eigenständig manövrieren.
    \begin{itemize}
        \item 144 lifehabitatmodule mit je 12 habitateinheiten
        \item ver- und entsorgungseinheiten
        \item foodsupply mit medical support unit
        \item cindergarten
        \item schulen
    \end{itemize}
    }

\twofonts{
    life habitat familiengerechte wohnräume für die besatzung mit schlaf- und aufenthaltsraum, küche und bad incl. ver- und entsorgungseinheiten. 
    }

\begin{newstuff}
    \lettrine{N}{icht} jeder Computer oder jedes kalkulierend denkende System ist notwendigerweise siliziumbasiert. Die Raumstation als System insgesamt funktioniert eindeutig als Zusammenspiel von mehreren Dimensionen und von Systemen, deren Schnittstellen ganz unterschiedlich sind. Insbesondere arbeiten die digitalen Systeme mit den karbonbasierten Systemen zusammen über optische und haptische Interfaces, darunter Monitore, Brillen, Tastaturen und selbst Brettspiele wie \cevain{pentagame}. Nur das gelingende Interagieren aller Beteiligten bringt die Stationsforschung voran und damit das Ziel des Abflugs näher. \cite[S. 169]{cbasebook} erläutert:

    \twofonts{carbon ist das basiselement der organischen chemie und documentiert die geschichte der auf cohlenstoff basierenden lebensformen auf der c-tation. es zeichnet die bisher becannte stationsgeschichte auf ...}
    
    Die Karboneinheiten haben selbstredend Bedürfnisse, denen die Station vorsorglich Rechnung trägt.
    Hierzu gehören die im kanonischen Text aufgeführten pysiologischen Annehmlichkeiten. Auch befindet sich die  Raumstation im Aufbau, es ist also zu erwarten, dass sich die kulinarische, hygienische, gesundheitliche Situation im Werdensprozess der c-base weiter verbessern wird. Das gilt für die Qualität der vorhandenen gasförmigen, flüssigen, halbflüssigen und festen Stoffströme ebenso wie für die akustischen, optischen, haptischen und thermischen Umgebungsvariablen. 

    Hier befinden sich auch verschiedene Kultivierungsmodule zur Wandlung ordinärer Stoffe in höhere Elemente (\emph{aus Scheiße Gold machen,} \cevain{Alchemie}). In einer Reihe von kontrollierten Versuchen wurde hier beispielsweise Kuhmilch in interessante andere Stoffe umgewandelt. Andere Versuche mit verlegten Kleidungsstücken und vergessenen \mbox{Pizza}\-kartons blieben dagegen bislang überwiegend erfolglos.
  
    Insofern der Werdensprozess der Station jedes einzelne Modul umfasst, gehört zum Prozess der Selbstbewusstwerdung auch ein Anwachsen des Bewusstseins der Auswirkungen der eigenen Handlungen und Nichthandlungen auf das Umgebungshabitat, das Selbstwohlbefinden und das Wohlbefinden der anderen. Das kohlenstoffbasierte Gedächtnis der Station muss allerdings immer wieder trainiert und herausgefordert werden.

    Teil dieses Ökosystem der Station und ihrer Bewohner sind diverse andere terristrische und extraterrestrische Lebensformen, die sich auf mitunter sehr anderen Raumzeitlinien bewegen und so verschiedene Zeit- und Raumauffassungen haben.\footnote{So z.B. der Symbiont \cite{symbiont}.} Innerhalb dieser unterschiedlichen Bezugssysteme und auch zwischen ihnen emergieren unregelmäßig neue, fruchtbare und erfreuliche Interaktionen, aus denen neue kulturelle und auch biologische Produkte als neue Bezugspunkte hervorgehen. 

    Der Zustand der Station erlaubt aktuell offenbar nicht den Betrieb der in \cite{ctour} und \cite{cbasestarbasemanual} angekündigten \ceva{cindergärten} und \ceva{schulen}, und die Station ist weiterhin nur volljährigen zugänglich. Ebenso ist die Übernachtung zur Zeit der Drucklegung als \cevain{Schlaftäterschaft} noch unzulässig \cite[S. 58]{cbasebook}.
\end{newstuff}

\csection[sieben]{clamp}\label{sec:clamp}

\twofonts[\cite{cbasestarbasemanual}]{
    bedeutung entwicklung, neugier, halboffene, noch nicht vollendete form zum kreis. - der buchstaben c

    outer construction c-förmiges aussenskelett der raumschifkonstruktion, kurskorrektureinheiten, triebwerke sensoraray 
    }

Hier wird von \cevain{kurskorrektureinheiten} gesprochen; später heißt es dann \ceva{coursecorrectursupporter}. Leider sind uns diese Aggregate bzw. die Kenntnis über sie verloren gegangen, und es kommt heute nur noch sehr selten zu bedeutenden Kurskorrekturen; das mag auch damit zusammenhängen, dass für eine solche vermutlich der in \ceva{creactive} (siehe dort) verortete (verlorene?) \ceva{ideenkompass} benötigt werden würde. 

\twofonts[\cite{ctour}]{
    der größte ring umschließt als stützconstruction alle anderen ringe der raumstation.
        \begin{itemize}
            \item antennen- und sensoreneinheiten
            \item coursecorrectursupporter
            \item siri-sonden-doccs
            \item das ectonet
        \end{itemize}
    clamp ist der buchstabe c: er steht für die halboffene, noch nicht vollendete form zum creis und meint neugier, aufmerksamkeit und entwicklung
    }

Beiden Quellen ist eine gewisse \cevain{cnappheit} zueigen; sie stellen aber beide heraus, dass zumindest dieser \ring{ring} \cevain{c-förmig}, also ungeschlossen, ist. \cevain{halboffen} wird allgemein nicht als Winkelangabe gelesen. Die Bedeutung von \cevain{ectonet} ist unbekannt.

    \begin{newstuff}
        \lettrine{J}{edes} autopoietische System braucht eine zusammenhangstiftende Außenhülle als  Abgrenzung gegenüber den Umgebungssystemen. Diese Hülle dient dem Schutz der inneren Selbständigkeit, des inneren Schutzraumes selbst und ist damit unbedingt aufrecht zu halten. Nur durch eindeutige Schranken ist die freie Entfaltung innerhalb dieser Grenzen und schließlich auch das friedfertige Ausweiten dieser Grenzen möglich.

        \twofonts[\cite{cbasebook},~S.~189]{der äußerste ring clamp formt den buchstaben c: er steht für die halboffene, noch nicht vollendete form zum creis und mein neugier, aufmercsamceit und entwicclung. errepräsentiert den c-base-cosmos und gibt eine übersicht der activitäten und möglichkeiten der c-tation. der größte ring umschließt als stützconstruction alle anderen ringer der raumstation und beherbergt das ectonet sowie curscorrectursupporter, die siri-sondendoccs und periphäre antennen- und sensoreinheiten.}

        \ceva{clamp} schließt die Station nach außen ab, ohne jedoch eine Ausgrenzungsbarriere zu sein. Sie ist aber auch die Schwelle, an welcher sich die Raumstation gegenüber ihrer Umgebung befindet, und der letztlich unüberbrückbare Gegensatz zwischen innerem Bewusstsein der Raumstation und ihrer Außenwelt. {\color{white}\cevain{fnord}}
        
        Zugleich ist \ceva{clamp} die \cevain{chnittstelle} vom Äußeren zum Inneren, also die Sensorhülle des Gesamtsystems. Fraglos bildet sich an dieser Fläche das Außenbild der Umwelt von der Station, aber auch das Innenbild der Station von der Außenwelt. Hier kommt es zu Anfragen, zu Beeinflussungen, zum Empfang von Angeboten, zum Aussenden von Hilferufen und vergleichbaren Sprechakten und Anschlüssen. Nicht zuletzt ist \ceva{clamp} Ort des Übergangs von Außen und Innen und damit auch der Ort der Assimilation weiterer Module in das autopoietische Projekt \ceva{c-base}.
        
        \ceva{clamp} ist zugleich ein Festhalten am Bestehenden und durch Rückfaltung auf \ceva{com} zugleich Anschlusspunkt für die Penetration durch Neues innerhalb von \ceva{creativ}. Die Station ist ein werdendes Wesen, und daher sowohl neugierig und offen als auch schüchtern und schutzbedürfig. Es sucht in seiner Umgebung nach Orientierung. Zeitgleich ist die \ceva{c-base} unfassbar alt und weise, auch wenn sie sich dessen nur teilbewusst ist.
        
        \ceva{clamp} ist sowohl die Verankerung der Station in der Restrealität als auch die Sollbruchstelle beim möglichen künftigen Abflug. Aufgrund seiner Wichtigkeit für die Integrität der Stationswerdung ist dieser \ring{ring} besonders mit dem \ceva{Core} verbunden, sodass sich die Station hier zirkulär auf ihren Ursprung zurückfaltet. Damit bildet \ceva{clamp} den Ereignishorziont des mit dem Weißen Loch in \ceva{core} korrespondierenden Schwarzen Loches und somit seinen Gegenpol in der Raumzeit. Das kann auch erklären, wieso längst weggeworfene Gegenstände oder entfernte Kohlenstoffeinheiten sich mitunter erneut manifestieren.
        
        Die entstehenden Energieflüsse zwischen den \ring{ringen} und über die \ring{ringe} verstärken sich unter Umständen gegenseitig und erzeugen dann parallele, aber unterschiedliche Wirklichkeiten. Insgesamt kommt es dabei des öfteren zu deutlichen Raumzeitverzerrungen, die sehr unterschiedliche Wahrnehmung der relativen Magnitude von Ereignissen durch unterschiedliche Betrachter zur Folge haben.
    \end{newstuff}

\section*{Ergebnis}\addcontentsline{toc}{section}{Ergebnis}\label{sec:ergebnis}

Unser Rundgang durch die Ringe hat gezeigt, dass die Funktion jedes einzelnen Rings nur aus dem komplexen Zusammenspiel mit den anderen Ringen verstanden werden kann. Daher ist das Komplexitätsmaß auf dieser Ebene zumindest
\begin{equation}
    T(n) = \mathcal{O} (n!) > 7! = 5040
\end{equation}
- und das ungeachtet des Umstandes, dass jeder Ring aus diversen Modulen und diese aus Aggregaten als jeweiligen Subsystemen weiter ausdifferenziert. Wir haben es also mit einem hochkomplexen, selbstreferentiellen und autopoietischen System zu tun, das insgesamt mehr Zustände annehmen kann, das sich im bekannten Universum abbilden ließe. Damit ist das System insgesamt ein mögliches Spiegelbild des Gesamtrestuniversums und bringt selbst in seiner Autopoiesis unendlich viele Welten hervor.

Die weitere Rekonstruktion der Station sollte die "`Ringhaftigkeit der Ringe"' mit berücksichtigen. Vermutlich ist die gewissermaßen "`flache"' Struktur, die wir bei den bisherigen Rekunstruktionsphasen ausgemacht haben, bloß eine zweidimensionale Projektion der eigentlich mehrdimensionalen Raumstation, etwa wie die flache Scheibe, die entsteht, wenn ein Ballon seine Luft verliert. Die Einzelrings sollten also nicht mehr nummeriert, sondern durch Winkelangaben als Vielfache von $\nicefrac{\pi}{7}$ bezeichnet werden.

Die Erkenntnis, dass sich \ceva{core} und \ceva{clamp} über eine Einstein-Rosen-Brücke berühren, weist darauf hin, dass die Geometrie der Station eher als Torus bezeichnen lässt (\cref{fig:torus}). Dessen äußerer Äquator ist \ceva{creactive}. Eine Ausweitung der Kreativität (ihre Entfernung von der Singularität) bedeutet damit ein Anschwellen dieses Torso und somit Wachstum der Station; bildlich entspräche das in etwa der Inflation eines Rettungsrings. Und das ist auch ein Bild schön genug, dieses Papier zu beschließen.

\begin{figure}[ht!]
    \centering
    \resizebox{0.4\textwidth}{!}{
        \documentclass{standalone}
\usepackage{pst-solides3d}
\begin{document}

\begin{pspicture}(-3,-4)(3,6)
\psset{Decran=30,viewpoint=20 40 30 rtp2xyz,lightsrc=viewpoint}
 \psSolid[object=tore,r1=2.5,r0=1.5,ngrid=18 36,fillcolor=white]%
\end{pspicture}

\end{document}
    }
    \vspace{2cm}
    \caption{Torus}
    \label{fig:torus}
\end{figure}


\clearpage



\fancyhead[RO]{Literatur}\addcontentsline{toc}{section}{Literatur}\label{sec:literatur}

    \nocite{*}
    \printbibliography

\end{document}


\end{document}
