\csection{culture}
% \section{culture \hspace{2ex} \raisebox{1pt}{{\fontspec{[ceva-c2.ttf]}(culture)}}}

\Hrule[drei]

\twofonts{
    der 3. ring von innen ist das cultumarium der c-base:
    hier kommt die crew zur entspannung und unterhaltung zusammen.
    \begin{itemize}
        \item großzügig angelegte freizeitpromenaden
        \item paradisedeccs
        \item culturedeccs mit c-no, triorama, performance \& events 
    \end{itemize}
    }

\twofonts{
    culture-deck obere decks (5-7) ringförmige kultur und freizeitpromenade mit parkanlage, cafes, triorama, cy\-ber\-world-5D. darunter: 2 unabhängig voneinander drehbare plattformen für wartung und reparatur der crafts in den shuttlebays 
    }
    
\begin{newstuff}
    \lettrine{D}{ie} Wortbedeutung von \ceva{culture} ist in etwa \emph{dem Wachstum dienende Pflege}. In diesem Ring reifen Ideen, aber auch kulturelle Objekte wie Kunstwerke, Programme, Systeme, Spiele und dergleichen mehr manifestieren sich und befördern somit die Reifung der partizipierenden Module und Raumfahrer. Nicht zufällig ist der dritte Ring die unmittelbare Stufe nach dem Andocken in \ceva{com}; hier wir die nächste Ebene des Miteinanders nach dem initialen Austausch und der Präprogrammierung erreicht. Entsprechend schreibt \cite[S. 67]{cbasebook}:

    \twofonts{der 3. ring von innen ist das culturmarium der c-base. hier kommt die crew zur entsprannung und unterhaltung in größzügig angelegten paradisedeccs zusammen. \textbf{culture} meint jegliche äußerung - auch leben an sich - materieller und geistiger art. ausstellungen, sportliche aktivitäten, performance \& theater, lesungen mit c-no, triorama, performance \& event werden begangen.}    

    Ähnlich wie die Raumstation insgesamt ist jedes teilhabende Modul ein selbsterschaffendes und selbsterhaltendes System, das über die Fähigkeit verfügt, Dinge und Konzepte größerer Schönheit hervorzubringen, die über es selbst hinausgehen. \ceva{culture} ist eine Gemeinschaftsfunktion, die in mehrere Richtungen wirkt: auf das Modul selbst zu seiner Selbstertüchtigung, auf das Miteinander der Module zu ihrem besseren gegenseitigen Verständnis, und auf die Gemeinschaft der Module als Teile des Werdensprozesses der Raumstation insgesamt.

    Im Werdensprozess der \ceva{c-base}\ \ceva{culture} vereinnahmt die Raumstation immer wieder bestehende Kulturtechniken, verwendet diese jedoch oft anders als vorgesehen und erforscht so immer neue Herangehensweisen an bestehende Systeme (\emph{hacking}). So kommt es zur Abwandlung bestehender Formate und Neuerschaffung von Spielen, Filmen und Klängen. Diese verdichten sich zu Mythen und Projektionen.

    Eine genauere Beschreibung der Energiequelle gibt \cite{cbasepressemap}:

    \twofonts{
        Die zentrale Antriebseinheit der Raumstation ist ein Cybernetischer Queck\-silber-Reaktor (CQR), dessen Feld die c-base von der Raumzeit ausschneiden soll, um sie über die Einstein-Rosenbaum-Brücke im Orbit eines noch relativ jungen Planeten im Sternbild Cassiopeia materialisieren zu lassen.} 
        
    Die \ceva{c-base} wird nicht so sehr durch materielle Energie, sondern erheblich auch durch kulturelle befeuert. So gilt mittlerweile als gesichert, dass die Aufeinanderfolge bestimmter Frequenzen (bzw. \emph{vibes}) zur Durchdringung von Hyperraumwänden und damit zum Absprengen von der Restrealität dienen kann. Entsprechend ist das \ceva{soundlab} integrativer Teil der propulsischen Vorrichtung der Station.

    Die so freigesetzten Strahlen bzw. Wellen unterschiedlicher Stofflichkeit und Periodizität müssen natürlich koordiniert werden, um sich nicht gegenseitig zu stören oder auszulöschen. Das Miteinander der Module ist Kern der \ceva{culture}, und dieses wird immer wieder durch intern ausgehandelte Regeln und Vorgänge koordiniert.
    % , die von Außenstehenden als ritualisierte Handlungen oder Spiele aufgefasst werden können. 
\end{newstuff}