\section*{Quellenlage}\addcontentsline{toc}{section}{Quellenlage}
\fancyhead[RO]{Quellenlage}
% \fancyhead[LO]{}

Abgesehen von Fragmenten ist die älteste Quelle das \cevain{starbase-manual7}~\cite{cbasestarbasemanual}; 2003 ist der terminus ante quem seiner Auffindung. Diese Quelle sagt über sich selber:
    \twofonts[\cite{cbasestarbasemanual}]{
    daten und fakten der raumstation unter berlin werden hier erstmalig präsentiert.
    }

Etwas später - (terminus ante quem: 2006) - erschien dann die \cevain{c-tour}~\cite{ctour}. Gegenüber dem \ceva{starbase-manual7} wirkt diese Quelle vergeistigter, schreibt beispielsweise mehr von Informations- als Materialtaustausch. Beide Quellen gelten heute als kanonisch.

Welche dieser beiden Quellen tatsächlich "`älter"' ist - also innerhalb der Zeitachse der Raumstation, nicht innerhalb der Entdeckungsgeschichte - ist nicht vollständig geklärt. 
Beide Texte sind eindeutig archaisch und wurden von großen Heiligen geschrieben.  Wir zitieren sie ob ihrer Bedeutung zu Beginn jedes Abschnitts in der Reihenfolge ihrer Auffindung.

Das heute als autoritativ angesehene, 2015 gedruckt erschienene \cevain{c-booc}~\cite{cbasebook} benutzt diese Quellen und entspricht ihnen im Großen und Ganzen; wo es weitere Informationen bereit hält, sind diese hier ebenfalls wörtlich angeführt. Diese drei Quellen bilden den \cevain{canon}.

Die übrigen Quellen können als abgeleitet oder unvollständig gelten, bieten aber doch mitunter brauchbare Hinweise; sie werden zu einzelnen Details hinzugezogen. Hierzu gehört  insbesondere die \cevain{pressemappe}~\cite{cbasepressemap} in unterschiedlichen Überlieferungsstufen und einem terminus ante quem 2007. 

Die \ceva{c-base} verwendet eine eigene Schreibweise namens  \cevain{clang}; dabei wird auf Kapitalisierung verzichtet,  Sibilanten und der stimmlose velare Plosiv (\emph{k}) werden durch den Buchstaben \cevain{c} ersetzt, usw; Ähnliches gilt für daraus gebildete Konsonantencluster.\footnote{Eine genaue Übersicht über die Ersetzungsregeln bietet~\cite[S.~46]{cbasebook}.} In unseren eigenen Texten verzichten wir darauf.

Die \ceva{c-base} kennt eine eigene Schrift namens \cevain{ceva}. Wir nutzen sie, um Primärquellen und Stationsfachbegriffe als solche auszuweisen, und zwar unabhängig von der im Original verwendeten Schrift. 

Lange Zitate wiederholen wir zur verbesserten Lesbarkeit in lateinischen Buchstaben. Die Originalschreibung (mit \ceva{clang}) wurde beibehalten; nur offensichtliche Tippfehler wurden korrigiert. 

Fachbegriffe aus der Stationssprache stehen auch innerhalb des Haupttextes in \ceva{\mbox{ceva}}. Beim ersten Auftreten geben wir eine lateinische Umschrift an: \cevain{Ring}. 

Unsere Interpretation beginnt in den folgenden Abschnitten jeweils mit einer Initiale, wodurch ausgedrückt wird, dass der Text einen interpretativen Charakter besitzt.



