\section{Quellenlage}%\addcontentsline{toc}{section}{Quellenlage}
\fancyhead[RO]{Quellenlage}
% \fancyhead[LO]{}

Die ältesten Quellen sind die fragmentarischen frühen Logbucheinträge, wobei \cite{cbaselogbuchpre} so genannte \cevain{c-files} sind; sie wurden dem \cevain{datenspeicher} entnommen.  Die Einträge in \cite{cbaselogbuchnow} stammen dagegen nach Aussage von \cite{cbaselogbuchpre} von der damaligen crew \cite{cbaselogbuchpre}, der \cevain{urcrew}. In \cite{cbaselogbuchnow} wird ein Artefakt namens \cevain{internetplanetarium} erwähnt, von dem es heißt: 

 \twofonts[\cite{cbaselogbuchnow}]{reconstruierte daten werden in form eines internetplanetariums aufbereitet und sind ab diesem tag der weltöffentlichceit zugänglich.} 
 
 Das \ceva{internetplanetarium} ist verschollen; sollte es wieder auftauchen, so würde es natürlich ob seiner Anciennität als autoritativ gelten.
 
Die wahrscheinlich älteste auf uns gekommene, vollständige Quelle ist das \cevain{starbase-manual7}~\cite{cbasestarbasemanual}; 2003 ist der terminus ante quem (t.a.q., frühestmögliche Zeitpunkt) seiner Auffindung. Diese Quelle sagt über sich selber \emph{expressis verbis}:
    \twofonts[\cite{cbasestarbasemanual}]{
    daten und fakten der raumstation unter berlin werden hier erstmalig präsentiert.
    }

Daher wird oft eine Identität von \cevain{internetplanetarium} und \cevain{starbase-manual7} vermutet. %Auf jeden Fall ist die Quelle autoritativ.

Etwas später - (terminus ante quem: 2006; wahrscheinliche Endform: 2008) - erschien dann die \cevain{c-tour}~\cite{ctour}. Gegenüber dem \ceva{starbase-manual7} wirkt diese Quelle durchgeistigter, schreibt beispielsweise mehr von Informations- als Materialtaustausch. Es könnte diese Quelle eine Art Update gegenüber dem \ceva{starbase-manual7} sein. Beide Quellen gelten heute als kanonisch.

Welche dieser beiden Quellen tatsächlich "`älter"' ist - also innerhalb der Zeitachse der Raumstation, nicht innerhalb der Entdeckungsgeschichte - ist nicht vollständig geklärt. Angesichts der zeitlichen \cevain{verwirrung} ist eine abschließende Klärung auch unwahrscheinlich.
Beide Texte sind eindeutig archaisch und wurden von großen Heiligen geschrieben.  Wir zitieren sie ob ihrer Bedeutung zu Beginn jedes Abschnitts in der Reihenfolge ihrer Auffindung.

Das epochemachende  \cevain{c-booc}~\cite{cbasebook} ist demgegenüber deutlich später, nämlich 2015, erschienen; es liegt in gedruckter Form vor und ist autoritativ. 
\ceva{\mbox{c-booc}} benutzt die älteren Quellen und entspricht ihnen im Großen und Ganzen; wo es weitere Informationen bereit hält, sind diese hier ebenfalls wörtlich angeführt. Diese Quellen bilden den \cevain{canon} (\cref{tab:kanon}) bzw. die \cevain{c\_rift} und damit die Grundlage unserer Exegese.

\begin{table}[ht!]
    \centering
    \begin{tabular}{r|lrl}
        \toprule
        % \multicolumn{2}{c}{Quelle} & Jahr & Epoche \\
        % \midrule
        %  \ceva{c-booc} & c-booc & 2015 & Übergang Mittelalter$\rightarrow$Neuzeit\\
        %  \ceva{c-tour} & c-tour & 2006 & frühes Mittelalter \\
        %  \ceva{starbase-manual7} & starbase-manual7 & 2003 & Frühzeit (?) \\
        %  \ceva{logbuch now} & logbuch now & 2000 & Archaik\\
        %  \ceva{logbuch pre} & logbuch pre & 1995 & Archaik, Vorzeit \\
        \multicolumn{2}{c}{\cevain{c\_rift}} & Jahr & Epoche \\
        \midrule
         \ceva{logbuch pre} & logbuch pre & 1995 & Archaik, Vorzeit \\
         \ceva{logbuch now} & logbuch now & 2000 & Archaik\\
         \ceva{starbase-manual7} & starbase-manual7 & 2003 & Frühzeit (?) \\
         \ceva{c-tour} & c-tour & 2006 & frühes Mittelalter \\
         \ceva{c-booc} & c-booc & 2015 & Übergang Mittelalter$\rightarrow$Neuzeit\\         
         \bottomrule
    \end{tabular}
    \caption{Die kanonischen Schriften, chronologisch}
    \label{tab:kanon}
\end{table}

Die übrigen Jahr oder unvollständig gelten, manche sind spätere Apokryphen. Sie bieten aber doch mitunter brauchbare Hinweise; sie werden zu einzelnen Details hinzugezogen. Hierzu gehört  insbesondere die \cevain{pressemappe}~\cite{cbasepressemap} in unterschiedlichen Überlieferungsstufen und einem terminus ante quem 2007 sowie der rezente \cevain{coredump} \cite{cbasewebsite}. 

\ceva{c-booc} erwähnt ihm vorangegangene Publikationen, darunter vier weitere ihm ähnliche Ausgaben (\cevain{allmanach}), ein \cevain{analog-logbuch} und diverse Websites, unter anderem aus Arachischer Zeit \cite[S. 40-42 u. 60-63]{cbasebook}. Diese Quellen waren uns leider nicht zugänglich, weshalb sie hier keine Berücksichtigung finden.


