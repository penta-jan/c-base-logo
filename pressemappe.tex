\twofonts[\cite{cbasepressemap}]{
    Die c-base wird in der Zukunft als orbitales Generationsschiff auf der Erde gebaut und dient dem Terraforming anderer Planeten.
    
    Die Fortbewegung im Raum erfolgt mittels eines vom Cybernetischen-Quecksilber-Reaktor (CQR) erzeugten Feldes, das die c-base aus der Raumzeit ausschneidet, um sie über die Einstein-Rosenbaum-Brücke im Orbit des zu formenden Planeten materialisieren zu lassen. 
    
    Bei der Berechnung der Eingangsgrössen kam es zu einem Flip-Flop der Asimov-Konstante. Anstatt des Raumes wurde die Zeit gefaltet, die c-base reiste 4,5 Milliarden Jahre in die Vergangenheit statt vorwärts in den Raum und crashte aus noch nicht restlos geklärten Gründen auf die Erdoberfläche, wo sie langsam im märkischen Sand versank. 
    
    Da aber die Raumstation ein molekulares Erinnerungsmodul besitzt, das sowohl Wissen als auch Zusammensetzung von Materie speichert und wieder herstellen kann, konnte sich vor ca. 100.000 Jahren der Boardcomputer c-beam eigenständig einschalten. Einzelne Stationselemente manifestierten sich im Laufe der Zeit, so unter anderem die Antenna - landläufig als Berliner Fernsehturm bekannt. Die Arbeit des Vereins c-base wirkt zunehmend als Katalysator in diesem Prozeß.
    
    Die Rekonstruktion 
    
    Die c-base entwickelt sich so zu einem Wissenspool und Ideenbiotop mit einer Vielzahl von kreativen, wissenschaftlich-technischen und zukunftsorientierten     Menschen… 
    
    Die ersten Phase der Rekonstruktion fand in der Oranienburger Straße statt. Im Jahr 2000 wurde die Ausgrabungsstätte in der Rungestraße eröffnet. Momentan befindet sich die ca. 300 Mitglieder umfassende Crew in der
    Rekonstruktionsphase 4 (cbrp4) der Multimodulstation RS20.
    
    Die über 700 m 2 rekonstruierte Fläche bietet den Membern  diverse Labore, Experimentierflächen, mediale Aufnahme- und Bearbeitungsstationen, Forschungs- und Konstruktionseinheiten, Seminarraum, SciFi-Bibliothek und ein
    vielseitig einsetzbares Kulturdeck mit entsprechender Bühnentechnik.
    }
