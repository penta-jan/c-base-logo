\csection{cience}
% \section{cience \hspace{2ex} \raisebox{1pt}{{\fontspec{[ceva-c2.ttf]}(cience)}}}



\Hrule[fuenf]

\twofonts[\cite{cbasestarbasemanual}]{
    bedeutung kunstwort. der prozeß konzeption, umsetzung, produkt. auch forschung, aus-/fortbildung in wissenschaft, handwerk und technik

    cience modulor hier sind alle wesentlichen forschungs-und entwicklungseinrichtungen untergebracht. eine auswahl an einzelnen arbeitsbereichen und stationen: evolutionsdesign wahrnehmungsforschung (social art) soziochemie (geruchsstoffe) bio\-labs medienlabore 
    }
    
\twofonts[\cite{ctour}]{
    cience ist der forschungsring der c-base: hier liegen alle wesentlichen forschungs- und entwicclungseinrichtungen der station in verschiedensten arbeitsbereichen focussiert beieinander. 
    die ersten ideen und wahrheiten werden hier geschaffen.
        \begin{itemize}
            \item genlabs des evolutionsdesign
            \item die biolabs mit arboretum
            \item die soziochemie der socialartists
            \item die medialabs
        \end{itemize}
    cience meint den prozess der conzeption, umsetzung, production, forschung, aus- und fortbildung in wissenschaft, handwerk und technik
    }

Herauszustellen ist, dass \ceva{c-tour} hier \cevain{erschaffen} mit \cevain{Wahrheit} zusammenfügt. Es wird also eine Wahrheit geschaffen, nicht bloß entdeckt.

\begin{newstuff}
    \lettrine{Ü}{berwiegen} in den vorgelagerten \ring{Ringen} noch deutlich die traumhaften, inspirationellen, dialogischen und meditativen Aspekte, so verdichtet sich der autopoietische Prozess der Wiedererschaffung der Raumstation in \ceva{cience} durch rigorose Methodik und Präzision der Hingabe an definierte Ziele, die in den vorherigen Stufen erahnt und erträumt wurden.

    Entsprechend kommt es hier zu einer Prüfung der Möglichkeiten innerhalb der bereits realisierten Umgebung mit den bereits hervorgebrachten Materialien, Werkzeugen und mitarbeitenden Modulen. Allerdings ist \ceva{cience} nicht nur Forschung, sondern auch Anwendung von Erkenntnissen an der Grenzlinie zum Unerreichten. Hier werden also neue Module und Artefakte produziert, die dann neue Möglichkeiten erschließen. Der Beitrag von \ceva{cience} zum Wachstum der Station ist damit zentral.

    \twofonts[\cite{cbasebook},~S.~135]{auf dem forschungsring cience der c-base liegen alle wesentlichen entwicklungseinrichtungen der station in verschiedensten arbeitsbereichen focussiert beieinander. [...] der wissenschaftliche bereich der c-tation vereint forschung und bildung und meint den prozess der conzeption, umsetzung, production, aus- und fortbildung in wissenschaft, handwerc und technic vergangenheitsbewältigung und cucunfstforschung werden in den genlabs des evolutionsdesign, den biolabs rund um das arboretum, der soziochemie der socialartists und in den medialabs betrieben.}

    \ceva{cience} ist nicht auf digitale Forschung, aber auch nicht auf materielle Experimente beschränkt. Die Erforschung des Vorhandenen, also der bereits manifesten Teile der Station umfasst auch die Beschäftigung mit den Wirkmechanismen seiner Bewohnenden, also ihrer Erregungszustände, ihrer Emotionen, ihrer Vorstellungen, ihres Miteinanders (\ceva{focussiert}). Dabei stellt sich heraus, das permanente Infragestellung zwar gestattet und notwendig ist, aber gewonnene Erkenntnisse auch als Ecksteine des Fortschrittes dienen und angenommen sind.

    Es ist in einem autopoietischen System nicht möglich, zwischen Hervorbringendensystem und Ergebnissystem zu unterscheiden. Insofern ist der aktuelle Erkenntnisstand der Wissenschaft von der \ceva{c-base} identisch mit dem aktuellen Zustand der Station. Es gilt mithin die "`informatische Identität"'
    \begin{equation}
        \text{code}\;\widehat{=}\;\text{documentation}
    \end{equation}
    Allerings ist im üblichen Zeitablauf die Vergangenheit immer größer ist als die Gegenwart. Daher gilt intertemporal
    \begin{equation}
        \text{code} < \text{documentation}
    \end{equation}
    was soviel bedeutet, wie: die Ansammlung vergangener Fakten ist immer größer als die Menge der aktuellen Fakten, und die Datensumme von Backups und Dokumentation ist immer größer als der aktuell ausgeführte Code. Entsprechend ist die Vergangenheit der Raumstation größer als ihre Gegenwart. Und dennoch ist von jedem gegenwärtigen Zeitpunkt aus die Zukunft unendlich größer als beide~\cite{encyclopaedia}.

    \twofonts[\par\hfill"`rabbinische weisheit"', \cite{cbasebook},~S.~47]{die zucunft hat eine lange vergangenheit}

    \ceva{cience} besteht aus dem Prozess des Forschens, ist aber zugleich die Summe bzw. die Potenz oder Fakultät der Erkenntnisse. Diese sind in der Raumstation verteilt und vernetzt abgelegt, und zwar sowohl virtuell als auch materiell. 
    
    Die Multisprachlichkeit der Raumstation, ihre nicht immer verständliche Datensystematik, die temporal und spatial unterschiedliche Erreichbarkeit einzelner Informationsbrocken und der  Informationsträger ist zugleich Objekt und Subjekt von \ceva{cience}.

    Insofern \ceva{cience} ein erzeugendes Subsystem der Station ist, kommt es hier besonders zur Erschaffung neuer Artefakte nach den durch \ceva{core} inspirierten Bauplänen. Jedes so emanierende Objekt schafft Faktizität und damit in sich selbst kaum widerlegbare Wahrheit. Ergo gilt: \cevain{wer baut hat recht}~\cite[S.~47]{cbasebook}.
\end{newstuff}