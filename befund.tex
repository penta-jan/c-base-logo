\section*{Befund}\addcontentsline{toc}{section}{Befund}
    
    \cite{ctour}, \cite{cbasepressemap} und \cite{cbasebook} stimmen im Befund und teils wörtlich überein. In \cite{ctour}, steht: 
    
    \twofonts{1995 wurden unter Berlin-Mitte die Überreste einer 4,5 Milliarden Jahre alten Raumstation entdeckt. Erste Forschungen ergaben, daß sich die c-för\-mi\-ge Raumstation mit ihrem Mittelpunkt unter dem heutigen Alexanderplatz befinden muß und aus 7 Ringen besteht. Aufgrund eines Fundstückes mit der Aufschrift "`\emph{c-base - be future compatible}"' und in Anlehnung an die Anzahl der Ringe, legte das anfänglich nur aus wenigen Mitgliedern bestehene Rekonstructionsteam den Projektnamen und die Aufteilung in sieben Arbeitsbereiche fest.}

    Außerdem:
    
    \twofonts{Die c-base ist eine abgestürzte raumstation. das unter berlin-mitte im märkischen sand versunkene artefact wird seit 1995 von über 100 zukunftsbegeisterten experten reconstruiert. Das raumschiff besteht aus sieben ineinander geschalteten c-förmigen ringen. jeder ring ist für ganz spezifische aufgabencluster modular ausgelegt.}
    
    
    Die Rekonstruktion zeigt eine fragmentierte Struktur von konzentrisch verschachtelten Ringen mit multiplen, untereinander verschiebbaren Modulen. 
    Die konzentrische Anordnung in der Architektur der Station zeigt \cref{fig:siebenringe}.

\begin{figure}[ht!]
    \centering
        \resizebox{0.6\textwidth}{!}{
        \documentclass{standalone}
\usepackage{tikz}
\usepackage{calc}
\usepackage{xcolor}
\begin{document}
%% c-base logo nachgebaut von penta.
%% alles nur geschätze Winkel und Abstände :/
%% um den code zu verstehen, einfach mal einzelne Teile auskommentieren und wieder einkommentieren (ctrl-#) und dann mal \draw[white] durch \draw[red] ersetzen, dann sieht man, was was ist.
%% viel Spaß damit.
\tikzset{
  pics/carc/.style args={#1:#2:#3}{
    code={
      \draw[pic actions] (#1:#3) arc(#1:#2:#3);
    }
  }
}
\definecolor{eins}{HTML}{FFFFFF}   %% Ring 1 "core"     - weiß - Mittelpunkt 
\definecolor{zwei}{HTML}{FF0000}   %% Ring 2 "com"      - rot -  Ring um Mittelpunkt
\definecolor{drei}{HTML}{FF7F00}   %% Ring 3  "culture" - orange - "Fenster" innen
\definecolor{vier}{HTML}{AAFF00}   %% Ring 4  "creativ" - grün -  fünf Module, quasi invertiert
\definecolor{fuenf}{HTML}{00FFFF}  %% Ring 5 "cience"  - blau - vier Module
\definecolor{sechs}{HTML}{800080}  %% Ring 6 "carbon" -  violett - drei Module mit "Strich"
\definecolor{sieben}{HTML}{4B0082} %% Ring 7 "clamp" -  indigo - viele "Fenster" außen

    \begin{tikzpicture}%
        \def\radi{10}%     

        %% äußerer Radius
        \filldraw[black] (0:0) circle (\radi-0.3);
        % was wegnehmen / weiß übermalen
        \draw[white, line width=1.6cm] (0:0) pic{carc=-55:90:\radi-0.72}; 

        %% clamp - viele kleine Fenster
        \foreach [count=\i] \ii in {%
        70,74,78,100,
        114,118,...,170,
        182,186,
        202,206,...,250,
        266,270,...,304}
            \draw[sieben, line width=0.5cm] (0:0) pic{carc=\ii:\ii-2:\radi-2}; 
        
        \draw[white, line width=3.05cm] (0:0) pic{carc=50:-55:\radi-3};
        
        \draw[white, line width=2cm] (0:0) pic{carc=-90:0:\radi-3.5};

  
        \draw[sechs, line width=1.8cm] (0:0) 
            pic{carc=82:135:\radi-3.5};
        \draw[sechs, line width=1.8cm] (0:0) 
            pic{carc=65:74:\radi-3.5};
        \draw[sechs, line width=1.8cm] (0:0) 
            pic{carc=53:62:\radi-3.5};
            
        \draw[black, line width=0.4cm] (0:0) 
            pic{carc=110:135:\radi-3.3};

        \foreach [count=\i] \ii in {1,2,3,4}
            \draw[fuenf, line width=1.5cm] (0:0) 
            pic{carc=90*\i:90*\i+45:\radi-5.5};


        \filldraw[vier] (0:0) circle (\radi-6.5);
        
        \foreach [count=\i] \ii in {1,2,3,4,5}
            \draw[black, line width=0.7cm] (0:0) 
            pic{carc=72*\i+27:72*\i+36+27:\radi-7};

        \filldraw[black] (0:0) circle (\radi-7.5);

        \foreach [count=\i] \ii in {160,172,...,490}
            \draw[drei, line width=0.4cm] (0:0) pic{carc=\ii:\ii-8:\radi-8};   
        
        \filldraw[zwei] (0:0) circle (\radi-8.5);
        \filldraw[black] (0:0) circle (\radi-8.8);
        \filldraw[eins] (0:0) circle (\radi-9.8);
     
    \end{tikzpicture}
\end{document}

    }
    \caption{Die 7 Ringe (Approximation)}
    \label{fig:siebenringe}
\end{figure}


    Die  Ringe werden in allen vorliegenden Texten von innen nach außen gezählt, bezeichnet und  farblich codiert  wie in  \cref{tab:ringe} aufgeführt. Wir verwenden die Werte aus \cite{cbasefarbschema}, das die älteste und zugleich genaueste Quelle zu sein scheint.\footnote{\cevain{jawohl mein designer} \cite[S. 47]{cbasebook}} 

    \begin{table}[ht!]
        \centering
        \begin{tabular}{rlllrr}
            \toprule
                1 & \ceva{core} & core & grau / weiß & \texttt{e7e7e8} & \Hrulek[eins]  \\
                2 & \ceva{com} & com & rot & \texttt{ed1c24} & \Hrulek[zwei] \\
                3 & \ceva{culture} & culture & orange & \texttt{fbad18} & \Hrulek[drei] \\
                4 & \ceva{creactive} & creactive & grün & \texttt{75c043}& \Hrulek[vier]  \\
                5 & \ceva{cience} & cience & cyan & \texttt{0089d0}& \Hrulek[fuenf]  \\
                6 & \ceva{carbon} & carbon & indigo & \texttt{0089d0}& \Hrulek[sechs]  \\
                7 & \ceva{clamp} & clamp  & ultraviolett / schwarz & \texttt{000000}& \Hrulek[sieben] \\
            \bottomrule
        \end{tabular}
        \caption{Nummern, Bezeichnung und Farbe der Ringe}
        \label{tab:ringe}
    \end{table}

    Die Ringfarbe korrespondiert nur schwach mit den Signalfarben innerhalb der Station, die wir hier der Vollständigkeit  halber in \cref{tab:bedeutungen} zeigen (nach \cite[S. 56]{cbasebook}).
    
    \begin{table}[ht!]
        \centering
        \begin{tabular}{lll}
            \toprule
                grau / weiß  & \Hrulek[eins] & überlebensstandard gesichert, temperature, drucc \\
                rot & \Hrulek[zwei] & lebendig, alarm, communication \\
                 orange  & \Hrulek[drei] & schädlich, aktiver prozess auf atomarer ebene \\
                grün & \Hrulek[vier]  & nicht humane biologische substanz, prozess \\
                cyan & \Hrulek[fuenf] & lowered thermal conditions \\
                indigo & \Hrulek[sechs] & \textit{(not assigned)}  \\
                schwarz & \Hrulek[sieben] & vacuum, death, hazard\\
            \bottomrule
        \end{tabular}
        \caption{Signalbedeutungen der Farben}
        \label{tab:bedeutungen}
    \end{table}

    Diese sieben konzentrischen Ringe werden im Folgenden einzeln genauer beschrieben. Dabei liegt der Fokus dieser Arbeit auf der allgemeinen Funktion innerhalb des autopoietischen Systems \ceva{c-base}; auf eine detaillierte Beschreibung einzelner Module, Artefakte und Bewohner wird verzichtet. Dem interessierten Leser sei dazu insbesondere \cite{cbasebook} nahegelegt. Aufgrund der multimodularen und letztlich höherdimensionalen Struktur der Station ist eine eineindeutige und ausschließliche Zuordnung einzelner Module zu bestimmten Ringen ohnehin fragwürdig.
    
    % Ähnlichkeiten mit lebenden Personen sind daher rein zufällig.
% \vfill