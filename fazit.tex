\section*{Zur Geometrie}\addcontentsline{toc}{section}{Zur Geometrie}\label{sec:fazit}
\fancyhead[RO]{Zur Geometrie}
\fancyhead[LO]{}

Unser Rundgang durch die \ring{Ringe} hat gezeigt, dass die Funktion jedes einzelnen \ring{Rings} nur aus dem komplexen Zusammenspiel mit den anderen  verstanden werden kann. Daher ist das Komplexitätsmaß auf dieser Ebene zumindest
\begin{equation}
    T(n) = \mathcal{O} (n!) > 7! = 5040
\end{equation}
- und das ungeachtet des Umstandes, dass jeder Ring aus diversen Modulen und diese aus Aggregaten als jeweiligen Subsystemen weiter ausdifferenziert. Wir haben es also mit einem hochkomplexen, selbstreferentiellen und autopoietischen System zu tun, das insgesamt mehr Zustände annehmen kann, als sich im bekannten Universum abbilden ließen. Damit ist das System insgesamt ein mögliches Spiegelbild des Gesamtrestuniversums und bringt selbst in seiner Autopoiesisnendlich viele Welten hervor.


Die aktuelle Form ist durch den Absturz der Raumstation mitverursacht und vermutlich ist die gewissermaßen "`flache"' Struktur, die wir bei den bisherigen Rekunstruktionsphasen ausgemacht haben, eine zweidimensionale Projektion der eigentlich mehrdimensionalen Raumstation. Darauf gehen wir hier etwas genauer ein.

Wir haben gesehen, dass sich \ceva{clamp} und \ceva{core} berühren bzw. über eine Einstein-Rosen-Brücke verbunden sind. Daraus folgern wir eine (ursprünglich, zukünftig) ringförmige Anordnung der Ringe; vgl. \cref{fig:ringring}

\begin{figure}[ht!]
    \centering
    \documentclass{standalone}
\usepackage{tikz}
\usetikzlibrary{decorations, decorations.text}
\usepackage{calc}
\usepackage{xcolor}
\usepackage{fontspec}
\newcommand{\ceva}[1]{~{\fontspec{[ceva-c2.ttf]}#1}}
\definecolor{eins}{HTML}{e7e7e8}   %% Ring 1 "core"     - weiß - Mittelpunkt, Ring um Mittelpunkt
\definecolor{zwei}{HTML}{ed1c24}   %% Ring 2 "com"      - rot -  "Fenster" innen
\definecolor{drei}{HTML}{fbad18}   %% Ring 3  "culture" - orange - fünf Module, quasi invertiert
\definecolor{vier}{HTML}{74c043}   %% Ring 4  "creactiv" - grün -  vier Module
\definecolor{fuenf}{HTML}{0089d0}  %% Ring 5 "cience"  - cyan (blau) - drei Module mit "Strich"
\definecolor{sechs}{HTML}{11357e}  %% Ring 6 "carbon" -  indigo - viele "Fenster" außen
\definecolor{sieben}{HTML}{000000} %% Ring 7 "clamp" -  schwarz, c-förmig
\definecolor{cbase}{HTML}{222222}  %% Körper der Raumstation    
\begin{document}
%% c-base logo nachgebaut von penta.
%% alles nur geschätze Winkel und Abstände :/
%% um den code zu verstehen, einfach mal einzelne Teile auskommentieren und wieder einkommentieren (ctrl-#) und dann mal \draw[white] durch \draw[red] ersetzen, dann sieht man, was was ist.
%% viel Spaß damit.
\tikzset{
  pics/carc/.style args={#1:#2:#3:#4}{
    code={
      \draw[postaction={decorate, decoration={text along path, raise=-2pt, text align={align=center}, text={\ceva{#4}}, reverse path}}] (#1:#3) arc(#1:#2:#3);
    }
  }
}%
    \begin{tikzpicture}
        \draw[gray, line width=50pt] (0:0) circle (3);
        \foreach [count=\i] \ring/\color in
            {core/eins,com/zwei,culture/drei,creactiv/vier,cience/fuenf,carbon/sechs,clamp/sieben}
            {%
                \draw[color=\color!50,line width=45pt] (0:0) pic{carc=\i*51.418-25:\i*51.418+25:3:\ring};
            }%
    \end{tikzpicture}
\end{document}

    \caption{Ringförmige Anordnung der \ring{Ringe}}
    \label{fig:ringring}
\end{figure}

Eine \emph{ringförmige Anordnung von Ringen }ist geometrisch darstellbar auf einer Oberfläche, die entsteht, wenn ein Kreis um einen Kreis (also quasi ein \ring{Ring} um einen \ring{Ring}) rotiert. Diese Figur ist der Rotationstorus; vgl. \cref{fig:torusweiss}

\begin{figure}[ht!]
    \centering
    \documentclass{standalone}
\usepackage{xcolor}
\usepackage[svgnames]{pstricks}
\usepackage{pst-solides3d}
\definecolor{eins}{HTML}{e7e7e8}   %% Ring 1 "core"     - weiß - Mittelpunkt, Ring um Mittelpunkt
\definecolor{zwei}{HTML}{ed1c24}   %% Ring 2 "com"      - rot -  "Fenster" innen
\definecolor{drei}{HTML}{fbad18}   %% Ring 3  "culture" - orange - fünf Module, quasi invertiert
\definecolor{vier}{HTML}{74c043}   %% Ring 4  "creactiv" - grün -  vier Module
\definecolor{fuenf}{HTML}{0089d0}  %% Ring 5 "cience"  - cyan (blau) - drei Module mit "Strich"
\definecolor{sechs}{HTML}{11357e}  %% Ring 6 "carbon" -  indigo - viele "Fenster" außen
\definecolor{sieben}{HTML}{000000} %% Ring 7 "clamp" -  schwarz, c-förmig
\definecolor{cbase}{HTML}{222222}  %% Körper der Raumstation    
\begin{document}

%% https://ftp.tu-chemnitz.de/pub/tex/graphics/pstricks/contrib/pst-solides3d/doc/pst-solides3d-doc.pdf

\begin{pspicture}(-4,-4)(4,4)
    \psset{Decran=20,viewpoint=5 11 25}
    \pstVerb{/iface 0 store}%
    \psSolid[
        r1=3,r0=2,
        object=tore,
        ngrid=14 14,
        RotY=30]
\end{pspicture}


\end{document}
    \caption{Ein Torus mit 14 (begradigten) Meridianen  und 14 Parallelkreisen}
    \label{fig:torusweiss}
\end{figure}

Nun stellt sich die Frage, wie die \ring{Ringe} auf so einem Torus angeordnet waren bzw. sein werden bzw. sollen. 

Grundsätzlich sind die \ring{Ringe} Kreisscharen. Es gibt drei verschiedene Gruppen von Kreisscharen auf einem Torus: Parallelkreise, Merididiane und Villarceau-Kreise. 

Parallelkreise würden etwa entstehen durch Schnitte eines gefärbten Torus wie z.B. in \cref{fig:torus-parallele} abgebildet. 

\begin{figure}[ht!]
    \centering
    \documentclass{standalone}
\usepackage{xcolor}
\usepackage[svgnames]{pstricks}
\usepackage{pst-solides3d}
\definecolor{eins}{HTML}{e7e7e8}   %% Ring 1 "core"     - weiß - Mittelpunkt, Ring um Mittelpunkt
\definecolor{zwei}{HTML}{ed1c24}   %% Ring 2 "com"      - rot -  "Fenster" innen
\definecolor{drei}{HTML}{fbad18}   %% Ring 3  "culture" - orange - fünf Module, quasi invertiert
\definecolor{vier}{HTML}{74c043}   %% Ring 4  "creactiv" - grün -  vier Module
\definecolor{fuenf}{HTML}{0089d0}  %% Ring 5 "cience"  - cyan (blau) - drei Module mit "Strich"
\definecolor{sechs}{HTML}{11357e}  %% Ring 6 "carbon" -  indigo - viele "Fenster" außen
\definecolor{sieben}{HTML}{000000} %% Ring 7 "clamp" -  schwarz, c-förmig
\definecolor{cbase}{HTML}{222222}  %% Körper der Raumstation    
\begin{document}

%% https://ftp.tu-chemnitz.de/pub/tex/graphics/pstricks/contrib/pst-solides3d/doc/pst-solides3d-doc.pdf

\begin{pspicture}(-4,-4)(4,4)
    \psset{Decran=20,viewpoint=5 11 25}
    \pstVerb{/iface 0 store}%
    \psSolid[
        hue = 0 1,
        % fcol=48 {iface (green)
        % iface 1 add (orange)
        % iface 2 add (orange)
        % iface 3 add (red) 
        % iface 4 add (red) 
        % iface 5 add (white) 
        % iface 6 add (white) 
        % iface 7 add (black) 
        % iface 8 add (black) 
        % iface 9 add (blue) 
        % iface 10 add (blue) 
        % iface 11 add (blue) 
        % iface 12 add (blue) 
        % iface 13 add (blue) /iface
        % iface 14 add store} repeat,
        r1=3,r0=2,
        object=tore,
        ngrid=14 14,
        RotY=30]
\end{pspicture}


\end{document}
    \caption{Parallelkreise}
    \label{fig:torus-parallele}
\end{figure}

Eine solche Anordnung ist zwar möglich, aber wenig plausibel, da die einzelnen Ringe in diesem Fall eher als Schreiben abgebildet werden würden. Auch ist nicht zu erkennen, wie es beim Absturz der Station dann zu einer konzentrischen Anordnung gekommen sein sollte. 

Eine Anodrnung der \ring{Ringe} als Meridiankreise  zeigt \cref{fig:torus-meridiane}.  
Dabei berühren sich hier im Inneren eben \ceva{core} und \ceva{clamp}; das passt zu unserer Interpretation des Kanons von der Geometrie der Station als Torus mit einem innenliegenden Wurmloch. In diesem berühren sich innen \ceva{core} und \ceva{clamp}; außen liegt \ceva{creactiv}. 

\begin{figure}[ht!]
    \centering
        \documentclass{standalone}
\usepackage{xcolor}
\usepackage[svgnames]{pstricks}
\usepackage{pst-solides3d}
\definecolor{eins}{HTML}{e7e7e8}   %% Ring 1 "core"     - weiß - Mittelpunkt, Ring um Mittelpunkt
\definecolor{zwei}{HTML}{ed1c24}   %% Ring 2 "com"      - rot -  "Fenster" innen
\definecolor{drei}{HTML}{fbad18}   %% Ring 3  "culture" - orange - fünf Module, quasi invertiert
\definecolor{vier}{HTML}{74c043}   %% Ring 4  "creactiv" - grün -  vier Module
\definecolor{fuenf}{HTML}{0089d0}  %% Ring 5 "cience"  - cyan (blau) - drei Module mit "Strich"
\definecolor{sechs}{HTML}{11357e}  %% Ring 6 "carbon" -  indigo - viele "Fenster" außen
\definecolor{sieben}{HTML}{000000} %% Ring 7 "clamp" -  schwarz, c-förmig
\definecolor{cbase}{HTML}{222222}  %% Körper der Raumstation    
\begin{document}

%% https://ftp.tu-chemnitz.de/pub/tex/graphics/pstricks/contrib/pst-solides3d/doc/pst-solides3d-doc.pdf

\begin{pspicture}(-4,-4)(4,4)
    \psset{Decran=20,viewpoint=5 11 25}
    \pstVerb{/iface 0 store}%
    \psSolid[
        fcol=48 {iface (green)
        iface 1 add (orange)
        iface 2 add (orange)
        iface 3 add (red) 
        iface 4 add (red) 
        iface 5 add (white) 
        iface 6 add (white) 
        iface 7 add (black) 
        iface 8 add (black) 
        iface 9 add (blue) 
        iface 10 add (blue) 
        iface 11 add (blue) 
        iface 12 add (blue) 
        iface 13 add (blue) /iface
        iface 14 add store} repeat,
        r1=3,r0=2,
        object=tore,
        ngrid=14 14,
        RotY=30]
\end{pspicture}


\end{document}
    \caption{Die \ring{Ringe} als Meridiankreise auf dem Torus}
    \label{fig:torus-meridiane}
\end{figure}


Da der äußerere Äquator (Meridian) \ceva{creactive} enspricht, bedeutet eine Ausweitung der \ceva{creactivität} ein Anschwellen dieses Torus und somit Wachstum der Station; bildlich entspräche das in etwa der Inflation eines Rettungsrings. Dabei ist zu beachten, dass eine Vergrößerung des rotierenden Kreises ohne gleichzeitige Ausweitung des Rototaionsradius im Torus zu einem Verschwinden des innenliegenden Loches führen könnte. Dann wäre die Station gewissermaßen an ihrer eigenen \ceva{creactivität} erstickt.

Betrachten wir nun die vielleicht interessanteste mögliche Anordnung von Kreisscharen auf einem Torso, nämlich die so genannten Villarceau-Kreise. Sie entstehen geometrisch (paarweise) durch den Schnitt einer deoppelberührenden Ebene mit dem Torso (\cref{fig:villarceaukreise}).

\begin{figure}[ht!]
    \centering
    \includesvg{Torus-vill-point.svg}
    \caption{Villarceau-Kreise \cite{villarceauag2gaeh}}
    \label{fig:villarceaukreise}
\end{figure}

Es lässt sich also eine Schar von parallelen Kreisen auf einem Torus finden, die alle perfekt kreisförmig und zudem kongruent sind. \cref{fig:villarceautorous} zeigt eine Annäherung, indem hier ein Torus mit $21\times 21$ Flächen belegt wurde.

\begin{figure}[ht!]\label{fig:villarceautorous}
    \centering
            \documentclass{standalone}
\usepackage{xcolor}
\usepackage[svgnames]{pstricks}
\usepackage{pst-solides3d}
\definecolor{eins}{HTML}{e7e7e8}   %% Ring 1 "core"     - weiß - Mittelpunkt, Ring um Mittelpunkt
\definecolor{zwei}{HTML}{ed1c24}   %% Ring 2 "com"      - rot -  "Fenster" innen
\definecolor{drei}{HTML}{fbad18}   %% Ring 3  "culture" - orange - fünf Module, quasi invertiert
\definecolor{vier}{HTML}{74c043}   %% Ring 4  "creactiv" - grün -  vier Module
\definecolor{fuenf}{HTML}{0089d0}  %% Ring 5 "cience"  - cyan (blau) - drei Module mit "Strich"
\definecolor{sechs}{HTML}{11357e}  %% Ring 6 "carbon" -  indigo - viele "Fenster" außen
\definecolor{sieben}{HTML}{000000} %% Ring 7 "clamp" -  schwarz, c-förmig
\definecolor{cbase}{HTML}{222222}  %% Körper der Raumstation    
\begin{document}

%% https://ftp.tu-chemnitz.de/pub/tex/graphics/pstricks/contrib/pst-solides3d/doc/pst-solides3d-doc.pdf

\begin{pspicture}(-4,-4)(4,4)
    \psset{Decran=20,viewpoint=5 11 25}
    \pstVerb{/iface 0 store}%
    \psSolid[
        fcol=48 {iface (green)
        iface 1 add (orange)
        iface 2 add (orange)
        iface 3 add (red) 
        iface 4 add (red) 
        iface 5 add (white) 
        iface 6 add (white) 
        iface 7 add (black) 
        iface 8 add (black) 
        iface 9 add (blue) 
        iface 10 add (blue) 
        iface 11 add (SkyBlue) 
        iface 12 add (SkyBlue) 
        iface 13 add (green) /iface
        iface 14 add store} repeat,
        r1=3,r0=2,
        object=tore,
        ngrid=27 14,
        RotY=30]
\end{pspicture}

\end{document}            
    \caption{Die \ceva{Ringe} als Villarceau-Kreise}
    \label{fig:villarceautorous}
\end{figure}


Nach \cref{fig:villarceautorous} wären die \ring{Ringe} auf einem Torus in Form verschlungener Bänder angeordnet. Sie winden sich um das Zentrum des Torus und zugleich um den Körper des Torus selbst. 
Es gibt keinen \ring{Ring}, der einen bevorzugten Ort einnimmt. Da sie kongruent sind, sind sie flächen- und längengleich.

Ob der Bedeutung toroidaler Geometrie für Fusionsreaktoren vermuten wir, dass wir hier der Geometrie des \cevain{Möbius-band-accelerators} auf der Spur sind (vgl.~\cref{sec:core}). 

Die weitere Erforschung dieser Geometrie und die möglichen Bedeutungen für die energetische Ausbeute des verschlungenen Miteinanders der sich ergänzenden \ring{Ringe} bleibt unser \cevain{facit}, also \emph{das zu tuende}. 

% Villarceau-Kreise entstehen bekanntlich durch Schntite von Doppelberührenden mit dem Torus paarweise entstehen. Bislang sind alle Forschungen allerdings von sieben (!) \ring{Ringen} der \ceva{c-base} ausgegangen. Sollten die Ringe ursprünglich und damit auch zukünftig Villarceau-Kreise sein, so muss es zu jedem \ring{Ring} einen \cevain{Anti-Ring} geben. Es könnte dies durch eine Umkehrung der Chiralität erfolgen.


% Und das ist ein Bild schön genug, dieses Papier zu beschließen.

% \begin{center}
%     \ceva{-- be future compatible --}    
    
%     -- be future compatible --
% \end{center}