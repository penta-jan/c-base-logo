\section{Zwischenergebnis}%\addcontentsline{toc}{section}{Zwischenergebnis}\label{sec:fazit}
\fancyhead[RO]{Zwischenergebnis}
% \fancyhead[LO]{}

Unser Rundgang durch die \ring{ringe} hat gezeigt, dass die Funktion jedes einzelnen \ring{rings} nur aus dem komplexen Zusammenspiel mit den anderen  verstanden werden kann. Daher ist das Komplexitätsmaß auf dieser Ebene zumindest
\begin{equation}
    T(n) = \mathcal{O} (n!) > 7! = 5040
\end{equation}
- und das ungeachtet des Umstandes, dass jeder \ring{ring} aus diversen Modulen und diese aus Aggregaten als jeweiligen Subsystemen besteht und sich so weiter ausdifferenziert. Wir haben es also mit einem hochkomplexen, selbstreferentiellen und autopoietischen System zu tun, das insgesamt mehr Zustände annehmen kann, als sich im bekannten Universum abbilden ließen. Damit ist das System insgesamt ein mögliches Spiegelbild des Restuniversums und bringt selbst in seiner Autopoiesisnendlich viele Welten hervor.

Die Exegese hat unter anderem ergeben, dass 
\begin{enumerate}
    \item die \ring{ringe} Funktionen eher als topologische Orte bezeichnen;
    \item die \ring{ringe} aus \ceva{Modulen} bestehen, die auch zwischen den \ring{ringen} beweglich sind,
    \item \ceva{creactive} eine Mittel- bzw. Äquatorialstellung einnimmt und zugleich besonders \cevain{powerful} ist;
    \item \ceva{core} und \ceva{clamp} über eine Einstein-Rosen-Brücke verbunden sind.
\end{enumerate}

Daraus lassen sich einige Überlegungen zur primären, sekundären und finalen Topologie der Station ableiten.

