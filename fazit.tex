\section{Zwischenergebnis}%\addcontentsline{toc}{section}{Zwischenergebnis}\label{sec:fazit}
\fancyhead[RO]{Zwischenergebnis}
% \fancyhead[LO]{}

Unser Rundgang durch die \ring{ringe} hat gezeigt, dass die Funktion jedes einzelnen \ring{rings} nur aus dem komplexen Zusammenspiel mit den anderen  verstanden werden kann. 

% Sie bestehen aus Modulen, die ihrerseits auf \ceva{decc}s angeordnet sind.

Die \cevain{c-tation} funktioniert insgesamt als Computer in einem Selbst\-fin\-dungs-, Selbst\-er\-fin\-dungs- und Rekonstruktionsprozess (Autopoiesis); sie erschafft sich durch die von ihr berufene \cevain{crew} selbst, durchaus auch durch Versuch und Irrtum, hin auf das Ziel das zu werden, was sie war und was sie sein wird. 

% Daher ist das Komplexitätsmaß auf dieser Ebene zumindest
% \begin{equation}
%     T(n) = \mathcal{O} (n!) > 7! = 5040
% \end{equation}
% - und das ungeachtet des Umstandes, dass jeder \ring{ring} aus diversen Modulen und diese aus Aggregaten als jeweiligen Subsystemen besteht und sich so weiter ausdifferenziert. 

% Wir haben es  mit einem hochkomplexen, selbstreferentiellen und autopoietischen System zu tun, das insgesamt mehr Zustände annehmen kann, als sich im bekannten Universum abbilden ließen. Damit ist das System insgesamt ein mögliches Spiegelbild des Restuniversums und bringt selbst in seiner Autopoiesis unendlich viele Welten hervor.

Die Exegese der Aussagen der kanonischen Schriften über die einzelnen \ceva{ringe} hat darüber hinaus unter anderem ergeben, dass 
\begin{enumerate}
    \item die \ring{ringe} Funktionen eher als topologische Orte bezeichnen;
    \item die \ring{ringe} aus \ceva{Modulen} bestehen, die  zwischen den \ring{ringen} beweglich sind,
    \item \ceva{creactive} eine Mittel- bzw. Äquatorialstellung einnimmt und zugleich besonders \cevain{powerful} ist;
    \item \ceva{core} und \ceva{clamp} über eine Einstein-Rosen-Brücke verbunden sind.
\end{enumerate}

Daraus lassen sich einige Überlegungen zur urpsprünglichen, aktuellen und finalen Topologie der Station ableiten.

