\csection{clamp}
% \section{clamp \hspace{2ex} \raisebox{1pt}{{\fontspec{[ceva-c2.ttf]}(clamp)}}}

\Hrule[sieben]

\twofonts[\cite{cbasestarbasemanual}]{
    bedeutung entwicklung, neugier, halboffene, noch nicht vollendete form zum kreis. - der buchstaben c

    outer construction c-förmiges aussenskelett der raumschifkonstruktion, kurskorrektureinheiten, triebwerke sensoraray 
    }

Hier wird bereits von \cevain{kurskorrektureinheiten} gesprochen; später heißt es dann \ceva{coursecorrectursupporter}. Leider sind uns diese Aggregate bzw. die Kenntnis über sie verloren gegangen, und es kommt heute nur noch sehr selten zu bedeutenden Kurskorrekturen.

\twofonts[\cite{ctour}]{
    der größte ring umschließt als stützconstruction alle anderen ringe der raumstation.
        \begin{itemize}
            \item antennen- und sensoreneinheiten
            \item coursecorrectursupporter
            \item siri-sonden-doccs
            \item das ectonet
        \end{itemize}
    clamp ist der buchstabe c: er steht für die halboffene, noch nicht vollendete form zum creis und meint neugier, aufmerksamkeit und entwicklung
    }

Beiden Quellen ist eine gewisse Knappheit zueigen; sie stellen aber beide heraus, dass zumindest dieser \ring{Ring} \cevain{c-förmig}, also ungeschlossen, ist.

    \begin{newstuff}
        \lettrine{J}{edes} autopoietische System braucht eine zusammenhangstiftende Außenhülle als  Abgrenzung gegenüber den Umgebungssystemen. Diese Hülle dient dem Schutz der inneren Selbständigkeit, des inneren Schutzraumes selbst und ist damit unbedingt aufrecht zu halten. Nur durch eindeutige Schranken ist die freie Entfaltung innerhalb dieser Grenzen und schließlich auch das friedfertige Ausweiten dieser Grenzen möglich.

        \twofonts[\cite{cbasebook},~S.~189]{der äußerste ring clamp formt den buchstaben c: er steht für die halboffene, noch nicht vollendete form zum creis und mein neugier, aufmercsamceit und entwicclung. errepräsentiert den c-base-cosmos und gibt eine übersicht der activitäten und möglichkeiten der c-tation. der größte ring umschließt als stützconstruction alle anderen ringer der raumstation und beherbergt das ectonet sowie curscorrectursupporter, die siri-sondendoccs und periphäre antennen- und sensoreinheiten.}

        \ceva{clamp} schließt die Station nach außen ab, ohne jedoch eine Ausgrenzungsbarriere zu sein. Sie ist aber auch die Schwelle, an welcher sich die Raumstation gegenüber ihrer Umgebung befindet, und der letztlich unüberbrückbare Gegensatz zwischen innerem Bewusstsein der Raumstation und ihrer Außenwelt. 
        
        Zugleich ist \ceva{clamp} die \cevain{chnittstelle} vom Äußeren zum Inneren, also die Sensorhülle des Gesamtsystems. Fraglos bildet sich an dieser Fläche das Außenbild der Umwelt von der Station, aber auch das Innenbild der Station von der Außenwelt. Hier kommt es zu Anfragen, zu Beeinflussungen, zum Empfang von Angeboten, zum Aussenden von Hilferufen und vergleichbaren Sprechakten. Nicht zuletzt ist \ceva{clamp} Ort des Übergangs von Außen und Innen und damit auch der Ort der Assimilation weiterer Module in das autopoietische Projekt \ceva{c-base}.
        
        \ceva{clamp} ist zugleich ein Festhalten am Bestehenden und durch Rückfaltung auf \ceva{com} zugleich Anschlusspunkt für die Penetration durch Neues innerhalb von \ceva{creativ}. Die Station ist ein werdendes Wesen, und daher sowohl neugierig und offen als auch schüchtern und schutzbedürfig. Es sucht in seiner Umgebung nach Orientierung. Zeitgleich ist die \ceva{c-base} unfassbar alt und weise, auch wenn sie sich dessen nur teilbewusst ist.
        
        \ceva{clamp} ist sowohl die Verankerung der Station in der Restrealität als auch die Sollbruchstelle beim möglichen künftigen Abflug. Aufgrund seiner Wichtigkeit für die Integrität der Stationswerdung ist dieser \ring{Ring} besonders mit dem \ceva{Core} verbunden, sodass sich die Station hier zirkulär auf ihren Ursprung zurückfaltet. Damit bildet \ceva{clamp} den Ereignishorziont des mit dem Weißen Loch in \ceva{core} korrespondierenden Schwarzen Loches und somit seinen Gegenpol in der Raumzeit. Das kann auch erklären, wieso längst weggeworfene Gegenstände oder entfernte Kohlenstoffeinheiten sich mitunter erneut manifestieren.
        
        Die entstehenden Energieflüsse zwischen den \ring{Ringen} und über die \ring{Ringe} verstärken sich unter Umständen gegenseitig und erzeugen dann parallele, aber unterschiedliche Wirklichkeiten. Insgesamt kommt es dabei des öfteren zu deutlichen Raumzeitverzerrungen, die sehr unterschiedliche Wahrnehmung der relativen Magnitude von Ereignissen durch unterschiedliche Betrachter zur Folge haben.
    \end{newstuff}