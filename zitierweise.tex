\section*{Zitierweise}

Die Primärquellen verwenden desöfteren  eine eigene Orthographie namens  \cevain{clang}; dabei wird auf Kapitalisierung verzichtet,  Sibilanten und der stimmlose velare Plosiv (\emph{k}) werden durch den Buchstaben \cevain{c} ersetzt, usw; Ähnliches gilt für daraus gebildete Konsonantencluster \cite[S.~46]{cbasebook}  \cite{clang}. Die Originalschreibung (mit \ceva{clang}) wurde beibehalten; nur offensichtliche Tippfehler wurden korrigiert. In unseren eigenen Texten verzichten wir darauf.

Die \ceva{c-base} kennt eine eigene Schrift namens \cevain{ceva}. Wir nutzen sie, um Primärquellen und Stationsfachbegriffe als solche auszuweisen, und zwar unabhängig von der im Original verwendeten Schrift. Mindestens beim ersten Auftreten geben wir eine lateinische Umschrift an: \cevain{ring}. Diese Stelle wird im Index \cref{sec:index} eingetragen.
% Lange Zitate wiederholen wir zur verbesserten Lesbarkeit in lateinischen Buchstaben. 


Daneben gibt es die Schrift \linekin{Linear Construct}; wir verwenden sie, wenn die Primärquellen ihrerseits eine Fremdsprache, konkret: Englisch, benutzen.

Eine Initiale kennzeichnet jeweils den Beginn unserer Interpretation zur Absetzung gegenüber der reinen Exegese.


