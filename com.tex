\csection[zwei]{com}\label{sec:com}



\twofonts[\cite{cbasestarbasemanual}]{
     bedeutung kommunikation ist austausch, konfrontation

    cdcd-ring obere decks (5-7) com-bay an- und abflugshalle mit warteraum be- und entladungszone untere decks (1-4) shuttlebay für raumfahrzeuge fracht- und laderäume montagehallen 
    }

Die früheste Quelle stellt mehr ab auf die physiologischen Befindlichkeiten und den Austausch von Fahrzeugen, Fracht und dergleichen; die kommunikativ-geistigen Aspekte stehen hier noch stärker im Hintergrund, ein archaischer Zug, der möglicherweise auf große materielle Not hinweist. Die Bedeutung von \ceva{cdcd} (auch: \ceva{edcd} ist unbekannt. Die Quelle setzt archaisch eine \ceva{modul} mit \ceva{ring} gleich.

Bedeutend ist die Erwähnung von ebenfalls 7 \ceva{deccs} (in heutiger Schreibweise), die alle durch \ceva{com} tangiert werden.
% Daraus ergeben sich interessante Fragestellungen zur Topologie der Station. 
% Die \ceva{deccs}  liegen offenbar quer zu den \ring{ringen}. 
Der Versuch, sie mit den einzelnen Begriffen (\ceva{com-bay} etc.) zu identifizieren, muss scheitern (vgl.\cref{sec:cience}). Auffällig ist die Wiederkehr der Zahl 7, die eindeutig einen Bezug zu den \ring{ringen} herstellt. Folglich tangigert \ceva{com} alle übrigen \ring{ringe}. Dafür spricht, dass so mancher Zugang zur \ceva{c-base} über einen der anderen \ring{ringe} gefunden hat. 


% \cref{tab:deccs} versucht eine Systematik.

% \begin{table}[ht!]
%     \centering
%     \begin{tabular}{lll}
%     \toprule
%         \multicolumn{3}{l}{\cevain{untere decks} (1-4)}\\
%         \midrule
%          1 & & shuttlebay für raumfahrzeuge  \\
%          2 & & fracht- und laderäume \\
%          3 & & montagehallen \\
%          4 & & ? \\
%          \midrule
%         \multicolumn{3}{l}{\cevain{obere decks} (5-7)}\\         
%         \midrule
%          5 & & com-bay \\
%          6 & & an- und abflugshalle mit warteraum\\
%          7 & & be- und entlaungszone\\
%         \bottomrule
%     \end{tabular}
%     \caption{Die \cevain{deccs} von \cevain{com} nach \cite{cbasestarbasemanual}}
%     \label{tab:deccs}
% \end{table}

 
\twofonts[\cite{ctour}]{
    der raumhafen der c-base: hier laufen die communicationseinrichtungen zusammen und verteilen sich die zugänge zur station.
        \begin{itemize}
            \item shuttlebays und hangars
            \item ancunftspromenade mit empfangsstation und wartehallen
            \item lifeboats und worcpots
            \item montagehallen, fracht- und laderäume 
        \end{itemize}
        com bedeutet communication; meint austausch und confrontation
    }

% Die Formulierung \ceva{verteilen sich die zugänge} hat Anlass gegeben zu der Vermutung, die eigentlichen Zugänge lägen anderswo, sie verteilten sich lediglich von hier aus. 

Auffällig ist die Vertauschung der Verben im ersten Satz, denn man würde ja erwarten: 
\emph{verteilen sich die communicationseinrichtungen} und 
\emph{laufen die zugänge zusammen}. Die Quelle betont mithin die Uneindeutigkeit der Richtungen; \ceva{com} ist sowohl materiell wie auch geistig bidirektional.




% Entsprechend wird \ceva{com} heute mehr als Kommunikationsstruktur denn als Ort materieller Verteilung betrachtet. 

Eine Interpretation von \ceva{com} als Abkürzung für \cevain{com$\cdot$mune} verbietet sich. Die Ähnlichkeit zu \cevain{com$\cdot$puter} gilt als Akzidenz.
    
\begin{newstuff}
    \lettrine{H}{erum} um \ceva{core} befindet sich eine Verteilungs- und Mitteilungsstruktur namens \ceva{com}, was etwa dem Zentralnervensystem, aber auch dem stofflichen Austauschsystem (Nahrungsaufnahme, Verdauung etc.) einer karbonbasierten Lebensform entspräche. Hier werden Inputs und Outputs vom und zum \ceva{core} koordiniert; doch zugleich geschieht in diesem Austausch etwas Wesentliches für die Selbstprogrammierung der Raumstation. In \ceva{com} findet der Austausch von Informationen, aber auch von Anregungen und Anweisungen positiver, negativer, fragender oder imaginärer Natur statt. 

    \twofonts[\cite{cbasebook},~S.~39]
    {der com-ring ist die discussions- und präsentationsplattform der c-tation. die c-base generiert neue ideen zwischen allen ebenen und ringen und eröffnet contactmöglichkeiten jeder art - austausch und confrontation - die sowohl zur erweiterung der crew als auch zum aufbau interstellarer beziehungen führen. }

    Die Autopoiesis (Selbsterschaffung) der Raumstation geschieht durch kommunikative Prozesse zwischen den bereits aktivierten Modulen - unabhängig davon, in welchem Zustand sich diese Module aktuell befinden. Gleichzeitig wirkt der Zustand jedes Moduls auf alle übrigen Module und damit auf den komplexen Anregungszustand des \ceva{core} selbst zurück. Die entstehenden Schwingungen und Wellen in diesem komplexen, mehrdimensionalen und intertemporalen Kommunikationsnetzwerk lassen Gebilde entstehen, die weiter in die äußeren \ring{ringe} getragen werden (siehe z.B. \ceva{culture}):
    \twofonts[\cite{cbasebook},~S.~39]{hier laufen sämtliche communicationseinrichtungen zusammen und verteilen sich die zugänge zur station.}

    \ceva{com} umfasst somit Informationseinheiten ebenso wie stoffliche Elemente und Bauteile in unterschiedlichen Aggregatzuständen und Aggregierungsgraden. Hier tauchen ständig neue Lebensformen und Aggregate verschiedener Be\-wusst\-seins- und Fertigungsstufen auf. Allen gemeinsam ist, dass sie nicht so, wie sie sind, abgeschlossen vollendet sind. Sie alle tragen bei zum Wiederwerden der Raumstation - durch Integration und durch Abwandlung. Daher ist Kom\-mu\-ni\-ka\-ti\-ons- Integrations- und Wandlungsfähigkeit unabdingbar für ein \cevain{andoccen} an der Raumstation. 

    Die Verbindungsfunktion von \ceva{com} führt zu Kurzschlüssen mit allen anderen Ringen, denn Kommunikation ist bekanntlich die Grundlage jedes autopoietischen Systems. Das reibungslose Funktionieren der Kommunikationskanäle ist daher eine der vornehmsten Aufgaben bei der \cevain{reconstruccion}. 
    % Last but not least umfasst dieser \ring{ring} auch die physischen Schleusen und Andockmöglichkeiten für den stofflichen, intellektuellen und emotionalen Austausch mit der Umwelt.
\end{newstuff}