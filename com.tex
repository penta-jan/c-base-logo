\csection{com}
% \section{com \hspace{2ex} \raisebox{1pt}{{\fontspec{[ceva-c2.ttf]}(com)}}}

\Hrule[zwei]

\twofonts{
    der raumhafen der c-base: hier laufen die communicationseinrichtungen zusammen und verteilen sich die zugänge zur station.
        \begin{itemize}
            \item shuttlebays und hangars
            \item ancunftspromenade mit empfangsstation und wartehallen
            \item lifeboats und worcpots
            \item montagehallen, fracht- und laderäume 
        \end{itemize}
    }

\twofonts{
    cdcd-ring obere decks (5-7) com-bay an- und abflugshallel mit warteraum be- und entladungszone untere decks (1-4) shuttlebay für raumfahrzeuge fracht- und laderäume montagehallen 
    }
    
\begin{newstuff}
    \lettrine{H}{erum} um \ceva{core} befindet sich eine Verteilungs- und Mitteilungsstruktur namens \ceva{com}, was etwa dem Zentralnervensystem einer karbonbasierten Lebensform entspräche. Hier werden Inputs und Outputs vom und zum \ceva{core} koordiniert; doch zugleich geschieht in diesem Austausch etwas Wesentliches für die Selbstprogramierung der Raumstation. In \ceva{com} findet der Austausch von Informationen, aber auch von Anregungen und Anweisungen positiver, negativer, fragender oder imaginärer Natur statt. 

    Dazu schreibt \cite[S. 39]{cbasebook}:
    \twofonts{der com-ring ist die discussions- und präsentationsplattform der c-tation. die c-base generiert neue ideen zwischen allen ebenen und ringen und eröffnet contactmöglichkeiten jeder art - austausch und confrontation - die sowohl zur erweiterung der crew als auch zum aufbau interstellarer beziehungen führen. }

    Die Autopoiesis (Selbsterschaffung) der Raumstation geschieht durch kommunikative Prozesse zwischen den bereits aktivierten Modulen - unabhängig davon, in welchem Zustand sich diese Module aktuell befinden. Gleichzeitig wirkt der Zustand jedes Moduls auf alle übrigen Module und damit auf den komplexen Anregungszustand des \ceva{core} selbst zurück. Die entstehenden Schwingungen und Wellen in diesem komplexen, mehrdimensionalen und intertemporalen Kommunikationsnetzwerk lassen Gebilde entstehen, die weiter in die äußeren Ringe getragen werden (siehe z.B. \ceva{culture}); in \cite[S. 39]{cbasebook} heißt es:
    \twofonts{hier laufen sämtliche communicationseinrichtungen zusammen und verteilen sich die zugänge zur station.}

    \ceva{com} umfasst somit Informationseinheiten ebenso wie stoffliche Elemente und Bauteile in unterschiedlichen Aggregatzuständen und Aggregierungsgraden. Hier tauchen ständig neue Lebensformen und Aggregate verschiedener Be\-wusst\-seins- und Fertigungsstufen auf. Allen gemeinsam ist, dass sie nicht so, wie sie sind, abgeschlossen vollendet sind. Sie alle tragen bei zum Wiederwerden der Raumstation - durch Integration und durch Abwandlung. Daher ist Kom\-mu\-ni\-ka\-ti\-ons- Integrations- und Wandlungsfähigkeit unabdingbar für ein Andocken an der Raumstation. %, und Wesen ohne solche Bereitschaft werden hier unter Umständen auch aussortiert. 

    Die Verbindungsfunktion von \ceva{com} führt zu Kurzschlüssen zu allen anderen Ringen, denn Kommunikation ist bekanntlich die Grundlage jedes autopoietischen Systems. Das reibungslose Funktionieren der Kommunikationskanäle ist daher eine der vornehmsten Aufgaben bei der \cevain{reconstruccion}. Und schließlich umfasst dieser Ring auch die physischen Schleusen und Andockmöglichkeiten für den stofflichen, intellektuellen und emotionalen Austausch mit der Umwelt.
\end{newstuff}