\csection{carbon}
% \section{carbon \hspace{2ex} \raisebox{1pt}{{\fontspec{[ceva-c2.ttf]}(carbon)}}}

\Hrule[sechs]

\twofonts{das wohnungsdeckmodul bietet familiengerechte lebensqualität für die crew:
    die c-base ist ein generationenschiff und hat platz für eine besatzungsstärke von bis zu 5000 carboneinheiten. 
    In einem notfall kann jede der 144 habitate von der station gelöst werden und eigenständig manövrieren.
    \begin{itemize}
        \item 144 lifehabitatmodule mit je 12 habitateinheiten
        \item ver- und entsorgungseinheiten
        \item foodsupply mit medical support unit
        \item cindergarten
        \item schulen
    \end{itemize}
    }

\twofonts{
    life habitat familiengerechte wohnräume für die besatzung mit schlaf- und aufenthaltsraum, küche und bad incl. ver- und entsorgungseinheiten. 
    }

\begin{newstuff}
    \lettrine{N}{icht} jeder Computer oder jedes kalkulierend denkende System ist notwendigerweise siliziumbasiert. Die Raumstation als System insgesamt funktioniert eindeutig als Zusammenspiel von mehreren Dimensionen und von Systemen, deren Schnittstellen ganz unterschiedlich sind. Insbesondere arbeiten die digitalen Systeme mit den karbonbasierten Systemen zusammen über optische und haptische Interfaces, darunter Monitore, Brillen, Tastaturen und selbst Brettspiele wie \cevain{pentagame}. Nur das gelingende Interagieren aller Beteiligten bringt die Stationsforschung voran und damit das Ziel des Abflugs näher. \cite[S. 169]{cbasebook} erläutert:

    \twofonts{carbon ist das basiselement der organischen chemie und documentiert die geschichte der auf cohlenstoff basierenden lebensformen auf der c-tation. es zeichnet die bisher becannte stationsgeschichte auf ...}
    
    Die Karboneinheiten haben selbstredend Bedürfnisse, denen die Station vorsorglich Rechnung trägt.
    Hierzu gehören die im kanonischen Text aufgeführten pysiologischen Annehmlichkeiten. Auch befindet sich die  Raumstation im Aufbau, es ist also zu erwarten, dass sich die kulinarische, hygienische, gesundheitliche Situation im Werdensprozess der c-base weiter verbessern wird. Das gilt für die Qualität der vorhandenen gasförmigen, flüssigen, halbflüssigen und festen Stoffströme ebenso wie für die akustischen, optischen, haptischen und thermischen Umgebungsvariablen. 

    Hier befinden sich auch verschiedene Kultivierungsmodule zur Wandlung ordinärer Stoffe in höhere Elemente (\emph{aus Scheiße Gold machen,} \cevain{Alchemie}). In einer Reihe von kontrollierten Versuchen wurde hier beispielsweise Kuhmilch in interessante andere Stoffe umgewandelt. Andere Versuche mit verlegten Kleidungsstücken und vergessenen \mbox{Pizza}\-kartons blieben dagegen bislang überwiegend erfolglos.
  
    Insofern der Werdungsprozess der Station jedes einzelne Modul umfasst, gehört zum Prozess der Selbstbewusstwerdung auch ein Anwachsen des Bewusstseins der Auswirkungen der eigenen Handlungen und Nichthandlungen auf das Umgebungshabitat, das Selbstwohlbefinden und das Wohlbefinden der anderen. Das kohlenstoffbasierte Gedächtnis der Station muss allerdings immer wieder trainiert und herausgefordert werden.

    Teil dieses Ökosystem der Station und ihrer Bewohner sind diverse andere terristrische und extraterrestrische Lebensformen, die sich auf mitunter sehr anderen Raumzeitlinien bewegen und so verschiedene Zeit- und Raumauffassungen haben.\footnote{So z.B. der Symbiont \cite{symbiont}.} Innerhalb dieser unterschiedlichen Bezugssysteme und auch zwischen ihnen emergieren unregelmäßig neue, fruchtbare und erfreuliche Interaktionen, aus denen neue kulturelle und auch biologische Produkte als neue Bezugspunkte hervorgehen. 

    Der  Zustand der Station erlaubt aktuell offenbar nicht den Betrieb der in \cite{ctour} und \cite{cbasestarbasemanual} angekündigten \ceva{cindergärten} und \ceva{schulen}, und die Station ist weiterhin nur volljährigen zugänglich. Ebenso ist die Übernachtung zur Zeit der Drucklegung als \cevain{Schlaftäterschaft} noch unzulässig \cite[S. 58]{cbasebook}.
\end{newstuff}