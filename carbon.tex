\csection{carbon}
% \section{carbon \hspace{2ex} \raisebox{1pt}{{\fontspec{[ceva-c2.ttf]}(carbon)}}}

\Hrule[sechs]


\twofonts[\cite{cbasestarbasemanual}]{
    bedeutung geschichte der auf kohlenstoff basierenden lebensformen. basiselement der organischen chemie, DNA

    life habitat familiengerechte wohnräume für die besatzung mit schlaf- und aufenthaltsraum, küche und bad incl. ver- und entsorgungseinheiten. 
    }

\twofonts[\cite{ctour}]{das wohnungsdeckmodul bietet familiengerechte lebensqualität für die crew:
    die c-base ist ein generationenschiff und hat platz für eine besatzungsstärke von bis zu 5000 carboneinheiten. 
    In einem notfall kann jede der 144 habitate von der station gelöst werden und eigenständig manövrieren.
    \begin{itemize}
        \item 144 lifehabitatmodule mit je 12 habitateinheiten
        \item ver- und entsorgungseinheiten
        \item foodsupply mit medical support unit
        \item cindergarten
        \item schulen
    \end{itemize}
    carbon ist das basiselement der organischen chemie und meint die geschichte der auf cohlenstoff basierenden lebensformen
    }

Hier wird erstmalig von einem \cevain{wohnungsdeckmodul} gesprochen. Ob es ein solches tatsächlich einmal gegeben hat oder sogar aktuell gibt, oder ob \ceva{c-tour} hier Legenden widergibt, ist Gegenstand hitziger Debatten. Sicher ist, dass es eines geben \emph{soll}.

\begin{newstuff}
    \lettrine{N}{icht} jeder Computer oder jedes kalkulierend denkende System ist notwendigerweise siliziumbasiert. Die Raumstation als System insgesamt funktioniert eindeutig als Zusammenspiel von mehreren Dimensionen und von Systemen, deren Schnittstellen ganz unterschiedlich sind. Insbesondere arbeiten die digitalen Systeme mit den karbonbasierten Systemen zusammen über optische, haptische und telepatische Interfaces. 
    %, darunter Monitore, Brillen, Tastaturen und selbst Brettspiele wie \cevain{pentagame}. 
    Nur das gelingende Interagieren aller Beteiligten bringt die Stationsforschung voran und damit das Ziel des \cevain{Abflugs} näher.

    \twofonts[\cite{cbasebook},~S.~169]{carbon ist das basiselement der organischen chemie und documentiert die geschichte der auf cohlenstoff basierenden lebensformen auf der c-tation. es zeichnet die bisher becannte stationsgeschichte auf ...}
    
    Die Karboneinheiten haben selbstredend Bedürfnisse, denen die Station vorsorglich Rechnung trägt.
    Hierzu gehören die im kanonischen Text aufgeführten pysiologischen Annehmlichkeiten. Auch befindet sich die  Raumstation im Aufbau, es ist also zu erwarten, dass sich die kulinarische, hygienische, gesundheitliche Situation im Werdensprozess der c-base weiter verbessern wird. Das gilt für die Qualität der vorhandenen gasförmigen, flüssigen, halbflüssigen und festen Stoffströme ebenso wie für die akustischen, optischen, haptischen und thermischen Umgebungsvariablen. 

    Hier befinden sich auch verschiedene Kultivierungsmodule zur Wandlung ordinärer Stoffe in höhere Elemente (\cevain{Alchemie} /\emph{"`aus Scheiße Gold machen"'}). In einer Reihe von kontrollierten Versuchen wurde hier beispielsweise Milch in interessante andere Stoffe umgewandelt. Versuche mit verlegten Kleidungsstücken und vergessenen \mbox{Pizza}\-kartons blieben dagegen bislang überwiegend erfolglos.
  
    Insofern der Werdensprozess der Station jedes einzelne Modul umfasst, gehört zum Prozess der Selbstbewusstwerdung auch ein Anwachsen des Bewusstseins der Auswirkungen der eigenen Handlungen und Nichthandlungen auf das Umgebungshabitat, das Selbstwohlbefinden und das Wohlbefinden der anderen. Das kohlenstoffbasierte Gedächtnis der Station muss allerdings immer wieder trainiert und herausgefordert werden.

    Teil dieses Ökosystem der Station und ihrer Bewohner sind diverse andere terristrische und extraterrestrische Lebensformen, die sich auf mitunter sehr anderen Raumzeitlinien bewegen und so verschiedene Zeit- und Raumauffassungen haben, wie z.B. der \cevain{Symbiont}~\cite{symbiont}. Innerhalb dieser unterschiedlichen Bezugssysteme und auch zwischen ihnen erwachsen unregelmäßig neue, fruchtbare und erfreuliche Interaktionen, aus denen neue kulturelle und auch biologische Produkte als neue Bezugspunkte hervorgehen. 

    Der gegenwärtige Zustand der Station erlaubt nicht den Betrieb der in~\cite{ctour} und~\cite{cbasestarbasemanual} angekündigten \ceva{cindergärten} und \ceva{schulen}, und die Station ist weiterhin nur volljährigen zugänglich. Ebenso ist die Übernachtung zur Zeit der Drucklegung als \cevain{Schlaf\-tä\-ter\-schaft} noch unzulässig~\cite[S. 58]{cbasebook}.
\end{newstuff}