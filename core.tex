\csection{core}
% \section{core \hspace{2ex} \raisebox{1pt}{{\fontspec{[ceva-c2.ttf]}(core)}}}

\Hrule[eins]

\twofonts{
    core ist der innerste ring der c-base. er ist die commandoeinheit der raumstation: hier liegen die brücce, der centralcomputer exitc-beam und externe speichereinheiten, sowie der mino-reactor, der cybernetische antrieb der c-base auf quecksilberbasis.
    Die antenna ersteckt sich zur Zeit auf 368m über Normal Null sensor-array.
    }

\twofonts{
    commando-, antenneneinheit verwaltungseinheit mit brücke und commandodeck centralcomputer c-beam MBA-Ein\-hei\-ten und externe Speichermodule obere antenna: z.Zt. 368m über NN sensor-aray. ausrichtung auf die planetenoberfläche und tower 
}

\begin{newstuff}
    \lettrine{I}{m}
    Im Zentrum der Raumstation befindet sich eine Antriebs- und Steuerungseinheit. Die genaue Funktionsweise ist bereits recht gut erforscht \cite[S.~31ff]{cbasebook}. Es handelt  sich um eine multidimensionale, autopoietische Struktur, in deren Inneren sich eine Energiequelle befindet \cite[S. 31]{cbasebook}:

    \twofonts{Möbius-Band-accumulator (MBA). hauptenergiespeicher; als ring umgibt er den $\rightarrow$ minoreactor. die eigenschaft, energie unbestimmter grössen / einheiten aufzunehmen, zu speichern, wird dadurch ermöglicht, dass sie auf einem autoinitiierten endlosband in ständigem fluss ist.}

    Neben den technischen Details ist vor allem Bedeutsam, dass sich hier ein autopoietisches Systems selbst (re)-konstruiert. 
    Der Mino-Reactor ist vermutlich eine Singularität, aus der nur Dinge herauskommen können ("`Weißes Loch"'). Die hervorgebrachte Energie und Materie fluktuiert allerdings stark, so dass es desöfteren zur Materialisierung unvorhergesehener Objekte kommt.

    Der Mino-Reaktor \ceva{core} ist Quelle von unerschöpflicher Energie, Materie und Ideen, die dann im Möbius-Band-Generator zu sich selbst und ihrer Bedeutung in ihrer Umwelt finden.
    Das bedeutet, dass sich hier eine Art "`Bewusstsein"' der Raumstation befand bzw. befindet und befinden wird. 
    Dieses vereint digitale und analoge Aspekte und ist nicht auf Siliziumverbidnungen beschränkt, sondern kommuniziert und lebt auch in und mit anderen karbonbasierten Bewusstseinsmanifestationen. 

    Das lässt sich vereinfachend etwas so beschreiben: der zentrale Computer, besser: das zentrale Bewusstsein, der zentrale Wille und damit die zentrale Antriebseinheit der abgestürzte Raumstation befindet sich in einem fortgesetzten "`Reboot"'-Prozess, in welchem der {\fontspec{[ceva-c2.ttf]}core} nach und nach die verschiedenen Funktionen der Raumstation wieder "`online"' bzw. zum Leben bringt. Das geschieht auf materieller Ebene erst nach Evozierung von diesen Prozess anstoßenden  Bewusstseinsprozessen. Konkret wirkt auf einer nicht-materiellen Dimension der \ceva{core} auf dafür empfängliche Menschen, die so von der \ceva{c-base} inspiriert werden, zu ihrer Rekonstruktion beizutragen. 

    Die Tatsache, dass es sich um eine abgestürzte, also beschädigte bzw. unvollständige Raumstation handelt, erklärt auch, weshalb viele der Rekonstruktionsansätze für gemeine Menschen schwer oder gar nicht verständlich bleiben. Der autopoietische Prozess der Selbstwiedererschaffung der c-base geschieht durch mitunter fast infantil anmutende experimentelle Prozesse und Vorprozesse. Doch wie beim werdenden Menschen sind alle diese Experimente Zwischenstufen zum entwickelten, höheren Bewusstsein.  

    Der halberwachte Bordcomputer der Raumstation scheint mit einer uns nicht genau bekannten Technik immer wieder Individuen mit besonderen Begabungen anzuziehen, die dann durch dasselbe Medium inspiriert werden, die Rekonstruktion der Raumstation voranzubringen. Welche Qualitäten genau zu dieser Auswahl führen, ist aktuell nur in Ansätzen bekannt. Jedoch scheint die Strategie des \ceva{core} bislang insgesamt erfolgreich.
\end{newstuff}


    

