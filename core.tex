\csection[eins]{core}\label{sec:core}

\twofonts[\cite{cbasestarbasemanual}]{
   bedeutung idee, zentrum. keimzelle c-base: der verein und seine crew
    
    commando-, antenneneinheit verwaltungseinheit mit brücke und commandodeck centralcomputer c-beam MBA-Ein\-hei\-ten und externe Speichermodule obere antenna: z.Zt. 368m über NN sensor-aray. ausrichtung auf die planetenoberfläche und tower 
}

Dies ist die früheste Erwähnung des \cevain{centralcomputer} und seiner Benennung als \cevain{\mbox{c-beam}} und die erste Beschreibung von \ceva{core} überhaupt.

\twofonts[\cite{ctour}]{
    core ist der innerste ring der c-base. er ist die commandoeinheit der raumstation: hier liegen die brücce, der centralcomputer c-beam und externe speichereinheiten, sowie der mino-reactor, der cybernetische antrieb der c-base auf quecksilberbasis.
    Die antenna ersteckt sich zur Zeit auf 368m über Normal Null sensor-array.

    core meint ursprung, idee, zentrum. die ceimzelle der c-base: ihre crew und der verein - von hier aus nimmt immer wieder alles seinen anfang
    }


Unklar ist die Bedeutung von \ceva{sencor-array}; vermutlich ein Hinweis auf Vorurteile der Kartographie. 

Beide Quellen schreiben \cevain{antenna}, vermutlich ein Hinweis auf eine feminine Qualität dieser Einheit unabhängig von ihrer phallischen Form als \cevain{tower}. Auch wird sie dadurch unterschieden von den in \cref{sec:clamp} erwähnten \cevain{antennen- und sensoreneinheiten}.

\begin{newstuff}
    \lettrine{I}{m}
    Im Zentrum der Raumstation befindet sich eine Antriebs- und Steuerungseinheit. Die genaue Funktionsweise ist bereits recht gut erforscht~\cite[S.~31ff]{cbasebook}. Es handelt  sich um eine multidimensionale, autopoietische Struktur, in deren Inneren sich eine Energiequelle befindet.

    \twofonts[\cite{cbasepressemap}]{Die Fortbewegung im Raum erfolgt mittels eines vom Cy\-ber\-ne\-ti\-schen-Queck\-sil\-ber-Reaktor (CQR) erzeugten Feldes, das die c-base aus der Raumzeit ausschneidet, um sie über die Einstein-Rosenbaum-Brücke im Orbit des zu formenden Planeten materialisieren zu lassen. 
    }

    Neben den technischen Details ist vor allem bedeutsam, dass sich hier ein autopoietisches Systems selbst (re-)konstruiert. 
    Der \ceva{Mino-Reactor} ist vermutlich eine Singularität, aus der nur Dinge herauskommen können ("`Weißes Loch"'). Die hervorgebrachte Energie und Materie fluktuiert allerdings stark, so dass es des Öfteren zur Materialisierung unvorhergesehener Objekte kommt.

    \twofonts[\cite{cbasebook}, S. 31]{Möbius-Band-accumulator (MBA). hauptenergiespeicher; als ring umgibt er den $\rightarrow$ minoreactor. die eigenschaft, energie unbestimmter grössen / einheiten aufzunehmen, zu speichern, wird dadurch ermöglicht, dass sie auf einem autoinitiierten endlosband in ständigem fluss ist.}

    \ceva{core} ist Quelle von unerschöpflicher Energie, Materie und Ideen, die dann im \ceva{Mö\-bi\-us-band-accelerator} zu sich selbst und ihrer Bedeutung in ihrer Umwelt finden.\footnote{Noch \cite{cbasestarbasemanual} spricht von \ceva{MBA-Einheiten}, also im Plural; in der späteren Literatur wird nur von \emph{dem} MBA gesprochen. Vermutlich  hat sich die Einheit der Befunde erst bei weiterer Ausgrabung herausgestellt.} 
    Das bedeutet, dass sich hier eine Art "`Bewusstsein"' der Raumstation befand bzw. befindet und befinden wird. 
    Dieses vereint digitale und analoge Aspekte und ist nicht auf Siliziumverbindungen beschränkt, sondern kommuniziert und lebt auch in und mit anderen karbonbasierten Bewusstseinsmanifestationen. 

    \twofonts[\cite{cbasepressemap}]{
        Da aber die Raumstation ein molekulares Erinnerungsmodul besitzt, das sowohl Wissen als auch Zusammensetzung von Materie speichert und wieder herstellen kann, konnte sich vor ca. 100.000 Jahren der Boardcomputer c-beam eigenständig einschalten. 
    }
    

    Der Prozess läuft vereinfacht gesagt so ab: der zentrale Computer \cevain{c-beam}, besser: das zentrale Bewusstsein, der zentrale Wille und damit die zentrale Antriebseinheit der abgestürzten Raumstation, befindet sich in einem fortgesetzten Reboot-Prozess, in welchem der {\fontspec{[ceva-c2.ttf]}core} nach und nach die verschiedenen Funktionen der Raumstation wieder "`online"' bzw. zum Leben bringt. Das geschieht auf materieller Ebene erst nach Evozierung von diesen Prozess anstoßenden  Bewusstseinsprozessen. Konkret wirkt auf einer nicht-materiellen Dimension der \ceva{c-beam} auf dafür empfängliche Menschen, die so von der \ceva{c-base} inspiriert werden, zu ihrer Rekonstruktion beizutragen. 

    \twofonts[\cite{cbaselogbuchnow}]{
    cünstler, designer, wissenschaftler, computertechniker, philosophen, pädagogen, naturwissenschaftler, fictionäre und alle anderen werden angesprochen um zu helfen
    }

    Hier werden \cevain{erstberufene} aufgelistet; interessanterweise stehen \ceva{cünstler} an erster Stelle, gefolgt von \ceva{designern} usw., wohingegen beispielsweise Programmierer, Polizisten, Mediziner, Juristen usw. nicht expizit erwähnt werden (wohl aber unter \ceva{alle anderen} fallen). Leider wissen wir nicht, zu welcher dieser Gruppen sich die \ceva{gründer} selber zählten; sind diese \ceva{erstberufenen} eine Ergänzung oder ein Spielgebild der \cevain{urcrew}?
    
    Umstritten ist die genaue Bedeutung von \ceva{fictionär}; vermutlich handelt es sich um einen zusammenfassenden Term, der auf die allgemeine Geisteshaltung abstellt, mithin darauf, durch \ceva{c-beam} inspiriert zu sein.

    Die Technik von \ceva{c-beam} beschreibt \ceva{c-booc} wie folgt:
    \twofonts[\cite{cbasebook},~S.~28]{c-beam ist die centrale scheibenförmige recheneinheit der c-base. sie besteht aus neuronalen holografischen speicherbausteinen und ist zu eigenständigem handeln fähig. ihren energiebedarf zieht sie aus dem $\rightarrow$ möbius-band-accumulator, den sie von oben und unten umschließt.}
    
    Die Tatsache, dass es sich um eine abgestürzte, also beschädigte bzw. unvollständige Raumstation handelt, erklärt auch, weshalb viele der Rekonstruktionsansätze für gemeine Menschen schwer oder gar nicht verständlich bleiben. Der autopoietische Prozess der Selbstwiedererschaffung der c-base geschieht durch mitunter fast infantil anmutende experimentelle Prozesse und Vorprozesse. Doch wie beim werdenden Menschen sind alle diese Experimente Zwischenstufen zum entwickelten, höheren Bewusstsein.  

    \twofonts[\cite{cbasepressemap}]{Die c-base entwickelt sich so zu einem Wissenspool und Ideenbiotop mit einer Vielzahl von kreativen, wis\-sen\-schaft\-lich-tech\-ni\-schen und zukunftsorientierten Menschen...}    

    Der halberwachte Bordcomputer der Raumstation \cevain{c-beam} scheint mit einer uns nicht genau bekannten Technik immer wieder Individuen mit besonderen Begabungen anzuziehen, die dann durch dasselbe Medium inspiriert werden, die Rekonstruktion der Raumstation voranzubringen. Welche Qualitäten genau zu dieser Auswahl führen, ist aktuell nur in Ansätzen bekannt. Jedoch scheint die Strategie des \ceva{core} bislang insgesamt erfolgreich.
\end{newstuff}


    

