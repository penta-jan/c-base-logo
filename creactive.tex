\csection{creactive}
% \section{creactive \hspace{2ex} \raisebox{1pt}{{\fontspec{[ceva-c2.ttf]}(creactive)}}}

\Hrule[vier]

\twofonts{
    der vierte ring von innen ist gleichzeitig auch der vierte ring von aussen. 
    wahrscheinlich diente er der crew deswegen als inspirationsgenerator. 
    in verschiedensten modulen sind creactive- und ideenfördernde einrichtungen untergebracht:
    \begin{itemize}
        \item die mobile creationbox, die an jede stelle des cience-rings fährt
        \item das exitcosmolab, die terraforming-wercstatt
        \item langschlafcabinen für die beliebten fivehoundred-year-projects (fyp)
        \item brainlabs und aerosphären
        \item gedächtniscammern, tancs, gedancengeneratoren
        \item völlig leere räume 
    \end{itemize}}

\twofonts{
    creation-box in diesem segment befinden sich unter anderem gedächtniskammern, langschlafkabinen für die beliebten fivehoundred-year-projects (fyp), entspannungs- und ruhekonsolen. die ersten ideen und wahrheiten werden hier geschaffen. die creationbox kann an jede stelle des cience-ring gefahren werden. the most powerful part of c-base 
    }

\begin{newstuff}
   \lettrine{W}{ährend} in \ceva{culture} das Miteinander der Module im Vordergrund steht, ist \ceva{creactive} hauptsächlich ein Ort zur Ermöglichung der schöpferischen Einsamkeit und des kreativen Miteinanders. Hier finden sich Orte, die dem konzentrierten Hervorbringen neuartiger Substanzen (etwa: \cevain{Bio-3D-Druck}) dienen neben solchen, an denen die Module sich der Selbstpflege und dem Refacturing ihrer Programme widmen können. Ferner gibt es Orte zum gemeinsamen Hervorbringen von Plänen, Aggregaten, Artefakten und Prozessen und zum Verbrennen von Kohlenstoffketten. \cite[S. 110]{cbasebook} fügt hinzu:

   \twofonts{auf dem creactive-ring bündelt die crew ihre künstlerischen und technischen visionen in actionen und projecte. hier befindet sich der ideenkompass für creativität, phantasie, fiction und activität.}

    Auch diese Räume ermöglichen Schaffensprozesse im Wiederaufbauprogramm der Raumstation durch ihre schlichte Verfügbarkeit, aber auch durch die in ihnen durch die umliegenden Ringe zur Verfügung stehenden Ressourcen materieller und immaterieller Art. Das sind Werkzeuge und Materialien, aber auch Baupläne, Anregungen und intelligente Hinweise anderer Module. 

    Diese Anregungen und Hinweise sind nicht immer vollständig so, wie erwartet. Das erklärt sich aus dem immer noch infantil-semidebilen Zustand der abgestürzten Raumstation, ist aber auch notwendige Voraussetzung zur vorurteilsfreien Neuschaffung des höherdimensionalen Bauplans für die Station und alle seiner Module, deren künftige Beschaffenheit notwendigerweise die Bewusstseinskapazitäten aller Einzelmodule übersteigt.

    Dieser Ring strebt danach, die Verbindung mit den übrigen Ringen zu vereinfachen (\cevain{Monorail}). Bei einigen Experimenten wurden bereits Geschwindigkeiten außerhalb relativistischer Mechanik erzielt, wodurch es mitunter zu Zeitdilatation oder gar Kausalitätsumkehrungen kommt. Über die Module dieses Rings schreibt \cite[S. 117]{cbasebook}:
    \twofonts{sämtliche labore der creactiveity liegen in einem compacten cegment concentriert beieinander. die plattform kann mobil auf dem ring verschoben werden, so dass sie jede stelle des innen liegenden culture rings oder des außen liegenden science rings erreicht. das crative modul ist zur stelle, dort wo es gebraucht wird.}

    Da laut \cite{ctour}, Satz 1 offensichtlich auch in der Stationsmathematik\footnote{
    Vgl. \cite[Kap. 5]{adams_life} \emph{Bistromathics.}
    }   gilt, dass der Mittelpunkt $M$ eines n-Tupels $I$ gegeben ist durch
    \begin{equation}
        \forall I = [a,b] \subset \mathbb{Z}: |I| \in 2\mathbb{Z}+1 \Rightarrow M(I) = \frac{a+b}{2}
    \end{equation}

    % bzw.

    % \begin{equation}
    %     \forall A = (a_1,\ldots,a_N) \in \mathbb{Z}^N:
    % \quad
    % N \equiv 1 \mod(2)
    % \quad\Rightarrow\quad
    % \forall i \in [1,N] \subset \mathbb{Z}:
    % \quad
    % \left(
    % a_i = M(A) \Leftrightarrow i = \frac{N+1}{2}
    % \right)
    % \end{equation}  
    
    besitzt \ceva{creactive} Mittelpunkt- bzw. Äquatorcharakter.
    Er ist derjenige Ring, um welchen sich die übrigen Ringe auf einander zurückfalten. So entsteht hier ein Kurzschluss und die Autoreferenz, die den autopoietischen Prozess am Leben hält.
\end{newstuff}