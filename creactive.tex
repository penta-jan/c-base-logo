\csection{creactive}%\label{sec:creactive}
% \section{creactive \hspace{2ex} \raisebox{1pt}{{\fontspec{[ceva-c2.ttf]}(creactive)}}}

\Hrule[vier]

\twofonts[\cite{cbasestarbasemanual}]{
    datenblatt: bedeutung kunstwort für ideenkompass: kreativität, phantasie, fiction, aktivität

    creation-box in diesem segment befinden sich unter anderem gedächtniskammern, langschlafkabinen für die beliebten fivehoundred-year-projects (fyp), entspannungs- und ruhekonsolen. die ersten ideen und wahrheiten werden hier geschaffen. die creationbox kann an jede stelle des cience-ring gefahren werden. the most powerful part of c-base 
    }

\twofonts[\cite{ctour}]{
    der vierte ring von innen ist gleichzeitig auch der vierte ring von aussen. 
    wahrscheinlich diente er der crew deswegen als inspirationsgenerator. 
    in verschiedensten modulen sind creactive- und ideenfördernde einrichtungen untergebracht:
    \begin{itemize}
        \item die mobile creationbox, die an jede stelle des cience-rings fährt
        \item das exitcosmolab, die terraforming-wercstatt
        \item langschlafcabinen für die beliebten fivehoundred-year-projects (fyp)
        \item brainlabs und aerosphären
        \item gedächtniscammern, tancs, gedancengeneratoren
        \item völlig leere räume 
    \end{itemize}
    creactivity meint den ideencompass für creativität, phantasie, fiction, activität
    }



\begin{newstuff}
   \lettrine{W}{ährend} in \ceva{culture} das Miteinander der Module im Vordergrund steht, ist \cevalong{creactive} hauptsächlich ein Ort zur Ermöglichung der schöpferischen Einsamkeit und des kreativen Miteinanders (\cevain{creactivität}). Hier finden sich Orte, die dem konzentrierten Hervorbringen neuartiger Substanzen (\cevain{Bio-\-3D--Druck}) dienen neben solchen, an denen die Module sich der Selbstpflege und dem Refacturing ihrer Programme widmen können. Ferner gibt es Orte zum gemeinsamen Hervorbringen von Plänen, Aggregaten, Artefakten und Prozessen und zum Verbrennen von Kohlenstoffketten. 

   \twofonts[\cite{cbasebook},~S.~110]{auf dem creactive-ring bündelt die crew ihre künstlerischen und technischen visionen in actionen und projecte. hier befindet sich der ideenkompass für creativität, phantasie, fiction und activität.}

    Auch diese Räume ermöglichen Schaffensprozesse im Wiederaufbauprogramm der Raumstation durch ihre schlichte Verfügbarkeit, aber auch durch die in ihnen durch die umliegenden \ring{Ringe} zur Verfügung stehenden Ressourcen materieller und immaterieller Art. Das sind Werkzeuge und Materialien, aber auch Baupläne, Anregungen und intelligente Hinweise anderer Module. 

    Diese Anregungen und Hinweise sind nicht immer vollständig so, wie erwartet. Das erklärt sich aus dem immer noch infantil-semidebilen Zustand der abgestürzten Raumstation, ist aber auch notwendige Voraussetzung zur vorurteilsfreien Neuschaffung des höherdimensionalen Bauplans für die Station und alle seiner Module, deren künftige Beschaffenheit notwendigerweise die Bewusstseinskapazitäten aller Einzelmodule übersteigt.

    \ring{creactive} strebt danach, die Verbindung mit den übrigen \ring{Ringen} zu vereinfachen (Stichwort: \cevain{Monorail}). Bei einigen Experimenten wurden bereits Geschwindigkeiten außerhalb relativistischer Mechanik erzielt, wodurch es mitunter zu Zeitdilatation oder gar Kausalitätsumkehrungen kommt. 
    
    \twofonts[\cite{cbasebook},~S.~117]{sämtliche labore der creactiveity liegen in einem compacten cegment concentriert beieinander. die plattform kann mobil auf dem ring verschoben werden, so dass sie jede stelle des innen liegenden culture rings oder des außen liegenden science rings erreicht. das crative modul ist zur stelle, dort wo es gebraucht wird.}

    Durch die \ceva{creactivität} faltent sich die \ceva{c-base} über diesen \ring{Ring} zurück auf sich selbst. So entstehen hier Kurzschlüsse und die Autoreferenz, die den autopoietischen Prozess am Leben hält. 

    Betrachten wir nun den in der kanonischen Literatur besonders herausgestellten Umstand, dass der vierte \ceva{Ring} von außen zugleich der vierte von innen ist. Wegen
        \begin{equation}
        \begin{array}{rcl}
            \forall A = (a_1,\ldots,a_N) \in \mathbb{Z}^N&:&
            \quad
            N\equiv 1\mod(2)\\
            \quad\Rightarrow\quad
            \forall i \in [1,N] \subset \mathbb{Z}&:&
            \quad
            \left(
            a_i = M(A) \Leftrightarrow i = \frac{N+1}{2}
            \right)
        \end{array}
        \end{equation}
    ist der vierte \ring{Ring} von sieben in der Tat genau in der Mitte.
    Von bistromathischen  Erklärungen abgesehen (vgl. \cite[Kap. 4]{adams_life})  bedeutet dies, dass \ceva{creactiv}  Mittelpunkt- bzw. Äquatorcharakter besitzt. 
    Er ist nicht nur derjenige Ring, um welchen sich die übrigen \ring{Ringe} auf einander zurückfalten, sondern auch der, welcher den multidimensionale Körper erweitert \cevain{-- the most powerful part of c-base}. Insofern ist \ceva{creactive} auch ein \cevain{creactor}.
\end{newstuff}