\csection[vier]{creactive}\label{sec:creactive}

\twofonts[\cite{cbasestarbasemanual}]{
    datenblatt: bedeutung kunstwort für ideenkompass: kreativität, phantasie, fiction, aktivität

    creation-box in diesem segment befinden sich unter anderem gedächtniskammern, langschlafkabinen für die beliebten fivehoundred-year-projects (fyp), entspannungs- und ruhekonsolen. die ersten ideen und wahrheiten werden hier geschaffen. die creationbox kann an jede stelle des cience-ring gefahren werden.
    }
     \linek[\cite{cbasestarbasemanual}]{the most powerful part of c-base}

Der Ausdruck \linekin{most$\cdot$powerful}  hat Meinungsverschiedenheiten erzeugt, glaubten doch einige \cevain{creactive} daraus ein Primat gegenüber den übrigen Ringen ableiten zu können (Dogma von der Unfehlbarkeit des Designers).

Inzwischen gibt es drei breit akzeptierte Interpretationen:
\begin{enumerate}
    \item Der Satz betont die besondere Verantwortung der \cevain{creactivität}, z.B. angesichts leerer Flaschen (vgl. \cevain{matelight}).
    \item Er drückt eine besonders enge Verbindung von \ceva{creactive} und \ceva{core} aus. Vermutlich liegt die Wahrheit einmal mehr in der Mitte, bzw. die Aussagen schließen einander nicht aus.
    \item Er muss mit \linekonly{future$\cdot$compatible} zusammen gelesen werden; \ceva{creactive} Entscheidungen sind nicht endgültig, sondern müssen \cevain{beginngültig} sein.  
\end{enumerate}
Unter \ceva{beginngültigkeit} wird allgemein die Fruchtbarkeit einer Idee verstanden. Das Gegenteil ist \emph{non sequitur,} beispielsweise das abwürgende Dogma der \ceva{puristen} vom eingeschränkten Farbraum.

%\footnote{Zur Anwendbarkeit des \emph{tertium non datur} in der \ceva{c-tation} siehe}
% ist die herrschende (wenn auch nicht unangefochtene) Lehrmeinung, dass dieser Satz die besondere Verantwortung der \cevain{creactivität} angesichts leerer Flaschen betont (vgl. \cevain{matelight} \cite{matelight}).

\twofonts[\cite{ctour}]{
    der vierte ring von innen ist gleichzeitig auch der vierte ring von aussen. 
    wahrscheinlich diente er der crew deswegen als inspirationsgenerator. 
    in verschiedensten modulen sind creactive- und ideenfördernde einrichtungen untergebracht:
    \begin{itemize}
        \item die mobile creationbox, die an jede stelle des cience-rings fährt
        \item das cosmolab, die terraforming-wercstatt
        \item langschlafcabinen für die beliebten fivehoundred-year-projects (fyp)
        \item brainlabs und aerosphären
        \item gedächtniscammern, tancs, gedancengeneratoren
        \item völlig leere räume 
    \end{itemize}
    creactivity meint den ideencompass für creativität, phantasie, fiction, activität
    }

Beide Quellen erwähnen ein Artefakt namens \cevain{creationbox}. Es konnte offenbar vor allem in \ceva{cience} fahren - war also ein Teil von \ceva{creactive} innerhalb von \ceva{cience}. Was genau dieses war, ist uns heutigen nicht mehr bekannt. Vielleicht handelt es sich auch um ein mythisch-metaphorisches Vehikel; die kanonischen Quellen bezeichneten dann damit die Rückeinführung der \ceva{creactivität} in die Wissenschaft. 

\ceva{c-tour} spricht an dieser Stelle erstmals vom \cevain{ideencompass}, der in den archaischen Schriften offenbar sogar mit \ceva{creactive} überhaupt gleichgesetzt wurde. Erst im \ceva{c-booc} ist dieser \ceva{ideencompass} ein Objekt innerhalb von \ceva{creactive}. Vermutlich handelt es sich um eine mit dem Navigationssystem verbundenen Einheit. Ein mitten in der Neuzeit aufgefundenes Artefakt - \cevain{pentagame} - wurde mit diesem Interface identifiziert, wobei auch dies nicht unumstritten ist. Das \ceva{cosmolab} wird seit 2020 ausgegraben (zunächst fehlerhaft als \cevain{robolab} identifiziert).

Einige Aufmerksamkeit bekam in jüngster Zeit der Ausdruck \ceva{völlig leere räume}. Aufgrund der archaischen Ausdrucksweise geht die herrschende Lehre heute davon aus, dass hier damals tatsächlich existierende, physische und physisch leere Räume gemeint waren - so schwer so etwas aus heutiger Perspektive auch vorstellbar ist. - Demgegenüber halten einige Interpreten dies für eine an Zen-Buddhismus gemahnende Zeile; wieder andere verweisen auf die Leere hinter der Stirn so mancher Adepten. Vermutlich wird sich dieser Streit niemals auflösen lassen.

\begin{newstuff}
   \lettrine{W}{ährend} in \ceva{culture} das Miteinander der Module im Vordergrund steht, ist \ceva{creactive} hauptsächlich ein Ort zur Ermöglichung der schöpferischen Einsamkeit und des kreativen Miteinanders (\cevain{creactivität}). Hier finden sich Orte, die dem konzentrierten Kacken (\cevain{bio-\-4d--druck}, siehe auch \ceva{carbon}) dienen neben solchen, an denen die Module sich der Selbstpflege und dem Refacturing ihrer Programme widmen können. Ferner gibt es Orte zum gemeinsamen Hervorbringen von Plänen, Aggregaten, Artefakten und Prozessen - und zum Verbrennen von Kohlenstoffketten. 

   \twofonts[\cite{cbasebook},~S.~110]{auf dem creactive-ring bündelt die crew ihre künstlerischen und technischen visionen in actionen und projecte. hier befindet sich der ideenkompass für creativität, phantasie, fiction und activität.}

   An allen Quellen fällt die Reichhaltigkeit des Vokabulars und die Listenhaftigkeit des Wortlauts ins Auge; erwähnt werden unter anderem \cevain{creationbox}, \cevain{cosmolab}~\cite{ctour} und \cevain{ideencompass}.

    Diese Räume und Aggregate ermöglichen Schaffensprozesse im Wiederaufbauprogramm der Raumstation durch ihre schlichte Verfügbarkeit, aber auch durch die in ihnen durch die umliegenden \ring{ringe} zur Verfügung stehenden Ressourcen materieller und immaterieller Art. Das sind Werkzeuge und Materialien, aber auch Baupläne, Anregungen und intelligente Hinweise anderer Module. 

    Diese Anregungen und Hinweise sind nicht immer vollständig so, wie erwartet. Das erklärt sich aus dem immer noch infantil-semidebilen Zustand der abgestürzten Raumstation, ist aber auch notwendige Voraussetzung zur vorurteilsfreien Neuschaffung des höherdimensionalen Bauplans für die Station und alle seiner Module, deren künftige Beschaffenheit notwendigerweise die Bewusstseinskapazitäten aller Einzelmodule übersteigt.

    \ceva{creactive} strebt danach, die Verbindung mit den übrigen \ring{ringen} zu vereinfachen (Stichwort: \cevain{monorail}). Bei einigen Experimenten wurden bereits Geschwindigkeiten außerhalb relativistischer Mechanik erzielt, wodurch es mitunter zu Zeitdilatation oder gar Kausalitätsumkehrungen kommt. 
    
    \twofonts[\cite{cbasebook},~S.~117]{sämtliche labore der creactivity liegen in einem compacten cegment concentriert beieinander. die plattform kann mobil auf dem ring verschoben werden, so dass sie jede stelle des innen liegenden culture rings oder des außen liegenden cience rings erreicht. das creactive modul ist zur stelle, dort wo es gebraucht wird.}

    Zwar sind grundsätzlich alle \ring{ringe} modular und überall zur Stelle, aber besonders herausgestellt wird dies vor allem für \ceva{creactiv}. Zugleich wird der Zusammenhalt betont: \ceva{compact ... concentriert beieinander}.
    Durch die \ceva{creactivität} faltent sich die \ceva{c-base} über diesen \ring{ring} zurück auf sich selbst. So entstehen hier Kurzschlüsse und die Autoreferenz, die den autopoietischen Prozess am Leben hält. Die schöpferische Kraft \ceva{creactivität} spielt eine\cevain{c\_lüsselrolle} bei der Emergenz der \ceva{c-tation}.

    \ceva{creactive} ist als Energieverstärkungs- und Umwandlungsmodul dem \ceva{MBA} (vgl. \cref{sec:core}) verwandt. Diese Umwandlungs- oder Energiequelle \ceva{creactivität} ist nicht zu verwechselnd mit \cevain{cunst}, die hier gar nicht erst erwähnt wird. Für \ceva{creactivität} ist \ceva{cunst} ein notwendiges Beiprodukt.
    
    Die \ceva{cwellen} rechtfertigen eine Sonderstellung von \ceva{creactive} in der Stationsdynamik und -topologie. Daraus lassen sich einige Schlüsse ziehen, was wir weiter unten, in \cref{sec:mathematik}, getan haben werden. 
\end{newstuff}