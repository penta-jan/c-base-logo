\section*{Kanonische Aussagen}\addcontentsline{toc}{section}{Kanonische Aussagen}
\fancyhead[RO]{Sachlage}
% \fancyhead[LO]{}

    \twofonts[\cite{cbasefund}]{
    an einem verregneten nachmittag im august 1995 stolperte Hardy Krause über ein herumliegendes teil in einem bauschacht nördlich des alexanderplatzes in berlin.
    [...]
    % verärgert trat er nach dem außerordentlich harten objekt, welches ihm innnerhalb der nächsten halben stunde wohl eine schöne beule bescheren würde und fluchte.
    % doch 
    dann betrachtete er das
    [...]
    % die beule verursachende 
    stück genauer und bemerkte neben einer revolutionären farbgebung (metallisch-violett) erstaunliches: auf diesem irgendwie nicht in diese umwelt passenden stück schrott waren schriftzeichen eingraviert
    [...]:
    % , die unbedingt einer genaueren untersuchung unterzogen werden mußten.
    % 
    % also grub ein team das fundstück aus und brachten es zu einem befreundeten radiochemiker. mit hilfe des kohlenstoff-14-tests (auch radiocarbon-methode genannt) fand er heraus, daß es, nach irdischem ermessen, mindestens 100.000 (!) jahre alt sein müßte.
    % 
    % außerdem enthielte es ein element mit einer ordnungszahl von über 200, das bisher, selbst mit den besten technischen möglichkeiten nicht herstellbar ist. es ist also älter als jedes bisher gefundene von menschen so kunstvoll bearbeitete metallstück und kann wahrscheinlich erst irgendwann in der fernen zukunft enstanden sein!?!

    % auf seiner oberfläche sind zudem irdische schriftzeichen eingraviert
    
    c-base project - be future compatible.
    }

Dieser Bericht ist die älteste vorgefundene Quelle und zugleich der Ur\-sprungs\-my\-thos der \ceva{c-base} mit dem Auffinden des \cevain{urartefacts}. Der legendäre "`Hardy Krause"' wird später mit \cevain{cynk} identifiziert. Der Verbleib des \ceva{urartefacts} ist erstaunlicherweise ungeklärt, was zu vielerlei Spekulation Anlass gibt. Auch fehlen verlässliche biographische Daten über die \ceva{gründer}. -- Später heißt es dann: 

    \twofonts[\cite{cbasestarbasemanual}]{1995 wurden unter Berlin-Mitte die Überreste einer 4,5 Milliarden Jahre alten Raumstation entdeckt. Erste Forschungen ergaben, daß sich die c-för\-mi\-ge Raumstation mit ihrem Mittelpunkt unter dem heutigen Alexanderplatz \emph{befinden muß} und aus 7 Ringen besteht. Aufgrund eines Fundstückes mit der Aufschrift "`\emph{c-base - be future compatible}"' und in Anlehnung an die Anzahl der Ringe, legte das anfänglich nur aus wenigen Mitgliedern bestehene Rekonstructionsteam den Projektnamen und die Aufteilung in sieben Arbeitsbereiche fest.
    
    construct: die raumstation besteht aus sieben ringen, zum teil drehbar. insgesamt hat sie einen durchmesser von 1650 metern
        }

Schon in dieser frühesten Quelle wird die \cevain{facultativität} angeführt: dass sich die Raumstation \ceva{befinden muß} und aus 7 Ringen besteht, sich  also notwendig so manifestieren soll, wo sie und wie sie bereits angelegt ist (\cevain{cucunftsarchäologie}). Mit den genauen Details befasst sich die \cevain{causalitätsforc\_ung}, deren wesentliche Aussage die Identität von Vergangenheit und Zukunft ist. Entspechendes gilt für die 7 \ring{Ringe}; sie \emph{müssen }sein.
    
    \twofonts[\cite{ctour}]{Die c-base ist eine abgestürzte raumstation. das unter berlin-mitte im märkischen sand versunkene artefact wird seit 1995 von über 100 zukunftsbegeisterten experten reconstruiert. Das raumschiff besteht aus sieben ineinander geschalteten c-förmigen ringen. jeder ring ist für ganz spezifische aufgabencluster modular ausgelegt.}

Bemerkenswert an dieser Stelle ist der Ausdruck \cevain{zukuftsbegeistert}; dies ist die früheste Erwähnung der Beeinflussung von Karboneinheiten durch \cevain{c-beam}. Dazu später mehr. Hier wird auch die grundsätzliche \cevain{Modularität} erstmals festgestellt. Das  \ceva{c-booc} fasst entspechend zusammen:

    \twofonts[\cite{cbasebook}]{
    die raumstation ist c-förmig aufgebaut, bestehend aus 7 ringen. der mittelpunkt befindet sich unter dem heutigen Alexanderplatz. das erste fundstücc (artefact) mit der inschrift "`the c-base project: be future compatible"' legte den projektnamen fest und in anlehnung an die anzahl der ringe entstand die aufteilung in sieben arbeitsbereiche.
    }

Im Gegensatz zu den älteren Quellen stellt das \ceva{c-booc} heraus, dass die Aufteilung in die \ring{Ringe} nicht willkürlich geschehen ist, sondern durch das erste \ceva{artefact} enstand. Die \ceva{c-base} ist mithin ihr eigener Namensgeber und selbst Urheber ihrer eigenen Struktur (Autopoiesis).

\begin{figure}[ht!]
    \centering
    \documentclass{standalone}
\usepackage{tikz}
\usetikzlibrary{decorations, decorations.text}
\usepackage{calc}
\usepackage{xcolor}
\usepackage{fontspec}
\newcommand{\cevapic}[1]{~{\fontspec{[ceva-c2.ttf]}#1}}
\definecolor{eins}{HTML}{e7e7e8}   %% Ring 1 "core"     - weiß - Mittelpunkt, Ring um Mittelpunkt
\definecolor{zwei}{HTML}{ed1c24}   %% Ring 2 "com"      - rot -  "Fenster" innen
\definecolor{drei}{HTML}{fbad18}   %% Ring 3  "culture" - orange - fünf Module, quasi invertiert
\definecolor{vier}{HTML}{74c043}   %% Ring 4  "creactiv" - grün -  vier Module
\definecolor{fuenf}{HTML}{0089d0}  %% Ring 5 "cience"  - cyan (blau) - drei Module mit "Strich"
\definecolor{sechs}{HTML}{11357e}  %% Ring 6 "carbon" -  indigo - viele "Fenster" außen
\definecolor{sieben}{HTML}{000000} %% Ring 7 "clamp" -  schwarz, c-förmig
\definecolor{cbase}{HTML}{222222}  %% Körper der Raumstation    
\begin{document}
%% c-base logo nachgebaut von penta.
%% alles nur geschätze Winkel und Abstände :/
%% um den code zu verstehen, einfach mal einzelne Teile auskommentieren und wieder einkommentieren (ctrl-#) und dann mal \draw[white] durch \draw[red] ersetzen, dann sieht man, was was ist.
%% viel Spaß damit.
\tikzset{
  pics/carc/.style args={#1:#2:#3:#4}{
    code={
      \draw[postaction={decorate, decoration={text along path, raise=-2pt, text align={align=center}, text={\ceva{#4}}, reverse path}}] (#1:#3) arc(#1:#2:#3);
    }
  }
}%
    \begin{tikzpicture}[scale=0.4]
        \draw[gray, line width=50pt] (0:0) circle (3);
        \foreach [count=\i] \ring/\color in
            {clamp/sieben,carbon/sechs,cience/fuenf,creactiv/vier,culture/drei,com/zwei,core/white}
            {%
                \draw[fill=\color,draw=none] (0:0) circle (13-1.5*\i);
                \node [fill=\color!40,rounded corners,draw] at (-90:13-1.5*\i-0.8) {\cevapic{\ring}};
                % \draw[color=\color!50,line width=45pt] (0:0) pic{carc=\i*51.418-25:\i*51.418+25:3:\ring};
            }%
            \draw[color=white,line width=122] (0:0) pic{carc=-21:21:2.5:};
    \end{tikzpicture}
\end{document}

    \caption{Idealtypische Anordnung der 7 \ring{Ringe}}
    \label{fig:ringconvention}
\end{figure}

Die Rekonstruktion ergibt eine fragmentierte Struktur von konzentrisch verschachtelten \cevain{Ringen} mit multiplen, untereinander verschiebbaren Modulen~\cite{cbasebook}~\cite{cbasepressemap}. \cref{fig:ringconvention} zeigt die konzentrische Anordnung der \ring{Ringe} nach der kanonischen Überlieferung.
        
\section*{Materieller Befund}\addcontentsline{toc}{section}{Materieller Befund}
\fancyhead[RO]{Materieller Befund}

Der materielle Befund weicht von dem Ideal leicht ab. Einige Ringe sind \ring{Ringe} geschlossen und nur die äußeren drei (\ceva{cience}, \ceva{carbon} und \ceva{clamp}) sind \cevain{c-förmig}. Die heute gängige Darstellung ist in \cref{fig:cbaselogo} wiedergegeben.

Dies ist der rezente Befund. Der vorhergehende, und der zukünftige, kann davon abweichen. Der vergangene Zustand ist Ausdruck der zukünftigen Konfiguration, und die Zukunft ist Produkt der Vergangenheit. Darüber hinaus kann es noch weitere Zeit- und Raumdimensionen geben, die selbst diese Einsicht als \cevain{flach} darstehen lassen würden.


\begin{figure}[ht!]
    \centering
        \resizebox{0.6\textwidth}{!}{
        \documentclass{standalone}
\usepackage{tikz}
\usepackage{calc}
\usepackage{xcolor}
\begin{document}
%% c-base logo nachgebaut von penta.
%% alles nur geschätze Winkel und Abstände :/
%% um den code zu verstehen, einfach mal einzelne Teile auskommentieren und wieder einkommentieren (ctrl-#) und dann mal \draw[white] durch \draw[red] ersetzen, dann sieht man, was was ist.
%% viel Spaß damit.
\tikzset{
  pics/carc/.style args={#1:#2:#3}{
    code={
      \draw[pic actions] (#1:#3) arc(#1:#2:#3);
    }
  }
}
\definecolor{eins}{HTML}{FFFFFF}   %% Ring 1 "core"     - weiß - Mittelpunkt 
\definecolor{zwei}{HTML}{FF0000}   %% Ring 2 "com"      - rot -  Ring um Mittelpunkt
\definecolor{drei}{HTML}{FF7F00}   %% Ring 3  "culture" - orange - "Fenster" innen
\definecolor{vier}{HTML}{AAFF00}   %% Ring 4  "creativ" - grün -  fünf Module, quasi invertiert
\definecolor{fuenf}{HTML}{00FFFF}  %% Ring 5 "cience"  - blau - vier Module
\definecolor{sechs}{HTML}{800080}  %% Ring 6 "carbon" -  violett - drei Module mit "Strich"
\definecolor{sieben}{HTML}{4B0082} %% Ring 7 "clamp" -  indigo - viele "Fenster" außen

    \begin{tikzpicture}%
        \def\radi{10}%     

        %% äußerer Radius
        \filldraw[black] (0:0) circle (\radi-0.3);
        % was wegnehmen / weiß übermalen
        \draw[white, line width=1.6cm] (0:0) pic{carc=-55:90:\radi-0.72}; 

        %% clamp - viele kleine Fenster
        \foreach [count=\i] \ii in {%
        70,74,78,100,
        114,118,...,170,
        182,186,
        202,206,...,250,
        266,270,...,304}
            \draw[sieben, line width=0.5cm] (0:0) pic{carc=\ii:\ii-2:\radi-2}; 
        
        \draw[white, line width=3.05cm] (0:0) pic{carc=50:-55:\radi-3};
        
        \draw[white, line width=2cm] (0:0) pic{carc=-90:0:\radi-3.5};

  
        \draw[sechs, line width=1.8cm] (0:0) 
            pic{carc=82:135:\radi-3.5};
        \draw[sechs, line width=1.8cm] (0:0) 
            pic{carc=65:74:\radi-3.5};
        \draw[sechs, line width=1.8cm] (0:0) 
            pic{carc=53:62:\radi-3.5};
            
        \draw[black, line width=0.4cm] (0:0) 
            pic{carc=110:135:\radi-3.3};

        \foreach [count=\i] \ii in {1,2,3,4}
            \draw[fuenf, line width=1.5cm] (0:0) 
            pic{carc=90*\i:90*\i+45:\radi-5.5};


        \filldraw[vier] (0:0) circle (\radi-6.5);
        
        \foreach [count=\i] \ii in {1,2,3,4,5}
            \draw[black, line width=0.7cm] (0:0) 
            pic{carc=72*\i+27:72*\i+36+27:\radi-7};

        \filldraw[black] (0:0) circle (\radi-7.5);

        \foreach [count=\i] \ii in {160,172,...,490}
            \draw[drei, line width=0.4cm] (0:0) pic{carc=\ii:\ii-8:\radi-8};   
        
        \filldraw[zwei] (0:0) circle (\radi-8.5);
        \filldraw[black] (0:0) circle (\radi-8.8);
        \filldraw[eins] (0:0) circle (\radi-9.8);
     
    \end{tikzpicture}
\end{document}

    }
    \caption{Faktische Anordnung der 7 \ring{Ringe}}
    \label{fig:cbaselogo}
\end{figure}


    Die sieben  \ring{Ringe} werden in den kanonischen Texten von innen nach außen gezählt, bezeichnet und  farblich codiert  wie in  \cref{tab:ringe} aufgeführt. Das \ceva{c-booc} nennt die Farben nur verbal~\cite[S.49]{cbasebook}. Ob diese Farbnamen im \ceva{c-booc} zur Verwendung von Primärfarben verpflichten, oder ob abgetönte und den jeweiligen Druckverhältnissen angepasste Farben verwendet werden dürfen, hat bereits zu Dogmenstreit geführt. Gegenüber dem \ceva{c-booc} liegt mit \cite{cbasefarbschema} allerdings eine ältere Quelle vor, die leicht getönte Farben nennt. Deswegen übernehmen wir hier die Position der \ceva{abtöner}\footnote{Das \ceva{c-booc} liefert den \cevain{clogan} "`\cevain{jawohl mein designer}"'~\cite[S. 47]{cbasebook}; dies wird in der herrschenden Lehrmeinung dahingehend interpretiert, dass es nicht dem Individuum obliegt, frei zu wählen, ob es \cevain{purist} oder \cevain{abtöner} wird. Vielmehr ist dies eine Frage der Prädestination. Und eben deswegen ist die Diskussion darüber schlussendlich müßig, da sie übermenschliche Entscheidungen betrifft.}.

    \begin{table}[ht!]
        \centering
        \begin{tabular}{rlllrr}
            \toprule
                1 & \ceva{core} & core & grau / weiß & \texttt{e7e7e8} & \Hrulek[eins]  \\
                2 & \ceva{com} & com & rot & \texttt{ed1c24} & \Hrulek[zwei] \\
                3 & \ceva{culture} & culture & orange & \texttt{fbad18} & \Hrulek[drei] \\
                4 & \ceva{creactive} & creactive & grün & \texttt{75c043}& \Hrulek[vier]  \\
                5 & \ceva{cience} & cience & cyan & \texttt{0089d0}& \Hrulek[fuenf]  \\
                6 & \ceva{carbon} & carbon & indigo & \texttt{0089d0}& \Hrulek[sechs]  \\
                7 & \ceva{clamp} & clamp  & ultraviolett / schwarz & \texttt{000000}& \Hrulek[sieben] \\
            \bottomrule
        \end{tabular}
        \caption{Nummern, Bezeichnung und Farbe der Ringe}
        \label{tab:ringe}
    \end{table}

    % Die Ringfarbe korrespondiert nur schwach mit den Signalfarben innerhalb der Station, die wir hier der Vollständigkeit  halber in \cref{tab:bedeutungen} zeigen (nach~\cite[S. 56]{cbasebook}).
    
    % \begin{table}[ht!]
    %     \centering
    %     \begin{tabular}{lll}
    %         \toprule
    %             grau / weiß  & \Hrulek[eins] & überlebensstandard gesichert, temperature, drucc \\
    %             rot & \Hrulek[zwei] & lebendig, alarm, communication \\
    %              orange  & \Hrulek[drei] & schädlich, aktiver prozess auf atomarer ebene \\
    %             grün & \Hrulek[vier]  & nicht humane biologische substanz, prozess \\
    %             cyan & \Hrulek[fuenf] & lowered thermal conditions \\
    %             indigo & \Hrulek[sechs] & \textit{(not assigned)}  \\
    %             schwarz & \Hrulek[sieben] & vacuum, death, hazard\\
    %         \bottomrule
    %     \end{tabular}
    %     \caption{Signalbedeutungen der Farben}
    %     \label{tab:bedeutungen}
    % \end{table}


    
    Aufgrund der multimodularen und letztlich höherdimensionalen Struktur der Station ist eine eineindeutige und ausschließliche Zuordnung einzelner Module zu bestimmten \ring{Ringen} fragwürdig. Erschwerend kommt hinzu, dass Raum und Zeit innerhalb der Station unregelmäßig gefaltet sind: 

    \twofonts[\cite{cbasepressemap}]{
    Die c-base wird in der Zukunft als orbitales Generationsschiff auf der Erde gebaut und dient dem Terraforming anderer Planeten. [...]
    
    % Die Fortbewegung im Raum erfolgt mittels eines vom Cybernetischen-Quecksilber-Reaktor (CQR) erzeugten Feldes, das die c-base aus der Raumzeit ausschneidet, um sie über die Einstein-Rosenbaum-Brücke im Orbit des zu formenden Planeten materialisieren zu lassen. 
    
    Bei der Berechnung der Eingangsgrössen kam es zu einem Flip-Flop der Asi\-mov-Konstante. Anstatt des Raumes wurde die Zeit gefaltet, die c-base reiste 4,5~Milliarden Jahre in die Vergangenheit statt vorwärts in den Raum und crashte aus noch nicht restlos geklärten Gründen auf die Erdoberfläche, wo sie langsam im märkischen Sand versank. 
    }

Der Begriff der \cevain{Faltung} von Raum und Zeit wird uns wieder begegnen. Tatsächlich ist nur dank Raum-Zeit-Faltung Wiederbegegnung überhaupt möglich. Mitunter kommt es bei solchen Faltungen allerdings zu Dimensionsdurchgängen wie z.B. dem \cevain{raumceitloch}:

    \twofonts[\cite{raumceitloch}]{das c-base raum-ceit-loch ist ein allgegenwärtigec ctationcphänomen, dac alle deccc erfaCt. man fängt es sich cwic\_endurch ein und c\_leppt es durch die module. das cbrcl ist eine paradoxonentfaltung im euclidic\_en raum, dac beconderc bei hyperactivität und concolidierter bewegungslocigceit cur wircung commt.}

Hier wird von \cevain{deccs} zusätzlich zu den \ring{Ringen} gesprochen. Die meisten Exe\-ge\-ten nehmen eine Identität an: $\text{\ceva{decc}} \equiv \text{\ceva{ring}}$.

Wir halten fest: 

\textbf{Die \ring{Ringe} sind Aspekte oder Funktionen, die ineinander verschachtelt sind, miteinander kommunizieren, einander brauchen und sich gegenseitig stärken, und erst in zweiter Linie  Orte im Raum.} 

Sie sind \textbf{nicht} einander ausschließende Kategorien.
   
Die kanonische Literatur über diese \ring{Ringe} wird im Folgenden wiedergegeben und die Funktion der \ring{Ringe} anschließend beschrieben. 

Da unser Fokus auf der Funktion innerhalb des autopoietischen Systems \ceva{c-base} liegt, wird auf eine detaillierte Beschreibung einzelner Projekte, Module, Artefakte, Module und Bewohner wird verzichtet. 
    
Eine sehr gute Übersicht über ausgewählte einzelne Projekte, Aggregate, Artefakte und Module bietet das kanonische \ceva{c-booc}~\cite{cbasebook}. Zur Funktion, Geschichte und Ausgrabungsgeschichte der Station und des Rolle des Vereins c-base e.V. sei auf die \ceva{Pressemappe}~\cite{cbasepressemap} verwiesen.