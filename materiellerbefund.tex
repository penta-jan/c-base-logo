\section{Materieller Befund}%\addcontentsline{toc}{section}{Materieller Befund}
\fancyhead[RO]{Materieller Befund}%

Die Rekonstruktion ergibt eine fragmentierte Struktur von konzentrisch verschachtelten \cevain{ringen} mit multiplen, untereinander verschiebbaren Modulen ~\cite{cbasebook}~\cite{cbasepressemap}. 
Der materielle Befund weicht von dem Ideal (vgl.~\cref{fig:ringconvention}) leicht ab. Die inneren vier \ring{ringe}, nämlich \ceva{core}, \ceva{com}, \ceva{culture} und \ceva{creactive}, sind geschlossene Kreise. Die äußeren drei, (\ceva{cience}, \ceva{carbon} und \ceva{clamp}), sind teiloffen, wobei insbesondere \ceva{clamp} \cevain{c-förmig} ausgeführt ist. Die heute gängige Darstellung ist in \cref{fig:cbaselogo} wiedergegeben.

Dies ist der rezente Befund. Der vorhergehende, und der zukünftige, kann davon abweichen. Der vergangene Zustand ist Ausdruck der zukünftigen Konfiguration, und die Zukunft ist Produkt der Vergangenheit. Darüber hinaus kann es noch weitere Zeit- und Raumdimensionen geben, die selbst diese Einsicht als \cevain{flach} darstehen lassen würden (vgl. \cref{sec:topologie}).


\begin{figure}[ht!]
    \centering
        \resizebox{0.6\textwidth}{!}{
        \documentclass{standalone}
\usepackage{tikz}
\usepackage{calc}
\usepackage{xcolor}
\begin{document}
%% c-base logo nachgebaut von penta.
%% alles nur geschätze Winkel und Abstände :/
%% um den code zu verstehen, einfach mal einzelne Teile auskommentieren und wieder einkommentieren (ctrl-#) und dann mal \draw[white] durch \draw[red] ersetzen, dann sieht man, was was ist.
%% viel Spaß damit.
\tikzset{
  pics/carc/.style args={#1:#2:#3}{
    code={
      \draw[pic actions] (#1:#3) arc(#1:#2:#3);
    }
  }
}
\definecolor{eins}{HTML}{FFFFFF}   %% Ring 1 "core"     - weiß - Mittelpunkt 
\definecolor{zwei}{HTML}{FF0000}   %% Ring 2 "com"      - rot -  Ring um Mittelpunkt
\definecolor{drei}{HTML}{FF7F00}   %% Ring 3  "culture" - orange - "Fenster" innen
\definecolor{vier}{HTML}{AAFF00}   %% Ring 4  "creativ" - grün -  fünf Module, quasi invertiert
\definecolor{fuenf}{HTML}{00FFFF}  %% Ring 5 "cience"  - blau - vier Module
\definecolor{sechs}{HTML}{800080}  %% Ring 6 "carbon" -  violett - drei Module mit "Strich"
\definecolor{sieben}{HTML}{4B0082} %% Ring 7 "clamp" -  indigo - viele "Fenster" außen

    \begin{tikzpicture}%
        \def\radi{10}%     

        %% äußerer Radius
        \filldraw[black] (0:0) circle (\radi-0.3);
        % was wegnehmen / weiß übermalen
        \draw[white, line width=1.6cm] (0:0) pic{carc=-55:90:\radi-0.72}; 

        %% clamp - viele kleine Fenster
        \foreach [count=\i] \ii in {%
        70,74,78,100,
        114,118,...,170,
        182,186,
        202,206,...,250,
        266,270,...,304}
            \draw[sieben, line width=0.5cm] (0:0) pic{carc=\ii:\ii-2:\radi-2}; 
        
        \draw[white, line width=3.05cm] (0:0) pic{carc=50:-55:\radi-3};
        
        \draw[white, line width=2cm] (0:0) pic{carc=-90:0:\radi-3.5};

  
        \draw[sechs, line width=1.8cm] (0:0) 
            pic{carc=82:135:\radi-3.5};
        \draw[sechs, line width=1.8cm] (0:0) 
            pic{carc=65:74:\radi-3.5};
        \draw[sechs, line width=1.8cm] (0:0) 
            pic{carc=53:62:\radi-3.5};
            
        \draw[black, line width=0.4cm] (0:0) 
            pic{carc=110:135:\radi-3.3};

        \foreach [count=\i] \ii in {1,2,3,4}
            \draw[fuenf, line width=1.5cm] (0:0) 
            pic{carc=90*\i:90*\i+45:\radi-5.5};


        \filldraw[vier] (0:0) circle (\radi-6.5);
        
        \foreach [count=\i] \ii in {1,2,3,4,5}
            \draw[black, line width=0.7cm] (0:0) 
            pic{carc=72*\i+27:72*\i+36+27:\radi-7};

        \filldraw[black] (0:0) circle (\radi-7.5);

        \foreach [count=\i] \ii in {160,172,...,490}
            \draw[drei, line width=0.4cm] (0:0) pic{carc=\ii:\ii-8:\radi-8};   
        
        \filldraw[zwei] (0:0) circle (\radi-8.5);
        \filldraw[black] (0:0) circle (\radi-8.8);
        \filldraw[eins] (0:0) circle (\radi-9.8);
     
    \end{tikzpicture}
\end{document}

    }
    \caption{Faktische Anordnung der 7 \ring{ringe}}
    \label{fig:cbaselogo}
\end{figure}


    Die sieben  \ring{ringe} werden in den kanonischen Texten von innen nach außen gezählt, bezeichnet und  farblich codiert wie in  \cref{tab:ringe} aufgeführt. Das \ceva{c-booc} nennt die Farben nur verbal~\cite[S.49]{cbasebook}. Ob diese Farbnamen im \ceva{c-booc} zur Verwendung von Primärfarben verpflichten (wie die \cevain{puristen} glauben), oder ob abgetönte und den jeweiligen Druckverhältnissen angepasste Farben verwendet werden dürfen (Position der \cevain{abtöner}), hat bereits zu erbitterten Auseinandersetzungen geführt. 
    
    Allerdings liegt mit  \cite{cbasefarbschema}  eine ältere Quelle vor, die leicht getönte Farben nennt. Wir folgen hier der Position der \cevain{abtöner}, die glauben, dass diese sieben Farben letztendlich nur Projektionen aus einem höherdimensionalen Farbraum jenseits menschlicher Wahrnehmung sind, und jede solche Projektion notwendig ihre Grenzen hat (vgl.~\cref{sec:mathematik}). 
    
    Aus der verbalen Überlieferung \cite[S. 47]{cbasebook} stammt der wohln noch ältere \cevain{clogan} \linekin{jawohl$\cdot$mein$\cdot$designer}. Dieser wird in der herrschenden Lehrmeinung dahingehend interpretiert, dass es nicht dem Individuum obliegt, frei zu wählen, ob es \cevain{purist} oder \cevain{abtöner} wird. Vielmehr ist dies eine Frage der Prädestination. Und eben deswegen ist die Diskussion darüber schlussendlich müßig, da sie übermenschliche Entscheidungen betrifft.

    \begin{table}[ht!]
        \centering
        \begin{tabular}{rlllrr}
            \toprule
                1 & \ceva{core} & core & grau / weiß & \texttt{e7e7e8} & \Hrulek[eins]  \\
                2 & \ceva{com} & com & rot & \texttt{ed1c24} & \Hrulek[zwei] \\
                3 & \ceva{culture} & culture & orange & \texttt{fbad18} & \Hrulek[drei] \\
                4 & \ceva{creactive} & creactive & grün & \texttt{75c043}& \Hrulek[vier]  \\
                5 & \ceva{cience} & cience & cyan & \texttt{0089d0}& \Hrulek[fuenf]  \\
                6 & \ceva{carbon} & carbon & indigo & \texttt{0089d0}& \Hrulek[sechs]  \\
                7 & \ceva{clamp} & clamp  & ultraviolett / schwarz & \texttt{000000}& \Hrulek[sieben] \\
            \bottomrule
        \end{tabular}
        \caption{Nummern, Bezeichnungen und Farben der Ringe}
        \label{tab:ringe}
    \end{table}

    % Die Ringfarbe korrespondiert nur schwach mit den Signalfarben innerhalb der Station, die wir hier der Vollständigkeit  halber in \cref{tab:bedeutungen} zeigen (nach~\cite[S. 56]{cbasebook}).
    
    % \begin{table}[ht!]
    %     \centering
    %     \begin{tabular}{lll}
    %         \toprule
    %             grau / weiß  & \Hrulek[eins] & überlebensstandard gesichert, temperature, drucc \\
    %             rot & \Hrulek[zwei] & lebendig, alarm, communication \\
    %              orange  & \Hrulek[drei] & schädlich, aktiver prozess auf atomarer ebene \\
    %             grün & \Hrulek[vier]  & nicht humane biologische substanz, prozess \\
    %             cyan & \Hrulek[fuenf] & lowered thermal conditions \\
    %             indigo & \Hrulek[sechs] & \textit{(not assigned)}  \\
    %             schwarz & \Hrulek[sieben] & vacuum, death, hazard\\
    %         \bottomrule
    %     \end{tabular}
    %     \caption{Signalbedeutungen der Farben}
    %     \label{tab:bedeutungen}
    % \end{table}


    
    Aufgrund der multimodularen und letztlich höherdimensionalen Struktur der Station ist eine eineindeutige und ausschließliche Zuordnung einzelner Module zu bestimmten \ring{ringen} fragwürdig. Erschwerend kommt hinzu, dass Raum und Zeit innerhalb der Station unregelmäßig gefaltet sind: 
    \twofonts[\cite{cbasepressemap}]{
    Die c-base wird in der Zukunft als orbitales Generationsschiff auf der Erde gebaut und dient dem Terraforming anderer Planeten. [...]
    
    % Die Fortbewegung im Raum erfolgt mittels eines vom Cybernetischen-Quecksilber-Reaktor (CQR) erzeugten Feldes, das die c-base aus der Raumzeit ausschneidet, um sie über die Einstein-Rosenbaum-Brücke im Orbit des zu formenden Planeten materialisieren zu lassen. 
    
    Bei der Berechnung der Eingangsgrössen kam es zu einem Flip-Flop der Asi\-mov-Konstante. Anstatt des Raumes wurde die Zeit gefaltet, die c-base reiste 4,5~Milliarden Jahre in die Vergangenheit statt vorwärts in den Raum und crashte aus noch nicht restlos geklärten Gründen auf die Erdoberfläche, wo sie langsam im märkischen Sand versank. 
    }

Der Begriff der \cevain{faltung} von Raum und Zeit wird uns wieder begegnen. Tatsächlich ist nur dank Raum-Zeit-Faltung Wiederbegegnung überhaupt möglich. Mitunter kommt es bei solchen Faltungen allerdings zu Dimensionsdurchgängen wie z.B. dem \cevain{raumceitloch}:
    \twofonts[\cite{raumceitloch}]{das c-base raum-ceit-loch ist ein allgegenwärtigec ctationcphänomen, dac alle deccc erfaCt. man fängt es sich cwic\_endurch ein und c\_leppt es durch die module. das cbrcl ist eine paradoxonentfaltung im euclidic\_en raum, dac beconderc bei hyperactivität und concolidierter bewegungslocigceit cur wircung commt.}

Hier wird von \cevain{deccs} zusätzlich zu den \ring{ringen} gesprochen. Die meisten Exe\-ge\-ten nehmen eine Identität an: $\text{\ceva{decc}} \equiv \text{\ceva{ring}}$.

Eine unerwartete Bestätigung des Absturzberichts erbrachte vor einigen Jahren das Auffinden eines verloren geglaubten Teils des \emph{Teppichs von Bayeux} aus dem Europäischen Mittelalter mit einer Darstellung der \ceva{c-tation} und der Inschrift \emph{hic statio spatialis terram caedit}.\footnote{Lat. \emph{An dieser Stelle stürzte [eine] Raumstation zur Erde.}}

% Wir halten fest: 
