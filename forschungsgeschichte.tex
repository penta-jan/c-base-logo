\section*{Forschungsgeschichte}\addcontentsline{toc}{section}{Forschungsgeschichte}

    Die \ceva{c-base} ist eine abgestürzte Raumstation unter Berlin, die sich seit 1995 rekonstruiert~\cite{cbasebook}. Bislang fanden folgende Rekonstruktionsphasen statt \cite{cbasepressemap} \cite{cbasebook}:
    \begin{enumerate}
        \item Phase I.  Februar 1995 -- Mai 2000. Nachbau und Rekonstruktion einer Schleusensektion der \ceva{c-base} Raumstation auf 270$m^2$ Fläche. Innenräume des \mbox{c-base} e.V. in der Oranienburger Str. 2. \cite{cbasepressemap} \cite{cbasebook}
        \item Phase II.   Juni 2000 -- August 2002. Nachbau und Rekonstruktion der Multimodulstation RS20 der \ceva{c-base} Raumstation auf 524$m^2$ Fläche. Neue Innenräume des \mbox{c-base} e.V. in der Rungestr. 20. \cite{cbasepressemap} \cite{cbasebook}
        \item Zwischendeck.  September 2002 -- Juli 2003. Auslagerung [...] zwecks Wartungsarbeiten in der RS20 an den Franz-Mehring-Platz Nr. 1. \cite{cbasepressemap} \cite{cbasebook}
        \item Phase III.  August 2003 -- heute. Erweiterung der Raumstationsfläche auf ca. 720$m^2$ auf 2 Etagen in der Multimodulstation RS20. Mainhall- und Brückenreconstruction, neue Schleusensektion, Ausbau \cevain{c-level} ... work in progress. \cite{cbasepressemap} \cite{cbasebook}
    \end{enumerate}

Die Ergebnisse dieser Rekonstruktionen fanden ihren Niederschlag in diversen Publikationen (siehe Literaturliste auf S. \pageref{sec:literatur}), deren wichtigste wir im Folgenden auswerten. So nicht weiter angegeben sind die Primärtexte in der Reihenfolge ihres vermutlichen Entstehens zitiert, nämlich \cite{cbasewebsite} $\rightarrow$ \cite{cbasestarbasemanual} und dann, abgesetzt im Fließtext, $\rightarrow$  \cite{cbasebook}. Die übrigen Quellen können als abgeleitet oder unvollständig gelten, bieten aber doch mitunter brauchbare Hinweise; sie werden zu einzelnen Details hinzugezogen.
