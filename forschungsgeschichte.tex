\section*{Forschungsgeschichte}\addcontentsline{toc}{section}{Forschungsgeschichte}
\fancyhead[RO]{Forschungsgeschichte}
\fancyhead[LO]{}

    Die \ceva{c-base} ist eine abgestürzte Raumstation unter Berlin, die sich seit 1995 rekonstruiert~\cite{cbasebook}, ein als \cevain{cbrp} - \cevain{c-base reconstruction project} bekannter Prozess. Folgende Rekonstruktionsphasen lassen sich unterscheiden~\cite{cbasepressemap}~\cite{cbasebook}:
    % \twofonts{
    \begin{enumerate}
        \item Phase I.  Februar 1995 -- Mai 2000 Oranienburger Str. 2.  Nachbau und Rekonstruktion einer Schleusensektion der \ceva{c-base} Raumstation auf 270$m^2$.~\cite{cbasepressemap}~\cite{cbasebook} 
        \item Phase II.   Juni 2000 -- August 2002 Rungestr. 20.  Nachbau und Rekonstruktion der Multimodulstation \cevain{RS20} der \ceva{c-base} Raumstation auf 524$m^2$.~\cite{cbasepressemap}~\cite{cbasebook} 
        \item Zwischendeck.  September 2002 -- Juli 2003 Franz-Mehring-Platz Nr. 1. 
        Auslagerung wegen Wartungsarbeiten in der \ceva{RS20}~\cite{cbasepressemap}~\cite{cbasebook} 
        \item Phase III.  August 2003 -- heute. Rungestr. 20. 
         Erweiterung der Fläche auf ca. 720$m^2$ auf 2 Etagen. Fortgesetzte Konstuktion.
         % in der Multimodulstation RS20. Mainhall- und Brückenrekonstruktion, neue Schleusensektion, Ausbau \cevain{c-level} ... work in progress.~\cite{cbasepressemap}~\cite{cbasebook} 
    \end{enumerate}
    % }

Die Ergebnisse dieser Rekonstruktionen fanden ihren Niederschlag in diversen Publikationen (siehe Literaturliste auf S.~\pageref{sec:literatur}), deren Wichtigste wir im Folgenden auswerten. 
