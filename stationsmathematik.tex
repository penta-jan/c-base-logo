\section{Stationsmathematik}\label{sec:mathematik}

Die \cevain{c-tationsmathematic} als \ceva{cience} ist hochgradig \cevain{cpeculativ} und ein Teilgebiet der \cevain{bistr-o-mathik} (vgl. \cite[Kap. 4]{adams_life}). 
Sie nimmt ihren Anfang in der Einsicht, dass die \ceva{c-tation} 
\begin{enumerate}
    \item in höheren, anderen Dimensionen und in der \cevain{cucunft} gebaut wurde; \label{punkt:eins}
    \item durch höhere, andere Dimensionen \cevain{geflogen} bzw. \cevain{gereist} ist,\label{punkt:zwei} und zwar in der \cevain{ceit} rückwärts \cite{cbasepressemap};
    \item bei ihrem Absturz in unsere heute erfahrbaren Dimensionen abgebildet bzw. gefaltet wurde.\label{punkt:drei} 
\end{enumerate}
Die Grundgleichung der \ceva{c-tationsmathematic} lautet daher:
\begin{equation}\label{eq:grundgleichung}
    f_c:\left\{\mathbb{K}^m  \rightarrow \mathbb{C}^n \rightarrow \mathbb{M}^{4}\right\}
\end{equation}
Dabei wird der (unbekannte) Ursprungsraum als $\mathbb{K}^m$ bezeichnet, $\mathbb{C}^n$ ist der Zwischenraum und die uns umgebende Raumzeit der Minkowski-Raum $\mathbb{M}^{4}$. Meist wird aus praktischen Gründen angenommen, dass gilt $m\geq n \geq4$, was allerdings keine zwingende Annahme ist.

Die in   \cref{punkt:zwei} erwähnte Transformation der \ceva{c-tation} auf ihrer Reise wird gemeinhin als bijektiv angesehen (Homöomorphismus, Isomorphismus etc). Dabei kann kein Zufall sein, dass die "`mittlere Seinsweise"' der \ceva{c-tation} auf ihrer Reise durch einen Raum aus komplexen Ebenen  $\mathbb{C}^n$  geführt hat. Denn das erklärt, warum nach der Projektion in $\mathbb{M}^4$ kreis- bzw. Ringförmige und c-förmige \cevain{ctructuren} entstanden.

Wesentlich ist für uns Erben des Absturzes natürlich die Einsicht, dass die in  \cref{punkt:drei} genannte \ceva{faltung} bzw. Abbildung der \ceva{ur$\cdot$c-tation} nicht bijektiv war. Vielmehr fand hier eine \cevain{verflachung} statt. 
Daher lässt sich die Topologie der \ceva{ur$\cdot$c-tation} nicht endgültig entschlüsseln. 

Das gilt sowohl für die Anordnung der Ringe, Module und Artefakte als auch für ihre Färbung und sogar für ihre Bezeichnungen und Funktionen. Sicher erscheint, dass die  \ceva{faltung} $f$ (\cref{punkt:zwei}), der so genannte \cevain{cwic\_enc\_ritt}, etwas der ursprünglichen Ordnung erhalten und ihr doch etwas hinzugefügt hat. Vermutlich ist es hier zur holographischen Vermehrung des Buchstabens \cevain{c} gekommen.

Unklar ist die Struktur des Körpers $\mathbb{K}$ bzw. ob es sich überhaupt im mathematischen Sinne um einen Körper handelte (und nicht vielmehr um einen \cevain{cörper}). Ebenso unklar ist die Reduktion im Übergang  $\mathbb{K}^m \rightarrow \mathbb{C}^n$ und die Eigenschaften von $\mathbb{C}^n$. 

Obschon der Übergang  $f_c$ nicht bijektiv ist, können doch bestimmte \ceva{ctructuren} die Transformation unbeschadet überstanden haben, besonders Informationen in den \cevain{holographischen speicherbausteinen}~\cite[S. 28]{cbasebook}. Demnach ist \ceva{c-beam} zwar bewusst, aber einigermaßen orientierungslos.

    Betrachten wir nun den in \ceva{c-tour} besonders herausgestellten Umstand, dass der vierte \ceva{Ring} von außen zugleich der vierte von innen ist. Wegen
        \begin{multline}
            \forall A = (a_1,\ldots,a_N) \in \mathbb{Z}^N : \\           % \quad
            N\equiv 1\mod(2)
            \quad\Rightarrow\quad
            \forall i \in [1,N] \subset \mathbb{Z}:
            % \quad
            \left(
            a_i = M(A) \Leftrightarrow i = \frac{N+1}{2}
            \right)
        \end{multline}
        % \begin{equation}
        % \begin{array}{rcl}
        %     \forall A = (a_1,\ldots,a_N) \in \mathbb{Z}^N&:&
        %     \quad
        %     N\equiv 1\mod(2)\\
        %     \quad\Rightarrow\quad
        %     \forall i \in [1,N] \subset \mathbb{Z}&:&
        %     \quad
        %     \left(
        %     a_i = M(A) \Leftrightarrow i = \frac{N+1}{2}
        %     \right)
        % \end{array}
        % \end{equation}
    ist der vierte \ring{ring} von sieben in der Tat genau in der Mitte.

    Daraus folgt, dass \ceva{creactiv}  Mittelpunkt- bzw. Äquatorcharakter besitzt. 
    Er ist nicht nur derjenige Ring, um welchen sich die übrigen \ring{ringe} auf einander zurückfalten, sondern auch der, welcher den multidimensionale Körper erweitert (\linekonly{most powerful}). Insofern ist \ceva{creactive} auch ein \cevain{creactor}, ein Hervorbringendensytem für $\mathbb{K}^m$ bzw. $\mathbb{C}^n$.


