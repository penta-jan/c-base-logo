\documentclass{standalone}
\usepackage{tikz}
\usepackage{calc}
\begin{document}
%% c-base logo nachgebaut von penta.
%% alles nur geschätze Winkel und Abstände :/
%% um den code zu verstehen, einfach mal einzelne Teile auskommentieren und wieder einkommentieren (ctrl-#) und dann mal \draw[white] durch \draw[red] ersetzen, dann sieht man, was was ist.
%% viel Spaß damit.
\tikzset{
  pics/carc/.style args={#1:#2:#3}{
    code={
      \draw[pic actions] (#1:#3) arc(#1:#2:#3);
    }
  }
}
    \begin{tikzpicture}%
        \def\radi{10}%     
        \filldraw[black] (0:0) circle (\radi-0.3);
        \draw[white, line width=1.6cm] (0:0) pic{carc=-55:90:\radi-0.72}; 
        \foreach [count=\i] \ii in {%
        70,74,78,100,
        114,118,...,170,
        182,186,
        202,206,...,250,
        266,270,...,304}
            \draw[white, line width=0.5cm] (0:0) pic{carc=\ii:\ii-2:\radi-2}; 
        
        \draw[white, line width=3.05cm] (0:0) pic{carc=50:-55:\radi-3};
        
        \draw[white, line width=2cm] (0:0) pic{carc=-90:0:\radi-3.5};

  
        \draw[white, line width=1.8cm] (0:0) 
            pic{carc=82:135:\radi-3.5};
        \draw[white, line width=1.8cm] (0:0) 
            pic{carc=65:74:\radi-3.5};
        \draw[white, line width=1.8cm] (0:0) 
            pic{carc=53:62:\radi-3.5};
            
        \draw[black, line width=0.4cm] (0:0) 
            pic{carc=110:135:\radi-3.3};

        \foreach [count=\i] \ii in {1,2,3,4}
            \draw[white, line width=1.5cm] (0:0) 
            pic{carc=90*\i:90*\i+45:\radi-5.5};


        \filldraw[white] (0:0) circle (\radi-6.5);
        
        \foreach [count=\i] \ii in {1,2,3,4,5}
            \draw[black, line width=0.7cm] (0:0) 
            pic{carc=72*\i+27:72*\i+36+27:\radi-7};

        \filldraw[black] (0:0) circle (\radi-7.5);

        \foreach [count=\i] \ii in {160,172,...,490}
            \draw[white, line width=0.4cm] (0:0) pic{carc=\ii:\ii-8:\radi-8};   
        
        \filldraw[white] (0:0) circle (\radi-8.5);
        \filldraw[black] (0:0) circle (\radi-8.8);
        \filldraw[white] (0:0) circle (\radi-9.8);
     
    \end{tikzpicture}
\end{document}
