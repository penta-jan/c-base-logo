\section*{Methodik der Qullenwidergabe}\addcontentsline{toc}{section}{Methodik der Qullenwidergabe}

Die \ceva{c-base} verwendet eine eigene Schreibweise namens  \cevain{clang}; dabei wird auf Kapitalisierung verzichtet,  Sibilanten und der stimmlose velare Plosiv (\emph{k}) werden mehr oder weniger ausschließlich durch den Buchstaben \cevain{c} ersetzt; Ähnliches gilt für daraus gebildete  Konsonantencluster.\footnote{Eine genaue Übersicht über die Ersetzungsregeln bietet \cite[S.~46]{cbasebook}.} In unseren eigenen Texten verzichten wir darauf.

Die \ceva{c-base} kennt eine eigene Schrift namens \cevain{ceva}. Wir nutzen sie, um Primärquellen und Stationsfachbegriffe als solche auszuweisen, und zwar unabhängig von der im Original verwendeten Schrift. 

Lange Zitate wiederholen wir zur verbesserten Lesbarkeit in lateinischen Buchstaben. Die Originalschreibung (mit \ceva{clang}) wurde beibehalten; nur offensichtliche Tippfehler wurden korrigiert. 

So nicht weiter angegeben, sind die Primärtexte in der Reihenfolge ihres vermutlichen Entstehens zitiert, nämlich \cite{cbasewebsite} $\rightarrow$~\cite{cbasestarbasemanual} und dann, abgesetzt im Fließtext, $\rightarrow$~\cite{cbasebook}. Die übrigen Quellen können als abgeleitet oder unvollständig gelten, bieten aber doch mitunter brauchbare Hinweise; sie werden zu einzelnen Details hinzugezogen.
 
Fachbegriffe aus der Stationssprache stehen auch innerhalb des Haupttextes in \ceva{\mbox{ceva}}. Beim ersten Auftreten geben wir eine lateinische Umschrift an: \cevain{cwelle}.




